%0       1         2         3         4         5         6         7         8
%2345678901234567890123456789012345678901234567890123456789012345678901234567890
%=======================================================================
\documentclass{book}
%=======================================================================

\newcommand{\req}[3]{
\begin{tabular}{rl}
ID & \textbf{R#1} \\
Status & \textsf{#2} \\
Acceptance & \textit{#3}
\end{tabular}
}


% INCLUDE DEVELOPMENT TEXT

\newcommand{\devel}[1]{\textbf{#1}}

% EXCLUDE DEVELOPMENT TEXT

% \newcommand{\devel}[1]{}


%=======================================================================
% Document layout
%=======================================================================

\setlength{\topmargin}{0.0in}
\setlength{\oddsidemargin}{0.0in}
\setlength{\evensidemargin}{0.0in}
\setlength{\textwidth}{6.0in}
\setlength{\textheight}{9.0in}

%=======================================================================
% Packages
%=======================================================================

\usepackage{wasysym}
\usepackage{epsfig}
\usepackage{url}

%=======================================================================
% Commands
%=======================================================================

\newcommand{\cello}{\textsf{Cello}}
\newcommand{\enzo}{\textsf{Enzo}}
\newcommand{\lcaperf}{\textsf{lcaperf}}
\newcommand{\lcatest}{\textsf{lcatest}}

\newcommand{\code}[1]{\textsf{#1}}

\newcommand{\note}[1]{\devel{\eighthnote\ \textit{#1} \\}}
\newcommand{\pargraph}[1]{\devel{\P\ \textbf{#1} \\}}

\newcommand{\todo}{\devel{$\circ$}}
\newcommand{\done}{\devel{$\bullet$}}
\newcommand{\halfdone}{\devel{\textcolor{gray}{$\bullet$}}}

\newcommand{\PROJECT}{\cello}

\newcommand{\TITLE}[3]{
\title{ {\huge \PROJECT\ #1}  \\ \vspace{0.1in}
     {\small Document Version: \textbf{#3}} \vspace{-0.1in}
    }
\author{      #2 \\
        Laboratory for Computational Astrophysics\\
        University of California, San Diego}
\maketitle}

%=======================================================================


%=======================================================================

\begin{document}

%=======================================================================
\TITLE{\cello\ Software Requirements Specification}{James Bordner}{$Rev$}
%=======================================================================

\tableofcontents

%=======================================================================
\chapter{Introduction} \label{c:intro}
%=======================================================================

\section{Purpose}
\section{Scope}
\section{Definitions, acronyms, abbreviations}
\section{References}
\section{Overview}

%=======================================================================
\chapter{Overall Description} \label{c:descr}
%=======================================================================
\section{Product perspective}
\section{Product functions}
\section{User characteristics}
\section{Constraints}
\section{Assumptions and dependencies}

%=======================================================================
\chapter{Specific Requirements} \label{c:req}
%=======================================================================

%=======================================================================
\section{System}
%=======================================================================

%-----------------------------------------------------------------------
\req
  {}
  {Proposed}
%
  {All software in the system will be open source}

%-----------------------------------------------------------------------
\req
  {}
  {Proposed}
%
  {}


%-----------------------------------------------------------------------
\req
  {}
  {Proposed}
%
  {Support for running multiple simulations
  [quantitative limit?]}

%-----------------------------------------------------------------------
\req
  {}
  {Proposed}
%
  {Al initial and boundary conditions should be explicitly
  specified by the user [as opposed to in \enzo\ where
  initial and boundary conditions are hard coded in ``problems'',
  such as ``shock tube'' or ``cosmology''].}

%-----------------------------------------------------------------------
\req
  {}
  {Proposed}
%
  {The simulation's initial and boundary condition specification
   should be sufficiently powerful to set up all known test problems
   in the literature that can be run given the supported physics
   capabilities. [Revise: how can this be tested?]}

%-----------------------------------------------------------------------
\req
  {}
  {Proposed}
%
  {The simulation input shall include ``deep'' parameter control of
    all supported physics and parallel data-structures. [The purpose
      is not to have undocumented compile-time parameters set with
      \code{\#define}'s scattered throughout the code, but still let
      users adjust low-level parameters if they want.]}

%-----------------------------------------------------------------------
\req
  {}
  {Proposed}
%
  {All standard test problems should be directly supported.  [List
  specific test problems]}

%-----------------------------------------------------------------------
\req
  {}
  {Proposed}
%
  {All problems are specified (specifiable) using a single user-written
   text file, plus optional data files for complex initial conditions. 
   [Basically this is the same as \enzo] }

%-----------------------------------------------------------------------
\req
  {}
  {Proposed}
%
  {User input may include a graphical user interface (GUI) @@@
  glossary: graphical user interface: for problem specification}

%-----------------------------------------------------------------------
\req
  {}
  {Proposed}
%
  {The system shall support simulations in $1$D, $2$D, and $3$D}

%-----------------------------------------------------------------------
\req
  {}
  {Proposed}
%
  {The system shall output text information to the user regarding the
  current state of the simulation}

%-----------------------------------------------------------------------
\req
  {}
  {Proposed}
%
  {The system should output visual information to the user regarding
  the current state of the simulation}

%-----------------------------------------------------------------------
\req
  {}
  {Proposed}
%
  {The system shall enable external modification of physics and
  data-structure parameters of a running simulation}

%-----------------------------------------------------------------------
\req
  {}
  {Proposed}
  {The system shall attempt to recover from numerical errors (NaN, Inf)}

%-----------------------------------------------------------------------
\req
  {} {Proposed}
%
  {The system shall support outputing data for restart dumps}

%-----------------------------------------------------------------------
\req
  {} {Proposed}
%
  {The system shall support restarting a computation that has not
  completed with at most $10\%$ loss of total simulation time}

%=======================================================================
\section{Problem definition}
%=======================================================================


\req
  {}
  {Proposed}
%
  {Problems shall be fully defined using an input text file, called
  the ``problem definition file''}

%-----------------------------------------------------------------------
\req
  {}
  {Proposed}
%
  {Problems may input HDF5 files as part of the problem definition}

%-----------------------------------------------------------------------
\req
  {}
  {Proposed}
%
  {No assumptions shall be made on field variable names in the source
  code}

%-----------------------------------------------------------------------
\req
  {}
  {Proposed}
%
  {Problem definition files shall be sufficiently powerful to define
  all standard test problems. [quantify: explicitly list]}

%-----------------------------------------------------------------------
\req
  {}
  {Proposed}
%
  {Problem definition files shall support defining simple regions in
  space, including half-spaces, spheres, ellipsoids}

%-----------------------------------------------------------------------
\req
  {}
  {Proposed}
%
  {Problem definition files shall support constructing composite
  regions using intersection, union, and set difference on simple
  regions}

%-----------------------------------------------------------------------
\req
  {}
  {Proposed}
%
  {Composite regions shall not have any imposed restriction on number
  of constituent simple regions}

%-----------------------------------------------------------------------
\req
  {}
  {Proposed}
%
  {Problem definition files shall support defining field properties
  within a regions}

%-----------------------------------------------------------------------
\req
  {}
  {Proposed}
%
  {Field properties may vary between regions}

%-----------------------------------------------------------------------
\req
  {}
  {Proposed}
%
  {Numerical values in problem definition files may include arithmetic
  operations}

%-----------------------------------------------------------------------
\req
  {}
  {Proposed}
%
  {Numerical values in problem definition files may include the
  functions \code{log}, \code{exp}, \code{sin}, \code{cos},
  \code{min}, \code{max}.}

%-----------------------------------------------------------------------
\req
  {}
  {Proposed}
%
  {Numerical values in problem definition files may include \code{x},
  \code{y}, \code{z}, and \code{t} to represent spacial and temporal
  coodinates.}

%=======================================================================
\section{Methods}
%=======================================================================

\subsection{Hydrodynamics}

%-----------------------------------------------------------------------
\req
  {}
  {Proposed}
%
  {The system shall perform hydrodynamics computations with at least
  2nd order in both space and time ``almost everywhere''}
\subsection{Self-gravity}

%-----------------------------------------------------------------------
\req
  {}
  {Proposed}
%
  {The system shall perform self-gravity computations with at least
  2nd order accuracy in both space and time ``almost everywhere''}
\subsection{Cosmological expansion}

%-----------------------------------------------------------------------
\req
  {}
  {Proposed}
%
  {The system shall support cosmological expansion as an option}

\subsection{Radiative cooling}

%-----------------------------------------------------------------------
\req
  {}
  {Proposed}
%
  {The system shall perform radiative cooling computations}
\subsection{Multi-species chemistry}

%-----------------------------------------------------------------------
\req
  {}
  {Proposed}
%
  {The system shall perform multispecies chemistry computations}
\subsection{Star formation}

%-----------------------------------------------------------------------
\req
  {}
  {Proposed}
%
  {The system shall perform star-formation computations}
\subsection{Magnetohydrodynamics}

%-----------------------------------------------------------------------
\req
  {}
  {Proposed}
%
  {The system shall perform magnetohydrodynamic computations}
\subsection{Radiation transfer}

%-----------------------------------------------------------------------
\req
  {}
  {Proposed}
%
  {The system shall perform radiative transfer computations}

%=======================================================================
\section{Fields}
%=======================================================================


%-----------------------------------------------------------------------
\req
  {}
  {Proposed}
%
  {The system shall be able to represent ideal gas density, velocity,
  and @@@energy}

%-----------------------------------------------------------------------
\req
  {}
  {Proposed}
%
  {The system shall be able to represent dark matter density and
  velocity}

%-----------------------------------------------------------------------
\req
  {}
  {Proposed}
%
  {The system should support multiple ideal gases}

%
%-----------------------------------------------------------------------
\subsection{Accuracy}
%-----------------------------------------------------------------------

%-----------------------------------------------------------------------
\req
  {}
  {Proposed}
%
  {Simulation dynamic resolution range @@@ glossary: dynamic
  resolution range: shall be at least $2^64$, e.g.~$1024^3$
  @@@glossary: root grid: with $27$ additional levels of refinement by
  $4$)}

%-----------------------------------------------------------------------
\req
  {}
  {Proposed}
%
  {Simulation resolution range should be at least $2^128$}

%-----------------------------------------------------------------------
\req
  {}
  {Proposed}
%
  {Simulation size shall be virtually unbounded (run on largest
  parallel computer available)}

%=======================================================================
\section{Performance}
%=======================================================================

%-----------------------------------------------------------------------
\req
  {}
  {Proposed}
%
  {The system shall support multiple levels of parallelism}

%-----------------------------------------------------------------------
\req
  {}
  {Proposed}
%
  {The system shall support a data-parallel model using point-to-point
  @@@glossary MPI-1 communication}

%-----------------------------------------------------------------------
\req
  {}
  {Proposed}
%
  {The system may support a data-parallel model using one-sided
  @@@glossary MPI-2 communication}

%-----------------------------------------------------------------------
\req
  {}
  {Proposed}
%
  {The system shall support multiple levels of parallelism}

%-----------------------------------------------------------------------
\req
  {}
  {Proposed}
%
  {The system shall support OpenMP shared memory parallelism with
  parallel efficiency at least $80\%$ of maximum based on flops and
  memory relative to single-core up to $100$ cores}

%-----------------------------------------------------------------------
\req
  {}
  {Proposed}
%
  {The system shall support uniform resolution problems with at least
  $10\%$ average core flop rate on largest platform sizes available.}

%-----------------------------------------------------------------------
\req
  {}
  {Proposed}
%
  {The system shall support moderately non-uniform resolution problems
  (dynamic resolution range of $2^32$) with at least $3\%$ average
  core flop rate on largest platform sizes available.}

%-----------------------------------------------------------------------
\req
  {}
  {Proposed}
%
  {The system shall support highly non-uniform resolution problems
  (dynamic resolution range of $2^48$) with $1\%$ average core flop
  rate and $50\%$ parallel efficiency on largest platform sizes
  available.}

%-----------------------------------------------------------------------
\req
  {}
  {Proposed}
%
  {The system shall support highly non-uniform resolution problems on
  the largest platform sizes available that have at least $1GB$ memory
  per computational element.}

%=======================================================================
\section{Input}
%=======================================================================


\begin{itemize}
\item Problem
\item Physics
\item Algorithms
\item Data
\item Parallel
\item I/O
\item Performance
\end{itemize}

Equivalent \enzo\ parameters

\begin{tabular}{ll}
ControlCourantSafetyFactor        & CourantSafetyNumber \\
ControlStopCycle           & StopCycle \\
ControlStopTime            & StopTime \\
HydroMethod                & HydroMethod \\
HydroParameter "diffusion"    & PPMDiffusionParameter \\
HydroParameter "dual-energy"  & DualEnergyFormalism \\
HydroParameter "eta1"    & DualEnergyFormalismEta1 \\
HydroParameter "eta2"    & DualEnergyFormalismEta2 \\
HydroParameter "flattening"   & PPMFlatteningParameter \\
HydroParameter "pressure free" & PressureFree \\
HydroParameter "steepening"  & PPMSteepeningParameter \\
Material1Gamma             & Gamma \\
OutputUserDt               & dtDataDump \\
OutputUserDumpName         & DataDumpName \\
ProblemBcLower             & LeftFaceBoundaryCondition \\
ProblemBcUpper             & RightFaceBoundaryCondition \\
ProblemDomainLower         & DomainLeftEdge \\
ProblemDomainUpper         & DomainRightEdge \\
ProblemStartCycle          & InitialCycleNumber \\
ProblemStartTime           & InitialTime \\
                           & ProblemType \\
\end{tabular}

There are some variables defined that can be accessed in the input file.
\begin{tabbing}
xxx\=xxxxxxx\=xxxxxxxxxxxxxxxxxxxxxxxxxxxxxxxx\=\kill
\> \code{x},\code{y},\code{z} \> Coordinates in problem distance units. \\
\> \code{t} \> Time in problem time units. \\
\> \code{X},\code{Y},\code{Z} \> Coordinates in computational units. \\
\> \code{T} \> Time in computational units. \\
\> \code{PI} \> $\pi$ \\
\end{tabbing}

%=======================================================================
\section{Output}
%=======================================================================

%-----------------------------------------------------------------------
\req
  {}
  {Proposed}
%
  {}

\appendix


\end{document}

%==================================================================
