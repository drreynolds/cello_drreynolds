%0       1         2         3         4         5         6         7         8
%2345678901234567890123456789012345678901234567890123456789012345678901234567890
%=======================================================================
\documentclass[10pt]{article}
%=======================================================================

\newcommand{\req}[3]{
\begin{tabular}{rl}
ID & \textbf{R#1} \\
Status & \textsf{#2} \\
Acceptance & \textit{#3}
\end{tabular}
}


% INCLUDE DEVELOPMENT TEXT

\newcommand{\devel}[1]{\textbf{#1}}

% EXCLUDE DEVELOPMENT TEXT

% \newcommand{\devel}[1]{}


%=======================================================================
% Document layout
%=======================================================================

\setlength{\topmargin}{0.0in}
\setlength{\oddsidemargin}{0.0in}
\setlength{\evensidemargin}{0.0in}
\setlength{\textwidth}{6.0in}
\setlength{\textheight}{9.0in}

%=======================================================================
% Packages
%=======================================================================

\usepackage{wasysym}
\usepackage{epsfig}
\usepackage{url}

%=======================================================================
% Commands
%=======================================================================

\newcommand{\cello}{\textsf{Cello}}
\newcommand{\enzo}{\textsf{Enzo}}
\newcommand{\lcaperf}{\textsf{lcaperf}}
\newcommand{\lcatest}{\textsf{lcatest}}

\newcommand{\code}[1]{\textsf{#1}}

\newcommand{\note}[1]{\devel{\eighthnote\ \textit{#1} \\}}
\newcommand{\pargraph}[1]{\devel{\P\ \textbf{#1} \\}}

\newcommand{\todo}{\devel{$\circ$}}
\newcommand{\done}{\devel{$\bullet$}}
\newcommand{\halfdone}{\devel{\textcolor{gray}{$\bullet$}}}

\newcommand{\PROJECT}{\cello}

\newcommand{\TITLE}[3]{
\title{ {\huge \PROJECT\ #1}  \\ \vspace{0.1in}
     {\small Document Version: \textbf{#3}} \vspace{-0.1in}
    }
\author{      #2 \\
        Laboratory for Computational Astrophysics\\
        University of California, San Diego}
\maketitle}

%=======================================================================

%\usepackage{glossary}
%\makeglossary

%    1. Introduction
% 
%          1. Purpose
% 
% <Identify the product whose software requirements are specified in this document, including the revision or release number. Describe the scope of the product that is covered by this SRS, particularly if this SRS describes only part of the system or a single subsystem.>
% 
%          2. Document Conventions
% 
% <Describe any standards or typographical conventions that were followed when writing this SRS, such as fonts or highlighting that have special significance. For example, state whether priorities  for higher-level requirements are assumed to be inherited by detailed requirements, or whether every requirement statement is to have its own priority.>
% 
%          3. Intended Audience and Reading Suggestions
% 
% <Describe the different types of reader that the document is intended for, such as developers, project managers, marketing staff, users, testers, and documentation writers. Describe what the rest of this SRS contains and how it is organized. Suggest a sequence for reading the document, beginning with the overview sections and proceeding through the sections that are most pertinent to each reader type.>
% 
%          4. Project Scope
% 
% <Provide a short description of the software being specified and its purpose, including relevant benefits, objectives, and goals. Relate the software to corporate goals or business strategies. If a separate vision and scope document is available, refer to it rather than duplicating its contents here. An SRS that specifies the next release of an evolving product should contain its own scope statement as a subset of the long-term strategic product vision.>
% 
%          5. References
% 
% <List any other documents or Web addresses to which this SRS refers. These may include user interface style guides, contracts, standards, system requirements specifications, use case documents, or a vision and scope document. Provide enough information so that the reader could access a copy of each reference, including title, author, version number, date, and source or location.>
% 
%    2. Overall Description
% 
%          1. Product Perspective
% 
% <Describe the context and origin of the product being specified in this SRS. For example, state whether this product is a follow-on member of a product family, a replacement for certain existing systems, or a new, self-contained product. If the SRS defines a component of a larger system, relate the requirements of the larger system to the functionality of this software and identify interfaces between the two. A simple diagram that shows the major components of the overall system, subsystem interconnections, and external interfaces can be helpful.>
% 
%          2. Product Features
% 
% <Summarize the major features the product contains or the significant functions that it performs or lets the user perform. Details will be provided in Section 3, so only a high level summary  is needed here. Organize the functions to make them understandable to any reader of the SRS. A picture of the major groups of related requirements and how they relate, such as a top level data flow diagram or a class diagram, is often effective.>
% 
%          3. User Classes and Characteristics
% 
% <Identify the various user classes that you anticipate will use this product. User classes may be differentiated based on frequency of use, subset of product functions used, technical expertise, security or privilege levels, educational level, or experience. Describe the pertinent characteristics of each user class. Certain requirements may pertain only to certain user classes. Distinguish the favored user classes from those who are less important to satisfy.>
% 
%          4. Operating Environment
% 
% <Describe the environment in which the software will operate, including the hardware platform, operating system and versions, and any other software components or applications with which it must peacefully coexist.>
% 
%          5. Design and Implementation Constraints
% 
% <Describe any items or issues that will limit the options available to the developers. These might include: corporate or regulatory policies; hardware limitations (timing requirements, memory requirements); interfaces to other applications; specific technologies, tools, and databases to be used; parallel operations; language requirements; communications protocols; security considerations; design conventions or programming standards (for example, if the customer-F�s organization will be responsible for maintaining the delivered software).>-A
% 
%          6. User Documentation
% 
% <List the user documentation components (such as user manuals, on-line help, and tutorials) that will be delivered along with the software. Identify any known user documentation delivery formats or standards.>
% 
%          7. Assumptions and Dependencies
% 
% <List any assumed factors (as opposed to known facts) that could affect the requirements stated in the SRS. These could include third-party or commercial components that you plan to use, issues around the development or operating environment, or constraints. The project could be affected if these assumptions are incorrect, are not shared, or change. Also identify any dependencies the project has on external factors, such as software components that you intend to reuse from another project, unless they are already documented elsewhere (for example, in the vision and scope document or the project plan).>
% 
%    3. System Features
% 
% <This template illustrates organizing the functional requirements for the product by system features, the major services provided by the product. You may prefer to organize this section by use case, mode of operation, user class, object class, functional hierarchy, or combinations of these, whatever makes the most logical sense for your product.>
% 
%          1. System Feature 1
% 
% <Don-F�t really say -Y-A�System Feature 1.� State the feature name in just a few words.>
% 
%       3.1.1 Description and Priority
% 
%             <Provide a short description of the feature and indicate whether it is of High, Medium, or Low priority. You could also include specific priority component ratings, such as benefit, penalty, cost, and risk (each rated on a relative scale from a low of 1 to a high of 9).>
% 
%       3.1.2 Stimulus/Response Sequences
% 
%             <List the sequences of user actions and system responses that stimulate the behavior defined for this feature. These will correspond to the dialog elements associated with use cases.>
% 
%       3.1.3 Functional Requirements
% 
%             <Itemize the detailed functional requirements associated with this feature. These are the software capabilities that must be present in order for the user to carry out the services provided by the feature, or to execute the use case. Include how the product should respond to anticipated error conditions or invalid inputs. Requirements should be concise, complete, unambiguous, verifiable, and necessary. Use �TBD� as a placeholder to indicate when necessary information is not yet available.>
% 
%  
% 
%             <Each requirement should be uniquely identified with a sequence number or a meaningful tag of some kind.>
% 
%  
% 
%                   REQ-1: 
% 
%                   REQ-2: 
% 
%          2. System Feature 2 (and so on)
% 
%    4. External Interface Requirements
% 
%          1. User Interfaces
% 
% <Describe the logical characteristics of each interface between the software product and the users. This may include sample screen images, any GUI standards or product family style guides that are to be followed, screen layout constraints, standard buttons and functions (e.g., help) that will appear on every screen, keyboard shortcuts, error message display standards, and so on. Define the software components for which a user interface is needed. Details of the user interface design should be documented in a separate user interface specification.>
% 
%          2. Hardware Interfaces
% 
% <Describe the logical and physical characteristics of each interface between the software product and the hardware components of the system. This may include the supported device types, the nature of the data and control interactions between the software and the hardware, and communication protocols to be used.>
% 
%          3. Software Interfaces
% 
% <Describe the connections between this product and other specific software components (name and version), including databases, operating systems, tools, libraries, and integrated commercial components. Identify the data items or messages coming into the system and going out and describe the purpose of each. Describe the services needed and the nature of communications. Refer to documents that describe detailed application programming interface protocols. Identify data that will be shared across software components. If the data sharing mechanism must be implemented in a specific way (for example, use of a global data area in a multitasking operating system), specify this as an implementation constraint.>
% 
%          4. Communications Interfaces
% 
% <Describe the requirements associated with any communications functions required by this product, including e-mail, web browser, network server communications protocols, electronic forms, and so on. Define any pertinent message formatting. Identify any communication standards that will be used, such as FTP or HTTP. Specify any communication security or encryption issues, data transfer rates, and synchronization mechanisms.>
% 
%    5. Other Nonfunctional Requirements
% 
%          1. Performance Requirements
% 
% <If there are performance requirements for the product under various circumstances, state them here and explain their rationale, to help the developers understand the intent and make suitable design choices. Specify the timing relationships for real time systems. Make such requirements as specific as possible. You may need to state performance requirements for individual functional requirements or features.>
% 
%          2. Safety Requirements
% 
% <Specify those requirements that are concerned with possible loss, damage, or harm that could result from the use of the product. Define any safeguards or actions that must be taken, as well as actions that must be prevented. Refer to any external policies or regulations that state safety issues that affect the product-F-F�s design or use. Define any safety certifications that must be satisfied.>-A
% 
%          3. Security Requirements
% 
% <Specify any requirements regarding security or privacy issues surrounding use of the product or protection of the data used or created by the product. Define any user identity authentication requirements. Refer to any external policies or regulations containing security issues that affect the product. Define any security or privacy certifications that must be satisfied.>
% 
%          4. Software Quality Attributes
% 
% <Specify any additional quality characteristics for the product that will be important to either the customers or the developers. Some to consider are: adaptability, availability, correctness, flexibility, interoperability, maintainability, portability, reliability, reusability, robustness, testability, and usability. Write these to be specific, quantitative, and verifiable when possible. At the least, clarify the relative preferences for various attributes, such as ease of use over ease of learning.>
% 
%    6. Other Requirements
% 
% <Define any other requirements not covered elsewhere in the SRS. This might include database requirements, internationalization requirements, legal requirements, reuse objectives for the project, and so on. Add any new sections that are pertinent to the project.>
% 
% Appendix A: Glossary
% 
% <Define all the terms necessary to properly interpret the SRS, including acronyms and abbreviations. You may wish to build a separate glossary that spans multiple projects or the entire organization, and just include terms specific to a single project in each SRS.>
% 
% Appendix B: Analysis Models
% 
% <Optionally, include any pertinent analysis models, such as data flow diagrams, class diagrams, state-transition diagrams, or entity-relationship diagrams.>
% 
% Appendix C: Issues List
% 
% < This is a dynamic list of the open requirements issues that remain to be resolved, including TBDs, pending decisions, information that is needed, conflicts awaiting resolution, and the like.>

%=======================================================================

\begin{document}
\small
%=======================================================================
\TITLE{\cello\ Software Requirements Specification}{James Bordner}{$Rev$}
%=======================================================================

%\tableofcontents

%=======================================================================
%=======================================================================
\section{Scope}
%=======================================================================

\begin{enumerate}
%
\item All software in the system will be open source
%
\item The system shall be partitioned into two sub-systems: the
  distributed AMR data structure framework ``\cello'', and the
  collection of physics algorithms ``\enzoii''
%
\item The \cello\ framework shall not have any dependencies on the
  \enzoii\ algorithms
%
\item The system shall support coupled systems of hyperbolic,
  elliptic, and parabolic PDE's
%
\item The system shall support simulations in $1$D, $2$D, and $3$D \\
  \textit{1D and 2D useful for testing}
%
\item The system shall support Eulerian, Lagrangian, or hybrid methods \\
%
\item The system shall support multiple concurrent simulations
\end{enumerate}

%=======================================================================
\section{Adaptive mesh refinement}
%=======================================================================

\begin{enumerate}
%
\item AMR data structure size (number of patches) is ``close'' to
  optimal \\
  \textit{Note that this eliminates fixed-sized patches}
%
\item The system shall support ``fast'' searching of neighboring
  patches \\
  \textit{indicates either direct storage of neighbors, or an octree
    or chaining mesh type data structure}
%
\item The system shall not impose limits on hierarchy size in terms of
  number of grid patches patches at least $10^6$) \textit{indicates
    partial or fully distributed AMR data structure}
%
\item The system shall support hierarchy depth of up to $64$ levels, for
  a dynamic range of $2^{64}$
  \textit{Note this is greater than the age of the universe in seconds for}
%
\item The system shall support adaptive timestepping at the grid patch level
%
\item Spacial change in resolution will be bounded above by $2$.
%
\item AMR levels of symmetric problems will cover symmetric areas
\end{enumerate}

%=======================================================================
\section{Mesh data}
%=======================================================================

\begin{enumerate}
%
\item The system should support multiple general problem-dependent
  multi-resolution fields on locally logically Cartesian grids.
%
\item Data layout of fields will be flexible enough to interface
  directly with most grid-based user functions
%
\item Field size shall not be bounded by software constraints,
  e.g.~\code{MAX\_ZONES}, etc.
%
\item The number of Fields shall not be bounded by software constraints.
%
\item Field size shall not be bounded by limited range of global indices
%
\item Field variables may be located at zone centers, faces, edges, or vertices
%
\item Fields will support level-dependent scaling to maintain numerical accuracy at high resolutions
%
\item Precision of field variables may be 32-bit or 64-bit 
%
\item Different fields may have different precision
%
\item Fields will support stencils spanning up to three ghost/guard cells
%
\item Minumum or maximum values of fields may be specified
\end{enumerate}

%=======================================================================
\section{Particle data}
%=======================================================================

\begin{enumerate}
%
\item The system shall support multiple user-defined particle groups \\
  \textit{e.g. star particles, dark matter particles, tracer
    particles, etc. }
%
\item Each particle group can include arbitrary user-defined
  scalar attributes
%
\item User code has access to particles by group, and all particle attributes
%
\item The system shall support ``Fast'' searching of nearby particles
%
\item Numbers of particles shall not be bounded by software constraints, e.g.~\code{MAX\_PARTICLES}
%
\item Data layout of particles will be flexible enough to interface
  directly with most grid-based user functions
%
\item The local accuracy of nearby particle positions shall not be
  limited by the global extent of particle positions.
\end{enumerate}

%=======================================================================
\section{User-supplied functions}
%=======================================================================

\begin{enumerate}
%
\item The system shall support user-supplied patch-based methods, analysis, and visualization
%
\item The system shall support user-supplied level-based methods, analysis, and visualization
%
\item The system shall support user-supplied hierarchy-based methods, analysis, and visualization
%
\item User functions may be written in Fortran, C, or C++
%
\item User functions may impose inter-resolution constraints,
  e.g.~flux correction
%
\item User functions may impose assertions on field or particles
%
\item User functions may specify local spacial resolution requirements
  by tagging patch cells for coarsening or refinement
%
\item User functions may specify local temporal resolution requirements
  by returning a minimum timestep
%
\item User functions may define how individual fields may be
  interpolated between levels
%
\item Any user function can indicate a failure
%
\item Any user code can include alternate methods or solvers to
  improve  robustness
%
\item User functions can define problem initialization code
%
\item Different user functions can indicate different field ghost zone
  depth requirements for different fields
%
\item Different user functions can indicate different particle ghost
  zone depth requirements for different particle groups
%
\item User functions indicate which fields and particle groups they
  read or modify
%
\item User functions may define local persistent or transient fields
  or particle groups
%
\end{enumerate}


%=======================================================================
\section{Run-time parameters}
%=======================================================================

\begin{enumerate}
\item The expressive power of parameter files is sufficient to specify
  all \enzo\ test problems supported by the available \enzoii\ physics
  algorithms
%
\item The expressive power of parameter files is sufficient to specify
  all standard test problems in the literature supported by the
  available \enzoii\ physics algorithms
%
\item The expressive power of parameter files shall be sufficient for
  deep control of all available physics algorithms and distributed
  data structures
%
\item Problems shall be fully defined using an input text file, called
  the ``parameter file''
%
\item Parameter files may include other parameter files
%
\item Initial conditions may be initialized using data files
%
\item Initial conditions may be initialized using functions
%
\item Input via single or multiple included parameter files
%
\item Both framework and user code have access to run-time parameters
  \textit{useful for user code to take advantage of framework
    parameters } \textit{indicates an API for parameters }
%
\item Sufficiently powerful parameter file language to specify user
  problems \textit{difficult to test, but ability to e.g.~express
    arbitrary scalar and logical expressions, including standard
    functions (sin, log, etc.) for initial conditions using x,y,z,t
    variables } \textit{in particular, \enzoii\ will be able to define
    any current hydro / gravity / MHD test problem that Enzo can,
    despite not having specialized problem types }
%
\item Explicit initial conditions via data files
%
\item Allow ``deep'' parameter control of parallelism, parallel data
  structures, and user code \textit{indicates many parameters, which
    indicates parameters grouped by common functionality to reduce
    complexity }
\end{enumerate}
 

%=======================================================================
\section{Parallelism}
%=======================================================================

\begin{enumerate}
%
\item No fundamental limit on process / thread counts \textit{i.e. no
    hard-coded MAX\_PROCESSORS}
%
\item Support distributed memory, shared memory, and hybrid
  parallelism
%
\item Include thread-safety \textit{requires user code to be
    thread-safe }
%
\item Support range of parallel task sizes, including single array
  cells (note conflicts with AMR data structure size for standard
  octree AMR)
%
\item direct or indirect support of GPU's \textit{at least ability to
    schedule groups of many tasks at once } \textit{requires support
    of GPU programming in user code }
%
\item The system shall support multilevel parallelism of up to 4 levels
  \textit{not just distributed and shared memory.  E.g. distinguish
    between cores, sockets, nodes, cabinets }
%
\item The system's load balancing of parallel tasks shall be efficient, reliable, and effective
  \textit{indicates hierarchical, takes into account cost to relocate
    task, and level-dependent (and user-defined) metrics }
%
\item The system's parallel task scheduling shall be efficient, reliable, and effective
  \textit{given load balance, schedule tasks.  Indicates hierarchical,
    ability to group-schedule (for GPU).  Could use component
    parallelism to improve load balancing on extreme
    scales~\cite{@@@Shalf} }
%
\item The system shall directly support both distributed memory and
  shared memory parallelism.
%
\item The system shall support  distributed memory data-parallelism
  using MPI
%
\item The system shall support distributed memory task-parallelism
  using MPI
%
\item The system shall support shared-memory data-parallelism using
  OpenMP
%
\item The system shall support data-parallelism using UPC
%
\end{enumerate}

%
%=======================================================================
\section{Monitoring}
%=======================================================================

\begin{enumerate}
%
\item The system shall support text logging of the history and current
  state of the simulation
%
\item Clear output to the user reguarding the current state of the
  simulation.  \\
  \textit{ progress, performance, warnings, errors, attempts at
    recovery } \\
  \textit{ can include text, web page, simple plots, to
    show a dashboard of current and historical state }
%
\item Output includes user-defined output \textit{indicates user code
    can call logging API, including web pages and simple plots }
%
\item Ability to view and modify data structure parameters dynamically
  at run time
%
\item The system shall support simple visualization logging (plots,
  graphs, field data, and particle data)
%
\item All logging data will be available from a single file (e.g. web
  page)
\end{enumerate}
%
%=======================================================================
\section{Run-time Control}
%=======================================================================

\begin{enumerate}
\item The system shall enable external modification of physics and
  data-structure parameters of a running simulation
\end{enumerate}

%=======================================================================
\section{I/O }
%=======================================================================

\begin{enumerate}
%
\item The system shall support data dumps of all or subset of data fields and
  particle groups
%
\item The system shall support user method-dependent data output
%
\item The system shall use a standardized file format for AMR grid and
  particle data
%
\item All disk output shall include code for accessing the data
\end{enumerate}
       
%=======================================================================
\section{Performance}
%=======================================================================

\begin{enumerate}
%
\item The system shall be scalable to $> 10^6$ cores with greater than
  $50\%$ parallel efficiency in terms of computation, memory usage, and communication
%
\item The system's memory usage shall be scalable to $> 10^6$ cores with 
  at most $1GB$ per core
%
\item The system shall exhibit highly competetive serial performance
%
\item The system shall exhibit highly competetive parallel performance
%
\item The system shall exhibit highly competetive parallel scaling
%
\item The system shall exhibit highly competetive memory usage
%
\item The system shall exhibit highly competetive performance of
  uniform resolution problems
%
\item The system shall exhibit highly competetive performance of
  moderately non-uniform resolution problems (dynamic resolution range
  of $2^{10}$)
%
\item The system shall exhibit highly competetive performanece of
  higmly non-uniform resolution problems (dynamic resolution range of
  up to $2^{64}$)
%
\end{enumerate}

%=======================================================================
\section{Resilience}
%=======================================================================

\begin{enumerate}
%
\item The system shall attempt to detect and recover from numerical errors (NaN, Inf) \\
  \textit{e.g. by switching to an alternative method or solver, or
    using a smaller timestep}
%
\item The system shall attempt to detect and recover from loss of
  processing elements, including cores, processes, or nodes \\
  \textit{depends on ability to detect failures, which may require OS
    or library support not currently available } \\ \textit{indicates
    virtualization of processors and nodes: can backtrack and continue
    with reduced hardware } \\ \textit{indicates ability to flag
    components as unreliable }
%
\item The system shall attempt to detect and recover from memory
  faults
%
\item The system shall attempt to detect and recover from network
  interconnect faults
%
\item The system shall attempt to detect and recover from disk faults
%
\item The system shall support checkpoint / restart to disk
  \textit{indicates ability of user code to call \cello\ resiliance /
    fault tolerance API }
%
\item The system shall support checkpoint / restart to memory
%
\item The system shall be resilient to multiple failures
%
\item The system shall attempt to predict failures a priori and act to
  prevent probable failures
%
\end{enumerate}


%\appendix
%\printglossary
\end{document}

%==================================================================
