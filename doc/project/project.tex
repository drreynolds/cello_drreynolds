%0       1         2         3         4         5         6         7         8
%2345678901234567890123456789012345678901234567890123456789012345678901234567890
%=======================================================================
\documentclass{article}[12pt]
%=======================================================================


% INCLUDE DEVELOPMENT TEXT

\newcommand{\devel}[1]{\textbf{#1}}

% EXCLUDE DEVELOPMENT TEXT

% \newcommand{\devel}[1]{}


%=======================================================================
% Document layout
%=======================================================================

\setlength{\topmargin}{0.0in}
\setlength{\oddsidemargin}{0.0in}
\setlength{\evensidemargin}{0.0in}
\setlength{\textwidth}{6.0in}
\setlength{\textheight}{9.0in}

%=======================================================================
% Packages
%=======================================================================

\usepackage{wasysym}
\usepackage{epsfig}
\usepackage{url}

%=======================================================================
% Commands
%=======================================================================

\newcommand{\cello}{\textsf{Cello}}
\newcommand{\enzo}{\textsf{Enzo}}
\newcommand{\lcaperf}{\textsf{lcaperf}}
\newcommand{\lcatest}{\textsf{lcatest}}

\newcommand{\code}[1]{\textsf{#1}}

\newcommand{\note}[1]{\devel{\eighthnote\ \textit{#1} \\}}
\newcommand{\pargraph}[1]{\devel{\P\ \textbf{#1} \\}}

\newcommand{\todo}{\devel{$\circ$}}
\newcommand{\done}{\devel{$\bullet$}}
\newcommand{\halfdone}{\devel{\textcolor{gray}{$\bullet$}}}

\newcommand{\PROJECT}{\cello}

\newcommand{\TITLE}[3]{
\title{ {\huge \PROJECT\ #1}  \\ \vspace{0.1in}
     {\small Document Version: \textbf{#3}} \vspace{-0.1in}
    }
\author{      #2 \\
        Laboratory for Computational Astrophysics\\
        University of California, San Diego}
\maketitle}

%=======================================================================


%=======================================================================

\begin{document}

%=======================================================================
\TITLE{\cello\ Software Project Management Plan}{James Bordner}{$Rev$}
%=======================================================================

\tableofcontents

%=======================================================================
\section{Overview} \label{s:overview}
%=======================================================================

\subsection{Project summary}

The purpose of the \cello\ project is to provide an open source
software application for high-performance computational astrophysics
and cosmology.  It will be used both as a testbed for experimenting
with new software organization, parallelization, distributed
datastructure, algorithm, and implementation techniques, as well as
for enabling cutting-edge astrophysics and cosmological science
simulations on the largest parallel high performance computers
available.  Objectives are to reliably provide high-quality numerical
solutions, computed by distributing the workload efficiently across
$100,000$ to $1,000,000$ computational floating-point units, while
maintaining a high level of utilization of available computational
resources.

\devel{Physics.}
The scope of physics capabilities are representations of baryonic and
dark matter, hydrodynamics, self-gravity, radiative cooling,
cosmological expansion, star formation, radiative cooling,
multi-species chemistry, magnetohydrodynamics, and radiation transfer.
diation transfer.

\devel{AMR.}  Computations will be performed at multiple spacial
   and temporal resolutions using adaptive mesh refinement (AMR).
   This will permit the physics modules to capture the full range of
   scales of interest, but without excessive computation and memory
   storage.  Methods and data-structures will be designed to maximize
   scalability in both computation and memory usage.  Characteristics
   of the AMR hierarchies, e.g.~grid patch characteristics (size, size
   quantization, shape, etc.) will be flexible to allow for
   optimization of refinement efficiency, parallel task sizes, and
   task computation efficiency.

   \devel{Parallelization levels.}  To take advantage of the
   hierarchical parallel nature of modern supercomputers, \cello\ will
   support multiple levels of parallelization in its methods and
   datastructures.  This will include the coarse-grain parallelism to
   concurrently run simulations in an ensemble at the node or
   supernode level, medium-grain parallelism to evolve grid patches or
   particle groups of a simulation concurrently across a distributed
   memory subsystem at the node or socket level, and fine-grain parallelism to
   evolve grid cells within grid patches or particles within a group
   using shared-memory parallelism at the socket or core level.

   \devel{Parallelization types.}  This parallelization will be
   modularized (as much as technologically feasible) to improve
   flexibility in choosing the best method(s) of parallelization for a
   given problem on a given parallel platform.  MPI (two-sided), MPI2
   (one-sided), OpenMP, and possibly UPC support will be included, as
   well as flexibility in choosing hybrid schemes such as MPI +
   OpenMP, or MPI + UPC.  Other schemes based on shared memory array
   libraries or POSIX pthreads may also be considered.
   Parallelization methods will be primarily data parallelism,
   supporting both distributed memory and shared memory in isolation
   or combination.  Support for collaberative (functional) parallelism
   and pipelining will also be considered.

%  load balancing

   \devel{Dynamic load balancing.}  Multiple levels of parallel tasks
   will be load balanced using hierarchical dynamic load balancing
   algorithms.  Load balancing schemes will use dynamically measured
   performance data gathered by the running simulation to make load
   balancing decisions, and will allow flexibility in optimizing the
   parallel distribution of computation, memory storage, or a
   combination of both.  Combining flexible hierarchical
   parallelization schemes with hierarchical load balancing
   algorithms, together with scalable methods and efficient AMR
   data-structures, are expected to lead to high parallel efficiency
   and scalability.

% Field data layout: vertical data movement optimization

   \devel{Deep memory hierarchies.}  
%
   The properties of individual grid patches, and details of how
   scalar and vector fields are stored within grid patches, will be
   flexible to permit optimizing the use of deep memory/cache
   hierarchies.  This includes hierarchical blocking of grid data to
   maximize reuse of data in caches, padding of arrays to reduce cache
   thrashing effects for low-associativity caches, and interleaving
   vector field data or select scalar fields to improve data locality.
   These capabilities, together with efficient methods and
   implementation of computations, are expected to lead to high
   single-thread computational efficiency and data movement through
   memory/cache hierarchies.

%  Performance monitoring

   \devel{Performance monitoring.}  
%
   Global performance-related measurements will be continuously
   collected for simulations.  This will include parallel
   communication amount, rates, and time; memory and computational
   load balance efficiency; memory usage and reference rates; floating
   point operation counts and rates; disk storage rates and amounts,
   and time.


\subsection{Assumptions and constraints}

The source code for \cello\ must be made publicly available for
other researchers in computational astrophysics and cosmology.

\subsection{Project deliverables}

\subsubsection{Development Documentation}

\begin{enumerate}
\item Software Project Management Plan (SPMP)
\item Software Requirements Specification (SRS)
\item Software Design Description (SDD)
\item Software Test Document (STD)
\end{enumerate}

\subsubsection{Software Documentation}
\begin{enumerate}
\item Tutorial Presentation
\item Developer Website
\item User Website
\item Developer Manual
\item User Manual
\item Reference Manual
\end{enumerate}

\subsubsection{Source Code}
\begin{enumerate}
\item Main application source code
\item Reusable library of software components
\item Support utilities
\end{enumerate}

\subsubsection{Test Suite}
\begin{enumerate}
\item Unit tests
\item System tests
\end{enumerate}

\subsection{Schedule and budget summary}


\begin{description}
\item [Year 1: ]
\item [Year 2: ]
\item [Year 3: ]
\end{description}

Physics
Datastructures
Parallelism

\subsection{Evolution of the SPMP}

The latest versions of this SPMP and other internal software
documentation will be made available on
\code{http://nordlys.ucsd.edu/projects/cello}.


%=======================================================================
\section{References} \label{s:references}
%=======================================================================


%=======================================================================
\section{Definitions} \label{s:Definitions}
%=======================================================================
%=======================================================================
\section{Project Organization} \label{s:organization}
%=======================================================================
%=======================================================================
\section{Managerial Process Plans} \label{s:managerial}
%=======================================================================
%=======================================================================
\section{Technical Process Plans} \label{s:technical}
%=======================================================================
%=======================================================================
\section{Supporting Process Plans} \label{s:supporting}
%=======================================================================
%=======================================================================
\section{Additional Plans} \label{s:Additional Plans}
%=======================================================================



\subsection{Software documents}


%=======================================================================
\section{Project Organization} \label{s:intro}
%=======================================================================

%-----------------------------------------------------------------------
\section{Software Engineering Practices}
%-----------------------------------------------------------------------


Tasks

\begin{enumerate}
\item research
\item requirements
\item risk-analysis
\item design
\item implementation
\item testing
\item debugging
\item refactoring
\item maintenance
\item project management
\end{enumerate}

Physics capabilities

\begin{enumerate}
\item Hydrodynamics
\item radiative cooling
\item UV background
\item cosmological expansion
\item Gravity
\item MHD
\item RT
\end{enumerate}

High-level datastructure capabilities

\begin{enumerate}
\item Unigrid
\item SAMR
\item Particles
\item CAMR
\item SAMR/CAMR hybridization
\end{enumerate}

Low-level  datastructure capabilities

\begin{enumerate}
\item Simple array
\item Blocked array
\item Padded array
\item Interleaved array
\end{enumerate}

Parallelism capabilities

\begin{enumerate}
\item MPI2
\item OpenMP
\item UPC
\end{enumerate}

IO and visualization

\begin{enumerate}
\item Serial HDF5
\item Parallel HDF5
\item IDL visualization
\item VTK visualization
\end{enumerate}

Meta-solver capabilities

\begin{enumerate}
\item Static solver
\item Self-tuning AMR parameters
\item Self-tuning array layout parameters
\item Self-tuning parallelism
\item Self-tuning IO
\end{enumerate}


%=======================================================================
\section{Project Management Plan} \label{s:intro}
%=======================================================================

%-----------------------------------------------------------------------
\section{Tasks} \label{s:intro}
%-----------------------------------------------------------------------

%-----------------------------------------------------------------------
\subsection{Task-1: } \label{s:intro}
%-----------------------------------------------------------------------
\subsubsection{Task-1 description} \label{s:intro}
%-----------------------------------------------------------------------
\subsubsection{Task-1 deliverables and milestones} \label{s:intro}
%-----------------------------------------------------------------------
\subsubsection{Task-1 resources needed} \label{s:intro}
%-----------------------------------------------------------------------
\subsubsection{Task-1 dependencies and constraints} \label{s:intro}
%-----------------------------------------------------------------------
\subsubsection{Task-1 risks and contingencies} \label{s:intro}

%-----------------------------------------------------------------------
\section{Assignments} \label{s:intro}
%-----------------------------------------------------------------------
%-----------------------------------------------------------------------
\section{Timetable} \label{s:intro}
%-----------------------------------------------------------------------


\end{document}

%==================================================================
