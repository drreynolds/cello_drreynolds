%0       1         2         3         4         5         6         7         8
%2345678901234567890123456789012345678901234567890123456789012345678901234567890
%=======================================================================
\documentclass{article}[12pt]
%=======================================================================


% INCLUDE DEVELOPMENT TEXT

\newcommand{\devel}[1]{\textbf{#1}}

% EXCLUDE DEVELOPMENT TEXT

% \newcommand{\devel}[1]{}


%=======================================================================
% Document layout
%=======================================================================

\setlength{\topmargin}{0.0in}
\setlength{\oddsidemargin}{0.0in}
\setlength{\evensidemargin}{0.0in}
\setlength{\textwidth}{6.0in}
\setlength{\textheight}{9.0in}

%=======================================================================
% Packages
%=======================================================================

\usepackage{wasysym}
\usepackage{epsfig}
\usepackage{url}

%=======================================================================
% Commands
%=======================================================================

\newcommand{\cello}{\textsf{Cello}}
\newcommand{\enzo}{\textsf{Enzo}}
\newcommand{\lcaperf}{\textsf{lcaperf}}
\newcommand{\lcatest}{\textsf{lcatest}}

\newcommand{\code}[1]{\textsf{#1}}

\newcommand{\note}[1]{\devel{\eighthnote\ \textit{#1} \\}}
\newcommand{\pargraph}[1]{\devel{\P\ \textbf{#1} \\}}

\newcommand{\todo}{\devel{$\circ$}}
\newcommand{\done}{\devel{$\bullet$}}
\newcommand{\halfdone}{\devel{\textcolor{gray}{$\bullet$}}}

\newcommand{\PROJECT}{\cello}

\newcommand{\TITLE}[3]{
\title{ {\huge \PROJECT\ #1}  \\ \vspace{0.1in}
     {\small Document Version: \textbf{#3}} \vspace{-0.1in}
    }
\author{      #2 \\
        Laboratory for Computational Astrophysics\\
        University of California, San Diego}
\maketitle}

%=======================================================================


%=======================================================================

\begin{document}

%=======================================================================
\TITLE{Software Project Management Plan}{James Bordner}{v0.0.1}
%=======================================================================

%=======================================================================
\section{Overview} \label{s:overview}
%=======================================================================

This do

\subsection{Project summary}

The purpose of the \cello\ project is to provide an open source
software application for high-performance computational astrophysics
and cosmology.  It will be used both as a testbed for experimenting
with new software organization, parallelization, distributed
datastructure, algorithm, and implementation techniques, as well as
for enabling cutting-edge astrophysics and cosmological science
simulations on the largest parallel high performance computers
available.

The scope of physics capabilities are hydrodynamics, self-gravity,
radiative cooling, star formation, multi-species chemistry,
magnetohydrodynamics, and radiation hydrodynamics, using algorithms
based on the \enzo\ application.  These basic physics algorithmic
components will be implemented within a new framework for distributing
multi-resolution computations.

Objectives are to reliably provide high-quality numerical solutions,
computed by distributing the workload efficiently across $100,000$ to
$1,000,000$ computational floating-point units, while maintaining a
high level of utilization of available computational resources.

\subsection{Assumptions and constraints}

The source code for \cello\ must be made publicly available for
other researchers in computational astrophysics and cosmology.

\subsection{Project deliverables}

\subsubsection{Software Documentation}
\begin{enumerate}
\item Tutorial
\item User Guide
\item Reference Guide
\item Developer Guide
\item Project web site
\item Technical papers
\end{enumerate}

\subsubsection{Software Source Code}
\begin{enumerate}
\item Main application source code
\item Reusable library of software components
\end{enumerate}

\subsubsection{Software Tests}
\begin{enumerate}
\item Test descriptions
\item Test source code and input files
\item Test results
\end{enumerate}

\subsection{Schedule and budget summary}


\begin{description}
\item [Year 1: ]
\item [Year 2: ]
\item [Year 3: ]
\item [Year 4: ]
\end{description}

Physics
Datastructures
Parallelism

\subsection{Evolution of the SPMP}

The latest versions of this SPMP and other internal software
documentation will be made available on
\code{http://nordlys.ucsd.edu/projects/cello}.


%=======================================================================
\section{References} \label{s:references}
%=======================================================================

Project Management Plan
Software Requirements Specifications
Software Design Document
Software Testing
User Guide
Reference Guide
Developer Guide

%=======================================================================
\section{Definitions} \label{s:Definitions}
%=======================================================================
%=======================================================================
\section{Project Organization} \label{s:organization}
%=======================================================================
%=======================================================================
\section{Managerial Process Plans} \label{s:managerial}
%=======================================================================
%=======================================================================
\section{Technical Process Plans} \label{s:technical}
%=======================================================================
%=======================================================================
\section{Supporting Process Plans} \label{s:supporting}
%=======================================================================
%=======================================================================
\section{Additional Plans} \label{s:Additional Plans}
%=======================================================================



\subsection{Software documents}


%=======================================================================
\section{Project Organization} \label{s:intro}
%=======================================================================

%-----------------------------------------------------------------------
\section{Software Engineering Practices}
%-----------------------------------------------------------------------


Tasks

\begin{enumerate}
\item research
\item requirements
\item risk-analysis
\item design
\item implementation
\item testing
\item debugging
\item refactoring
\item maintenance
\item project management
\end{enumerate}

Physics capabilities

\begin{enumerate}
\item Hydrodynamics
\item radiative cooling
\item UV background
\item cosmological expansion
\item Gravity
\item MHD
\item RT
\end{enumerate}

High-level datastructure capabilities

\begin{enumerate}
\item Unigrid
\item SAMR
\item Particles
\item CAMR
\item SAMR/CAMR hybridization
\end{enumerate}

Low-level  datastructure capabilities

\begin{enumerate}
\item Simple array
\item Blocked array
\item Padded array
\item Interleaved array
\end{enumerate}

Parallelism capabilities

\begin{enumerate}
\item MPI2
\item OpenMP
\item UPC
\end{enumerate}

IO and visualization

\begin{enumerate}
\item Serial HDF5
\item Parallel HDF5
\item IDL visualization
\item VTK visualization
\end{enumerate}

Meta-solver capabilities

\begin{enumerate}
\item Static solver
\item Self-tuning AMR parameters
\item Self-tuning array layout parameters
\item Self-tuning parallelism
\item Self-tuning IO
\end{enumerate}


%=======================================================================
\section{Project Management Plan} \label{s:intro}
%=======================================================================

%-----------------------------------------------------------------------
\section{Tasks} \label{s:intro}
%-----------------------------------------------------------------------

%-----------------------------------------------------------------------
\subsection{Task-1: } \label{s:intro}
%-----------------------------------------------------------------------
\subsubsection{Task-1 description} \label{s:intro}
%-----------------------------------------------------------------------
\subsubsection{Task-1 deliverables and milestones} \label{s:intro}
%-----------------------------------------------------------------------
\subsubsection{Task-1 resources needed} \label{s:intro}
%-----------------------------------------------------------------------
\subsubsection{Task-1 dependencies and constraints} \label{s:intro}
%-----------------------------------------------------------------------
\subsubsection{Task-1 risks and contingencies} \label{s:intro}

%-----------------------------------------------------------------------
\section{Assignments} \label{s:intro}
%-----------------------------------------------------------------------
%-----------------------------------------------------------------------
\section{Timetable} \label{s:intro}
%-----------------------------------------------------------------------


\end{document}

%==================================================================
