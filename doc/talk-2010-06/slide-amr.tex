\begin{frame}[fragile] 
%--------------------------------------------------
\frametitle{AMR Variants}
%--------------------------------------------------
\begin{itemize}
\item "Structured" AMR
\begin{itemize}
\item   e.g.~\enzo, Chombo, SAMRAI
\end{itemize}
\item "Octree-based" AMR
\begin{itemize}
\item   e.g.~Flash, ...
\end{itemize}
\item Each approach has scalability issues
\end{itemize}
\end{frame}


\begin{frame}[fragile] 
%--------------------------------------------------
\frametitle{Structured AMR scalability issues}
%--------------------------------------------------
\begin{itemize}
\item  gridding placement algorithm
\begin{itemize}
\item tradeoff between mesh quality and scaling
\item \enzo\ favors speed
\end{itemize}
\item  complicated hierarchy structure
\begin{itemize}
\item     variable number of children
\item     variable number of siblings
\item     complicated communication
\end{itemize}
\item  variable-sized task sizes
\begin{itemize}
\item     scheduling
\item     load balancing
\end{itemize}
\item  load balancing 
\end{itemize}
\end{frame}

\begin{frame}[fragile] 
%--------------------------------------------------
\frametitle{Octree AMR scalability issues}
%--------------------------------------------------
\begin{itemize}
\item fixed-size patches inefficient for "shallow" AMR
\begin{itemize}
\item  refining a single point involves entire root-level patch
\end{itemize}
\item octree refinement inefficient for "deep" AMR
\end{itemize}
%       - reduced operations for physics modules compared to SAMR
%         - cubical typically small blocks
%         - limited inter-block configurations [ figure ]
\end{frame}

%     - "AMR hierarchy" defines changes in resolution
%     - "AMR patch" defines areas of uniform resolution
%     - hierarchy size should be proportional to "amount" of resolution change
\begin{frame}[fragile] 
%--------------------------------------------------
\frametitle{Cello AMR solution}
%--------------------------------------------------
%   AMR:        totally new AMR approach: generalization of octrees--rationale
%   AMR:        additional level of refinement: patches + blocks
%   AMR:        new [currently proprietary] enhancement for "large, shallow" AMR
%   AMR:        new [currently proprietary] enhancement for "deep, targeted" AMR
%   AMR:        no level jumps, symmetric refinement for symmetric problems
%   AMR:        ghost zones optionally allocated only when needed
\begin{itemize}
\item  Based on octree since most scalable
\begin{itemize}
\item    e.g.~ALPS framework has scaled to >32K cores
\begin{itemize}
\item      (uniform time-stepping helps)
\end{itemize}
\end{itemize}
\item Mesh data associated with leaf nodes only
\begin{itemize}
\item   sibling data transfer only
\begin{itemize}
\item     no parent-child data transfer
\end{itemize}
\end{itemize}
\item Quantized number of siblings
\begin{itemize}
\item   ``1/4'', 1, or 4 per face
\end{itemize}
\item Enhancements to address other issues
\begin{itemize}
\item   Coalesce patches in same refinement level to reduce AMR overhead
\begin{itemize}
\item     ...but still maintain fixed task sizes
\end{itemize}
\item   Targeted refinement for deep AMR problems
\begin{itemize}
\item      ...but still maintain $r=2$ refinement everywhere
\end{itemize}
\end{itemize}
\end{itemize}
\end{frame}

\begin{frame}[fragile] 
%--------------------------------------------------
\frametitle{Cello AMR enchancement: patch coalescing}
%--------------------------------------------------
\begin{minipage}{2.2in}
\includegraphics<1>[width=2.2in]{dots.png}
\includegraphics<2>[width=2.2in]{dots-4-1.png}
\includegraphics<3>[width=2.2in]{dots-16-5.png}
\end{minipage}
\begin{minipage}{1.6in}
\footnotesize
      \begin{itemize}
        \item \ENHANCE{1}begin with a collection of point sources
        \item \ENHANCE{2}octree refinement with $r=2$ has 2137 nodes (L=10)
        \item \ENHANCE{3}targeted refinement with $r=4$ has 158 nodes
      \end{itemize}
\end{minipage}

\end{frame}


\begin{frame}[fragile] 
%--------------------------------------------------
\frametitle{Cello AMR enchancement: targeted refinement}
%--------------------------------------------------
%       - enhancements for "deep" problems (many refinement levels)
%         - octree refinment not "targeted": single point involves entire domain
%         - we will allow refinement by 4 and possibly 8
%            - still maintain smooth level jumps by 2 everywhere
%         - will address integer index range issues
%         - will address floating point positional precision issues
\end{frame}

\begin{frame}[fragile] 
%--------------------------------------------------
\frametitle{Cello AMR enhancements: precision and ranges}
%--------------------------------------------------
\begin{itemize}
\item Precision and range issues arise for ``wide'' or ``deep'' problems
\item    Global position precision
\begin{itemize}
\item store positions using patch-local coordinates
\item eliminates ``catostraphic cancellation''
\item single-precision probably sufficient
\item coordinate changes between patches is trivial
\item global positions computed quickly if needed
\end{itemize}
\item Global index range
\begin{itemize}
\item indices known implicitly from octree-like data structure
\item global indices computed quickly if needed
\end{itemize}
\end{itemize}
\end{frame}

\begin{frame}[fragile] 
%--------------------------------------------------
\frametitle{Cello AMR enhancements: distributed patches}
%--------------------------------------------------
%     - Cello adds a new "level" to AMR
%       - "AMR block" defines task size
%       - typically small, e.g. $4^3$ to $8^3$
%       - a patch is composed of one or more blocks
%       - unigrid problem is single patch
%       - decouples "uniform resolution" requirements from "parallel task size"
\begin{itemize}
\item Subdivide AMR patches into regular ``blocks''
\begin{itemize}
\item   a block represents a parallel task
\begin{itemize}
\item     large patches are distributed
\item     unigrid problem is single large distributed AMR patch
\end{itemize}
\item   sizes can be fixed / quantized ($4^3$, $8^3$, etc.)
\begin{itemize}
\item     simplifies load balancing
\item     reduces memory fragmentation
\item     size optimizable for performance
\end{itemize}
\end{itemize}
\item Fully or partially distributed hierarchy metadata
\begin{itemize}
\item   Can go far with non-distributed tree
\begin{itemize}
\item     octree node overhead: 24 bytes per node
\item     \enzo\ grid overhead: 1500 bytes per grid
\end{itemize}
\item   All AMR blocks connected to neighbors and children
\begin{itemize}
\item     ``ghost blocks'' analagous to ghost zones
\end{itemize}
\item Load balancing 
\end{itemize}
\end{itemize}
\end{frame}

\begin{frame}[fragile] 
%--------------------------------------------------
\frametitle{Cello AMR enhancements: adaptive timestepping}
%--------------------------------------------------
\begin{itemize}
\item Uniform timestepping: $dt(cycle)$
\begin{itemize}
\item   highest parallel efficiency
\item   involves extra timesteps on coarse levels
\item   not feasible for deep problems
\item   computing timestep requires global synchronization
\end{itemize}
\item Level-adaptive timestepping: $dt(cycle,level)$
\begin{itemize}
\item   reduced parallel efficiency 
\item   interpolation required at level boundaries
\item   computing timestep requires global synchronization
\end{itemize}
\item Patch-adaptive timestepping: $dt(cycle,level)$
\begin{itemize}
\item   computing timestep does not require global synchronization
\item   slow localized regions will not impact global performance as much
\item   interpolation may be required between patches in same level
\end{itemize}
\end{itemize}
\end{frame}

