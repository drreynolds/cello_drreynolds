%=======================================================================
\section{\code{Simulation} Component} \label{s:component-simulation}
%=======================================================================

Given \code{Physics}, \code{Methods}, and \code{Mesh} or
\code{Particles} data structures, specifies and implements the
top-level sequencing and properties of the \code{Problem} (or
\code{Problems} in the case of an ensemble) to run.

\begin{itemize}
\item \code{Simulation}
\item \code{Ensemble}
\item \code{Control}
\item \code{Problem}
\item \code{Physics}
\item \code{Method}
\item \code{IO}
\end{itemize}

\subsection{Attributes.}

\subsection{Operations}

%=======================================================================
\subsection{\code{Ensemble} Component} \label{s:component-ensemble}
%=======================================================================

The \code{Ensemble} class is used to define a ensemble of simulations.  

\subsection{Attributes.}

\subsection{Operations}


%=======================================================================
\subsection{Control Component} \label{ss:component-control}
%=======================================================================


Global simulation control.

Output types and parameters

\begin{itemize}
\item checkpoint (dump all)
\item output (specific fields)
\item movies (type and rate)
\item analysis (type of analysis, rate)
\item level of output (files for timestep, time, etc.)
\end{itemize}

%-----------------------------------------------------------------------
\subsection{Use Cases}
%-----------------------------------------------------------------------
%-----------------------------------------------------------------------
\subsection{Parameters}
%-----------------------------------------------------------------------

Problem parameters specify the setup of the physical problem,
including initial conditions of relevant data fields and boundary
conditions.

  dimensionality
  domain extents
  initial conditions (materials, subregions, input)
 boundary conditions (periodic, in-/out-flow, specified, dynamic)

Problem parameters include initial conditions and boundary conditions.

Different types of boundary conditions are supported, including
periodic, in- and out-flow, specified, and dynamic.  Different
boundary conditions can be specified for the entire domain, on
separate faces, on subregions of faces, or on specific zones.
Different boundary conditions can be specified for different fields.
%-----------------------------------------------------------------------
\subsection{Use Cases}
%-----------------------------------------------------------------------

\begin{verbatim}
   problem {
      boundary {
         x:lower = reflecting
         x:upper = { type = reflecting }
         y       = { type = periodic }
         z       = { type = inflow,  value = 1.0 }
         z       = { outflow, 1.0 }
      }
   }
\end{verbatim}

\begin{verbatim}
   XM = boundary { x = domain:lower[0] }
   XP = boundary { x = domain:upper[0] }
   YM = boundary { y = domain:lower[1] }
   YP = boundary { y = domain:upper[1] }
   ZM = boundary { z = domain:lower[2] }
   ZP = boundary { z = domain:upper[2] }
   field {
      name = "density"
      value(XM) = 0
      value(XP) = 0
      value(YM) = value (YP)
      value(ZM) = +t
      value(ZM) = -t
   }
\end{verbatim}

%=======================================================================
\subsection{Domain} \label{ss:component-domain}
%=======================================================================

The \code{domain} function is used to specify properties of the
domain.  Domains are boxes aligned with the axes of the computational
coordinate system, and are uniquely determined by the spacial
dimension, and the lowest and highest points in the domain.

%-----------------------------------------------------------------------
\subsubsection{Use Cases}
%-----------------------------------------------------------------------

\begin{verbatim}
   domain { 
      dimension = 3
      lower     = <-3e9,-3e9,-3e9>
      upper     = <3e9,3e9,3e9>
   }
\end{verbatim}

\begin{verbatim}
   domain { 3, <-3e9,-3e9,-3e9>, <3e9,3e9,3e9> }  // Implicit ordering
\end{verbatim}

\begin{verbatim}
   domain { 
      dimension = 3
      upper     = 3e9        // expand scalar to vector
      lower     = -upper     // parameters can be accessed as values
   }
\end{verbatim}

Include errors.

%-----------------------------------------------------------------------
\subsubsection{Parameters}
%-----------------------------------------------------------------------

 \todo\ \textit{Decide: allow defaults?  allow optional parameters?  Special
 \code{OPT\_} prefix for optional parameters?  Write out explicit copy
 of input file?}

The \code{domain} function has three parameters

\begin{tabular}{lll} \\
Name & Type & Restrictions \\ \hline
\code{dimension} & Scalar & $1-3$ \\
\code{lower}     & Vector & length = \code{dimension} \\
\code{upper}     & Vector & length = \code{dimension}, \code{upper} $>$ \code{lower}
\end{tabular}

Lower and upper points are given in units given by \code{units},
described in \S\ref{ss:params-units}.

%-----------------------------------------------------------------------
\subsubsection{Restrictions}
%-----------------------------------------------------------------------

\begin{enumerate}
\item The dimension must be 1, 2, or 3.
\item The number of coordinates in both lower and upper points must equal the dimension.
\item Each coordinate of the lower point must be strictly greater than the corresponding coordinate of the upper point.
\end{enumerate}


%-----------------------------------------------------------------------
\subsubsection{Parameters}
%-----------------------------------------------------------------------

%-----------------------------------------------------------------------
\subsection{\code{Boundary}} \label{ss:component-boundary}
%-----------------------------------------------------------------------

The \code{Boundary} class is used to define the boundary conditions on
the domain.

%-----------------------------------------------------------------------
\subsection{\code{Initial}} \label{s:component-initial}
%-----------------------------------------------------------------------

The \code{Initial} class is used to define the initial conditions for
a problem.



