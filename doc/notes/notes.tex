%0       1         2         3         4         5         6         7         8
%2345678901234567890123456789012345678901234567890123456789012345678901234567890
%=======================================================================
\documentclass{article}
%=======================================================================


% INCLUDE DEVELOPMENT TEXT

\newcommand{\devel}[1]{\textbf{#1}}

% EXCLUDE DEVELOPMENT TEXT

% \newcommand{\devel}[1]{}


%=======================================================================
% Document layout
%=======================================================================

\setlength{\topmargin}{0.0in}
\setlength{\oddsidemargin}{0.0in}
\setlength{\evensidemargin}{0.0in}
\setlength{\textwidth}{6.0in}
\setlength{\textheight}{9.0in}

%=======================================================================
% Packages
%=======================================================================

\usepackage{wasysym}
\usepackage{epsfig}
\usepackage{url}

%=======================================================================
% Commands
%=======================================================================

\newcommand{\cello}{\textsf{Cello}}
\newcommand{\enzo}{\textsf{Enzo}}
\newcommand{\lcaperf}{\textsf{lcaperf}}
\newcommand{\lcatest}{\textsf{lcatest}}

\newcommand{\code}[1]{\textsf{#1}}

\newcommand{\note}[1]{\devel{\eighthnote\ \textit{#1} \\}}
\newcommand{\pargraph}[1]{\devel{\P\ \textbf{#1} \\}}

\newcommand{\todo}{\devel{$\circ$}}
\newcommand{\done}{\devel{$\bullet$}}
\newcommand{\halfdone}{\devel{\textcolor{gray}{$\bullet$}}}

\newcommand{\PROJECT}{\cello}

\newcommand{\TITLE}[3]{
\title{ {\huge \PROJECT\ #1}  \\ \vspace{0.1in}
     {\small Document Version: \textbf{#3}} \vspace{-0.1in}
    }
\author{      #2 \\
        Laboratory for Computational Astrophysics\\
        University of California, San Diego}
\maketitle}

%=======================================================================


%=======================================================================

\begin{document}

%=======================================================================
\TITLE{Notes}{James Bordner}{}
%=======================================================================

%=======================================================================
\section{\enzo\ performance and scaling issues}
%=======================================================================

Compared to those available today, high performance computers in the
upcoming generation are expected to be significantly broader and
deeper.  ``Broader'' in the sense of increased computational
parallelism, with the number of simultaneous floating-point operations
approaching a million @@@; and ``deeper'' in the sense that
parallelism will be organized into more layers (in particular with the
addition of a ``core'' layer) and deeper memory hierarchies.

For a scientific application to be truly efficient on high performance
computers, it must take the fullest advantage of both the machine's
breadth and depth.  Memory access patterns must make efficient use of
the memory hierarchy by maximizing data reuse and hiding latency;
communication within each parallel hierarchy level must be efficiently
managed; parallel computations must be evenly load-balanced within
each parallel hierarchy level; parallel efficiency must be maintained
at all levels, from nodes to floating-point units; global
synchronization points must be kept to a bare minimum or eliminated
entirely; all software algorithms' computation, memory usage, and
communication must all be scalable with both problem size and machine
size; software must be designed to efficiently map the software
datastructures---even when they are dynamically adapting to changes in
the underlying scientific problem being solved---to the available
hardware resources; and the software may need to adapt to changing
hardware resources, due to increased fault rates from increased
hardware complexity and size.  All of this and more are required for
scientific applications, in addition to solving the target scientific
problem.

Fortuntely, both new and improved tools are available for high
performance scientific computing applications to take advantage of the
increasingly complex hardware resources.  Improved compiler
parallelizing and optimization technology, new and improved
application software algorithms and datastructures, better performance
and optimization software tools and techniques, new libraries and
languages for supporting parallelization, parallel debugging tools,
and software engineering practices that help control the ever-growing
software complexity, and help increase reliability, flexibility, and
adaptability.


\newcommand{\REF}[1]{\ref{#1}}
\newcommand{\TAG}[1]{\texttt{#1}}

\begin{tabular}{c|l|l}
\REF{issue:amr} & \TAG{issue:amr} & AMR distributed datastructure \\
\REF{issue:amr-neighbors} & \TAG{issue:amr-neighbors} & Sibling search \\
\REF{issue:amr-balance} & \TAG{issue:amr-balance} & Dynamic load balancing \\
\REF{issue:amr-rebuild} & \TAG{issue:amr-rebuild} & Rebuild hierarchy \\
\REF{issue:amr-ghost-update} & \TAG{issue:amr-ghost-update} & Boundary update \\
\REF{issue:particle-movement} & \TAG{issue:particle-movement} & Particle movement \\
\REF{issue:particle-location} & \TAG{issue:particle-location} & Particle location \\
\REF{issue:libraries} & \TAG{issue:libraries} & External libraries \\
\REF{issue:memory-fragment} & \TAG{issue:memory-fragment} & Memory fragmentation \\
\REF{issue:memory-hierarchy-use} & \TAG{issue:memory-hierarchy-use} & Hierarchical memory utilization \\
\REF{issue:method-gravity} & \TAG{issue:method-gravity} & Gravity solver \\
\REF{issue:method-timestep-global} & \TAG{issue:method-timestep-global} & Global timesteps \\
\REF{issue:method-timestep-sliver} & \TAG{issue:method-timestep-sliver} & Sliver timesteps \\
\REF{issue:fault-detect} & \TAG{issue:fault-detect} & Fault detection \\
\REF{issue:fault-recover} & \TAG{issue:fault-recover} & Fault recovery \\
\REF{issue:data-io} & \TAG{issue:data-io} & I/O performance and reliability \\
\REF{issue:data-analyse} & \TAG{issue:data-analyse} & Data analysis \\
\REF{issue:data-archive} & \TAG{issue:data-archive} & Data archiving \\
\REF{issue:code-vectorize} & \TAG{issue:code-vectorize} & Vector utilization \\
\REF{issue:performance-measure} & \TAG{issue:performance-measure} & Evaluating modifications \\
\REF{issue:performance-tests} & \TAG{issue:performance-tests} & Test problems 
\end{tabular}

%-----------------------------------------------------------------------
\subsection{AMR distributed datastructure} \label{issue:amr}
%-----------------------------------------------------------------------
%-----------------------------------------------------------------------
\subsection{Sibling search}\label{issue:amr-neighbors}
%-----------------------------------------------------------------------
%-----------------------------------------------------------------------
\subsection{Dynamic load balancing} \label{issue:amr-balance}
%-----------------------------------------------------------------------
%-----------------------------------------------------------------------
\subsection{Rebuild hierarchy} \label{issue:amr-rebuild}
%-----------------------------------------------------------------------
%-----------------------------------------------------------------------
\subsection{Boundary update} \label{issue:amr-ghost-update}
%-----------------------------------------------------------------------
%-----------------------------------------------------------------------
\subsection{Particle movement} \label{issue:particle-movement}
%-----------------------------------------------------------------------
%-----------------------------------------------------------------------
\subsection{Particle location} \label{issue:particle-location}
%-----------------------------------------------------------------------
%-----------------------------------------------------------------------
\subsection{External libraries} \label{issue:libraries}
%-----------------------------------------------------------------------
%-----------------------------------------------------------------------
\subsection{Memory fragmentation} \label{issue:memory-fragment}
%-----------------------------------------------------------------------
   problem at C++ - Fortran interface 
%-----------------------------------------------------------------------
\subsection{Hierarchical memory utilization} \label{issue:memory-hierarchy-use}
%-----------------------------------------------------------------------
%-----------------------------------------------------------------------
\subsection{Gravity solver} \label{issue:method-gravity}
%-----------------------------------------------------------------------

  Enzo often needs to efficiently solve Poisson's equation over the
  grid hierarchy, most notably to compute the gravitational
  acceleration at each point.  It does this using a combination FFT
  and multi-grid approach, where an FFT is used on the root grid (to
  easily allow for periodic boundary conditions), and multi-grid is
  used in each patch, with boundary conditions interpolated from
  parent to child.  This approach is quite fast (anecdotal evidence
  places it at least several times faster than the FLASH AMR code),
  but has two problems.  The first is that it produces a relatively
  noisy solution, which is generally fine for cosmological
  simulations, but more accurate results are required for the other
  applications that Enzo is now being used for.  The second problem is
  that the acceleration predicted in the ghost zones of a patch can be
  inconsistent with the solution derived in the sibling patch.  The
  solution across sibling grids can be matched with an iterative
  approach, but at the cost of more communication between patches.  We
  intend to switch gravity solvers and use the approach described in
  \cite{hg00}.  This is quite similar to our current approach but adds
  a correction cycle that uses the fine grid solution to improve the
  coarse grid result, which is then re-interpolated to the child grid.
  This technique both dramatically decreases the size of the remaining
  error, and restores consistency between patches.  This will allow us
  to drop the current sibling iteration step and so reduce the amount
  of communication required.


%-----------------------------------------------------------------------
\subsection{Global timesteps} \label{issue:method-timestep-global}
%-----------------------------------------------------------------------

  Timesteps are determined globally: all grids on all processors
  within a level of the AMR hierarchy are advanced at the same
  timestep.  The issue is that, since the timestep is global, a global
  synchronization step must be performed.  For runs that use tens or
  hundreds of thousands of processors, this global synchronization can
  potentially become a bottleneck.  In principle, the timestep does
  not need to be determined globally, but could be determined on a
  grid by grid basis.  Not only would this not require a global
  synchronization, but it would permit grids to advance at larger
  timesteps than they otherwise would be restricted to.  A drawback
  would be how to handle the interface between neighboring grids that
  are being advanced at different timesteps.  It is not yet known how
  to do this without sacrificing accuracy, but is worthwhile to
  persue.

%-----------------------------------------------------------------------
\subsection{Sliver timesteps} \label{issue:method-timestep-sliver}
%-----------------------------------------------------------------------

  ``Sliver'' timesteps arise from global timestepping on successive
  levels of the hierarchy not being integral factors of one another.
  If the timestep on one level is $k$, and the timestep on the
  next-finer level is $k/2-\epsilon$, then three timesteps must be
  performed on the finer level instead of two.  It is not known how to
  handle these; however, one possibility would be to restrict
  timesteps between levels to be integral factors of each other,
  analagous to mesh refinement factor being an integer.  Another
  possibility would be to determine the finest-level timestep, then
  ``backtrack'' to coarser grids and adjusting their timesteps to
  avoid sliver timesteps.  It is not known whether this would be
  possible.

%-----------------------------------------------------------------------
\subsection{Fault detection} \label{issue:fault-detect}
%-----------------------------------------------------------------------
%-----------------------------------------------------------------------
\subsection{Fault recovery} \label{issue:fault-recover}
%-----------------------------------------------------------------------
%-----------------------------------------------------------------------
\subsection{I/O performance and reliability} \label{issue:data-io}
%-----------------------------------------------------------------------
   I/O speed and capacity will lag
%-----------------------------------------------------------------------
\subsection{Data analysis} \label{issue:data-analyse}
%-----------------------------------------------------------------------
   Off-site data movement
%-----------------------------------------------------------------------
\subsection{Data archiving}\label{issue:data-archive}
%-----------------------------------------------------------------------
%-----------------------------------------------------------------------
\subsection{Vector utilization} \label{issue:code-vectorize}
%-----------------------------------------------------------------------
%-----------------------------------------------------------------------
\subsection{Evaluating modifications}\label{issue:performance-measure}
%-----------------------------------------------------------------------
%-----------------------------------------------------------------------
\subsection{Test problems} \label{issue:performance-tests}
%-----------------------------------------------------------------------


%=======================================================================
\section{\enzo\ performance and scaling solutions}
%=======================================================================

\begin{tabular}{c|l|l}
\REF{solution:amr-grid-refactor} & \TAG{solution:amr-grid-refactor} & Grid class refactoring \\
\REF{solution:amr-cache} & \TAG{solution:amr-cache} & Cached AMR datastructure  \\
\REF{solution:amr-large-grids} & \TAG{solution:amr-large-grids} & Favor larger grids  \\
\REF{solution:amr-dynamic-ghosts} & \TAG{solution:amr-dynamic-ghosts} & Dynamic ghost zones  \\
\REF{solution:amr-octree} & \TAG{solution:amr-octree} & Octree  \\
\REF{solution:amr-boxtree} & \TAG{solution:amr-boxtree} & Recursive Binary Box Tree (RBBT)  \\
\REF{solution:amr-grid-quantized} & \TAG{solution:amr-grid-quantized} & Quantized grid patch sizes  \\
\REF{solution:amr-balance-trees} & \TAG{solution:amr-balance-trees} & Load balance trees not patches  \\
\REF{solution:amr-balance-performance} & \TAG{solution:amr-balance-performance} & Load balance using performance measurements  \\
\REF{solution:amr-balance-hilbert} & \TAG{solution:amr-balance-hilbert} & Load balance using space-filling curves  \\
\REF{solution:amr-balance-hierarchical} & \TAG{solution:amr-balance-hierarchical} & Hierarchical load balancing  \\
\REF{solution:amr-traversal-local} & \TAG{solution:amr-traversal-local} & Traverse local grids only  \\
\REF{solution:amr-grid-reuse} & \TAG{solution:amr-grid-reuse} & Reuse existing grid classes  \\
\REF{solution:amr-balance-split} & \TAG{solution:amr-balance-split} & Inter-level refinement  \\
\REF{solution:particles-group} & \TAG{solution:particles-group} & Particle groups  \\
\REF{solution:parallel-hybrid} & \TAG{solution:parallel-hybrid} & Hybrid parallelism \\
\REF{solution:parallel-pgas} & \TAG{solution:parallel-pgas} & PGAS distributed AMR hierarchy  \\
\REF{solution:parallel-onesided} & \TAG{solution:parallel-onesided} & MPI-2 one-sided not MPI-1 two-sided  \\
\REF{solution:parallel-dynamic-procs} & \TAG{solution:parallel-dynamic-procs} & Dynamic process creation / deletion  \\
\REF{solution:parallel-dynamic-tasks} & \TAG{solution:parallel-dynamic-tasks} & Dynamic task allocation  \\
\REF{solution:parallel-data-analysis} & \TAG{solution:parallel-data-analysis} & Weakly-coupled parallel data analysis  \\
\REF{solution:parallel-subblocks} & \TAG{solution:parallel-subblocks} & Grid patch subblocks  \\
\REF{solution:method-p3dfft} & \TAG{solution:method-p3dfft} & P3DFFT for unigrid gravity \\
\REF{solution:method-hypre-fac} & \TAG{solution:method-hypre-fac} & HYPRE FAC for AMR gravity and radiation  \\
\REF{solution:performance-lcaperf} & \TAG{solution:performance-lcaperf} & \lcaperf\ performance monitoring  \\
\REF{solution:data-io-asynch} & \TAG{solution:data-io-asynch} & Asynchronous I/O  \\
\REF{solution:memory-management} & \TAG{solution:memory-management} & Large mallocs  
\end{tabular}

%-----------------------------------------------------------------------
\subsection{Grid class refactoring} \label{solution:amr-grid-refactor}
%-----------------------------------------------------------------------

For moderate numbers of processors, the current datastructure used for
storing the AMR grid hierarchy is adequate.  Even though the hierarchy
topology (i.e. metadata) is stored redundantly on each processor, the
extra memory overhead involved is insignificant because the data
fields are vastly larger than \enzo's individual C++ \code{grid} objects.
However, as the number of processors increases, this memory overhead
increases as well.  For the processor counts required for
petascale-level computing, the storage overhead would dominate,
to the point where we would be memory rather than cpu limited.
Thus, for \enzo\ to scale to the petascale level, the memory overhead for
storing the AMR hierarchy must be reduced.

\newcommand{\numgrids}{n}
\newcommand{\numprocs}{p}
\newcommand{\sizegrid}{|G|}
\newcommand{\sizegridlocal}{|G_l|}
\newcommand{\sizegridremote}{|G_r|}
\newcommand{\sizefield}{|F|}
\newcommand{\sizefields}{\sum_i^n |F_i|}

The memory required for storing \enzo's current AMR
datastructure can be approximated as $\sizefield + \numgrids \numprocs
\sizegrid$, where the first term $\sizefield$ is the field
variable data, and the second term $\numgrids \numprocs \sizegrid$ is
the overhead for storing the grid hierarchy data structure.  Here
$\numgrids$ is the number of grid patches, $\numprocs$
is the number of processors, and $\sizegrid$ is the storage required
by a C++ \code{grid} object.


One approach to reducing the size of the overhead term would be
to split the \code{grid} class into two subclasses \code{grid\_local}
and \code{grid\_remote}, and use  \code{grid\_local} objects for local grids
that contain field data, and \code{grid\_remote} objects for grid
patches whose data fields reside on another processor.
This modification would change the overhead storage term from
$\numgrids \numprocs \sizegrid$ to $\numgrids ((\numprocs-1)
\sizegridremote + \sizegridlocal$), where $\sizegridremote$ is the
size of the \code{grid\_remote} class and $\sizegridlocal$ is the size
of the \code{grid\_local} class.  The advantage of doing this is that the
\code{grid\_remote} class would be made much smaller, since most of
the variables in the \code{grid} class are only required for local
grids.  Also, a vast majority of the grid classes are these much
smaller \code{grid\_remote} objects.  Thus the memory savings for this
modification would be quite large---a factor of roughly $15$ over the
current approach.

%-----------------------------------------------------------------------
\subsection{Cached AMR datastructure}  \label{solution:amr-cache}
%-----------------------------------------------------------------------
%-----------------------------------------------------------------------
\subsection{Favor larger grids}\label{solution:amr-large-grids}
%-----------------------------------------------------------------------


   Reduces size of hierarchy datastructure
   Reduces ghost zone overhead: reduced memory and communication
   Improves computation / datastructure manipulation ratio
   Reduced frequency of hierarchy rebuilding
%-----------------------------------------------------------------------
\subsection{Dynamic ghost zones} \label{solution:amr-dynamic-ghosts}
%-----------------------------------------------------------------------

  \enzo's PPM hydrodynamics algorithm requires a grid to have a level
  of ghost zones three deep to store data from neighboring and parent
  grids, which may lie on different processors.  The memory overhead
  required for storing these ghost zones can be very high, especially
  when the grid is small.  The ``half-efficiency'' point in terms of
  memory storage for cubical grid patch is $23^3$, meaning that a grid
  of that size has roughly the same number of active zones as ghost
  zones, and is higher for non-cubical grids.

  Currently, these ghost zones are stored permanently, allowing all
  communication to be done in a single step before the hydrodynamics
  solve.  However, since the ghost zones are only required in isolated
  locations in the code, and the values are always refreshed before
  the ghost zones are accessed, they do not need to be stored
  permanently.  The opposite extreme to storing ghost zones
  permanently would be to allocate them a grid patch at a time only
  when needed, and deallocated as soon as they are no longer required.
  In this case, the number of grids with ghost zones would be at most
  the number of active parallel threads.

  Although this would improve memory use significantly, it would
  unfortunately be extremely inefficient due to communication latency.
  Currently, with all ghost zones in all grids being refreshed
  together, this latency is hidden.

  However, it would be possible to optimize storage use versus hiding
  communication latency by partitioning grids into some small number
  of groups, and allocating ghost zones only a group at a time.  We
  propose to try this.

%-----------------------------------------------------------------------
\subsection{Octree} \label{solution:amr-octree}
%-----------------------------------------------------------------------
%-----------------------------------------------------------------------
\subsection{Recursive Binary Box Tree (RBBT)}  \label{solution:amr-boxtree}
%-----------------------------------------------------------------------
%-----------------------------------------------------------------------
\subsection{Quantized grid patch sizes} \label{solution:amr-grid-quantized}
%-----------------------------------------------------------------------
%-----------------------------------------------------------------------
\subsection{Load balance trees not patches} \label{solution:amr-balance-trees}
%-----------------------------------------------------------------------

%-----------------------------------------------------------------------
\subsection{Load balance using performance measurements}  \label{solution:amr-balance-performance}
%-----------------------------------------------------------------------

%-----------------------------------------------------------------------
\subsection{Load balance using space-filling curves}  \label{solution:amr-balance-hilbert}
%-----------------------------------------------------------------------

The current load balancer allocates new patches on the same
processor as the parent patch and then moves them as required in an
attempt to retain parent-sibling locality.  However, performance
analysis shows that most communication is done to sibling grids, and
so we will develop an alternate load balancer which uses a space
filling curve (such as a Hilbert or morton curve) in order to both
load balance and maintain maximal sibling locality.  This technique
has been used successfully in other SAMR applications
\cite{WHH03,LWBH07}.


%-----------------------------------------------------------------------
\subsection{Hierarchical load balancing}\label{solution:amr-balance-hierarchical}
%-----------------------------------------------------------------------

Finally, Enzo's current load balancer does not address deep-memory
hierarchies, multi-core architectures and heterogeneous/hybrid design
that are common in the emerging HPC systems. On multi-core based
systems, the communication cost across nodes may be several orders of
magnitude higher than that within the shared memory nodes. To reduce
inter-node communication, the load balancer will allocate related
subgrids, e.g. parent and sibling subgrids, within the shared memory
nodes.  Within the current MPI implementation, this can be achieve
simply through the correct ordering of nodes within the space filling
curve described earlier.  Below, we will describe a more comprehensive
approach to the new hybrid architecture.

%-----------------------------------------------------------------------
\subsection{Traverse local grids only}\label{solution:amr-traversal-local}
%-----------------------------------------------------------------------

  \enzo's datastructures are currently implemented such that a
  processor traverses a linked list of all grids in a level, tests
  each grid to see whether it is assigned to the given processor, and
  only perform computations on the grid data if the grid is assigned
  to the processor, otherwise it continues on to the next grid in the
  list.  For relatively small numbers of processors this is
  reasonable, but for very large numbers of processors, the overhead
  of traversing the entire global hierarchy is not scalable.

  An alternative would be to generate a second linked list containing
  only grids assigned to this processor.  This would eliminate the
  need for subsequent testing of whether a grid is assigned to the
  processor, and would also shorten the linked list by a factor of
  $P$, the number of processors.

%-----------------------------------------------------------------------
\subsection{Reuse existing grid classes} \label{solution:amr-grid-reuse}
%-----------------------------------------------------------------------
%-----------------------------------------------------------------------
\subsection{Inter-level refinement} \label{solution:amr-balance-split}
%-----------------------------------------------------------------------
When load balancing two nested grids, instead of
sending child grid to another processor, split grids
in ``half'' to maintain parent-child locality.
%-----------------------------------------------------------------------
\subsection{Particle groups} \label{solution:particles-group}
%-----------------------------------------------------------------------
%-----------------------------------------------------------------------
\subsection{Hybrid parallelism}\label{solution:parallel-hybrid}
%-----------------------------------------------------------------------
   ``basic hybrid will make stack replication worse''
   HYPRE can use OpenMP

  Switching from MPI to hybrid MPI + OpenMP could help \enzo's
  communication efficiency on hardware's hierarchical parallelism.
  Studies have shown that the overhead of OpenMP due to repeated
  thread spawning and false-sharing of cache lines can reduce
  performance.  Using OpenMP in a coarse-grain manner, by spawning
  threads just once at the beginning of a simulation, would control
  the parallel loop overhead.  While it's not known whether this
  hybrid approach would always be more efficient than stand-alone MPI,
  it will probably still be more efficient on some archituctures, or
  with some implementations of MP or OpenMP.  Assuming \enzo\ could
  still be run in MPI-only mode, adding the option to use MPI+OpenMP
  would only increase \enzo's efficiency.

  Another improvement would be to distribute the entire grid hierarchy
  on each processor.  The easiest approach would be to store one copy
  per shared memory node, which would decrease the memory storage
  further by a factor equal to the number of processors per node. This
  could easily be implemented with one of the hybrid parallel models
  described in the next section.

   While the above modifications to the AMR datastructure should allow
   \enzo\ to run on machines with on the order of $10^4$ to $10^5$
   processors, extending to $10^6$ processors may require reducing the
   overhead even further.  The ultimate improvement memory-wise would
   be to store a single copy of the grid hierarchy, though depending
   on the node interconnect that would cause a communication
   bottleneck.  A refinement to this approach would be to store one
   copy of the hierarchy for every $M$ processors, where $M$ is some
   machine-dependent number chosen to balance the tradeoff between
   memory and communication overhead.

  The emergence of multi-core architectures will profoundly change
  parallel programming in high performance computing. Although the
  question of what is the best programming model for multicore
  architectures is largely unsettled, it is widely accepted that there
  are two common ways for parallel programming in multicore systems.
  One is \emph{the flat MPI model}, adopted by the current version of
  Enzo, in which separate single-threaded MPI processes are executed
  on each processing core. The other is called \emph{hybrid
  programming model}, where one MPI process is used per node and a
  multi-threaded, shared memory approach is exploited in the multiple
  cores. For example, OpenMP, Pthreads or the recently developed
  UPC/CAF may be used with MPI.  The disadvantage of the flat MPI
  model with hierarchical architectures is that it involves
  unnecessary overhead to copy data within a node.

%-----------------------------------------------------------------------
\subsection{PGAS distributed AMR hierarchy} \label{solution:parallel-pgas}
%-----------------------------------------------------------------------
   replace explicit messaging with UPC or CAF
   can still have redundancy in representation to reduce communication
   See cached
%-----------------------------------------------------------------------
\subsection{MPI-2 one-sided not MPI-1 two-sided}  \label{solution:parallel-onesided}
%-----------------------------------------------------------------------

  Switching from using MPI send/receives exclusively for all
  processor-to-processor communication to MPI2's one-sided
  communication would reduce synchronization overhead.  Additionally,
  it could reduce data movement and intermediate buffers, and would
  simplify the programming, since only one side of the transfer needs
  to be programmed instead of both.

%-----------------------------------------------------------------------
\subsection{Dynamic process creation / deletion} \label{solution:parallel-dynamic-procs}
%-----------------------------------------------------------------------

   For error fault tolerance

%-----------------------------------------------------------------------
\subsection{Dynamic task allocation} \label{solution:parallel-dynamic-tasks}
%-----------------------------------------------------------------------
%-----------------------------------------------------------------------
\subsection{Weakly-coupled parallel data analysis}\label{solution:parallel-data-analysis}
%-----------------------------------------------------------------------
%-----------------------------------------------------------------------
\subsection{Grid patch subblocks} \label{solution:parallel-subblocks}
%-----------------------------------------------------------------------

%-----------------------------------------------------------------------
\subsection{P3DFFT for unigrid gravity} \label{solution:method-p3dfft}
%-----------------------------------------------------------------------

%-----------------------------------------------------------------------
\subsection{HYPRE FAC for AMR gravity and radiation}  \label{solution:method-hypre-fac}
%-----------------------------------------------------------------------
%-----------------------------------------------------------------------
\subsection{\lcaperf\ performance monitoring} \label{solution:performance-lcaperf}
%-----------------------------------------------------------------------

\lcaperf\ is a small yet practical toolkit
 designed to measure a wide range of performance-related data,
 including performance counter hardware data (via PAPI~\cite{BrDo00}),
 MPI communication (via MPI's profiling interface), dynamic memory
 allocation (via overloaded C++ \code{new} and \code{delete}
 operators), and software datastructure-related metrics, e.g.~number
 of grids, number of grid zones, or number of timesteps per level (via
 user-defined counters), all for multiple user-defined code sections.
 \lcaperf's very flexible post-processing utilities in turn can
 generate almost arbitrary derived metrics, by applying arithmetic,
 finite-difference, quadrature, interpolation, and reduction
 operations to the collected performance data.  Since derived metrics
 can be defined in terms of multiple runs of the application, derived
 quantaties trivially encompass such metrics as load balance
 efficiency histories or parallel efficiency histories.  \lcaperf\ can
 in turn plot any derived metric versus any other derived metric via
 the \code{gnuplot} package.  \lcaperf\ is thus ideally suited to
 monitoring the comprehensive parallel performance of large-scale

 Our plan is to extend \lcaperf\ in three ways.  First, to support
 additional parallelism models, such as some subset of OpenMP, HPC, or
 COF, so that \lcaperf\ can more directly monitor the effectiveness of
 the proposed hybrid parallelization enhancements to \enzo.  Second,
 to add support in the API to allow the application to access its own
 collected performance data, which would enable \enzo\ to monitor its
 own performance, and hence adapt to it.  This functionality could be
 used immediately in the load-balancing module, which requires
 accurate performance estimates.  Third, to improve the useability of
 the post-processing utilities.  Currently the utilities are
 collectively powerful, but rather low-level and tedious for the user.
 We plan to provide higher-level utilites, controlled by user input
 files, that will ultimately allow the user or automatic regression
 testing suite to directly generate a comprehensive performance report
 tailored to the application for a collection of related runs.

  \note{Differences between parallel levels may require different
  parallel communication technology: processors in node share memory;
  cores within processors share memory bandwidth}

%-----------------------------------------------------------------------
\subsection{Asynchronous I/O} \label{solution:data-io-asynch}
%-----------------------------------------------------------------------

%-----------------------------------------------------------------------
\subsection{Large mallocs} \label{solution:memory-management}
%-----------------------------------------------------------------------


\begin{verbatim}
Enzo scaling issues
Many opportunities for improving scaling, though one stands out above others.
  Main: AMR hierarchy representation
     grid refactoring: reduce the hierarchy storage
        Cite Fowler ``Refactoring'': ``Extract Hierarchy'' refactorization.
        grid a ``swiss army knife'' class
        grid -> grid\_local grid_remote
     MPI + OpenMP hybrid: reduce the number of copies of the hierarchy
        OpenMP can by dynamically threaded
        Enzo was originally thread parallel (SGI)
     UPC for distributing storage of hierarchy representation as well as data
    store required neighbor information (local and adjacent grids only)
    update less frequently (rebuild less often, and expense of larger grids)
  load balancing
     hierarchical parallelism: balance between nodes as well as within nodes

  Performance evaluation     
     evaluate using lcaperf
       monitors MPI communication via PMPI
       monitors computation (floating point and memory accesses) via PAPI
       monitors HDF5 I/O through user-defined metrics
       monitors dynamic memory allocation by overloading new/delete: current,
         and high-water marks
       Flexible quantization of performance data: currently can accumulate global
          metrics, or metrics per component, per timestep, and per
          hierarchy level

  External libraries
     HDF5
     HYPRE
     lcaperf: side project to Enzo, optional, will require modifications
       to support OpenMP, and it writes one file per MPI process
    
  particle movement
  rebuild hierarchy
  boundary update
  allocate field data in chunks
  memory fragmentation
     separate allocation of field data and grid classes
     many small grid classes, but uniform size
  

HYPRE 
   scalability


Collins:

\enzo\  has allowed us to maximize the resources currently available in
high performance computing.  We are confident that the improvements
outlined above, namely reducing the hierarchy overhead and hybrid
MPI/OpenMP, will carry us into Peta Scale computing quite well.
However, we feel that it is important to begin anticipating problems
we might encounter as we reach towards 200,000 cores, as well as ways
to most efficiently utilize the machinery not yet available to us.

They physics modules in \enzo\  scale extremely well up to all measurable
processor counts, so the last remaining roadblocks all entail grid
accounting and manipulation mechanisms. The first is the load
balancing.  While \enzo's load balancing has worked well through the
Tera Scale Computing era, the advent of hierarchical parallelism as
will be seen in Blue Waters offers a new advantage, in that shorter
communication times within a node/super node can be better taken
advantage of for grid patches that require a relatively high
communication traffic between them.  To this end, we will be making
the load balancing aware of the communication cost of distribution, as
well as the computation and memory costs.  We will also be utilizing
the dynamic analysis capability of lcaPerf to monitor the simulation's
needs, and adapt the load balancing accordingly.

We anticipate that the next area of improvement needs (because lets
face it, a code like this will always have improvements to be made)
will be in the routines that examine and rebuild \enzo's dynamic
hierarchy of subgrids.  There are two parts that rely heavily on
communication that, depending on how UPC and MPI2 work in combination
with the extant machinery, may be bottle necks to performance.  The
first is the broadcasts of new grids that are generated at every
timestep, and the second is the copy of data from old grids to new.
While these will of course be improved by the refactoring of the
hierarchy and the introduction of hybrid MPI/OpenMP, if they prove to
still be problematic, we are working on tools to avoid these problems
now, so if they do become a problem our performance isn't stalled very
long.  These include altering the rebuild schedule, while anticipating
and mitigating the loss of accuracy that may come from not rebuilding
every timestep, and creating new processor/grid relational mappings to
reduce the amount of work needed to determine communication needs.
\end{verbatim}

%==================================================================
\end{document}
%=======================================================================

