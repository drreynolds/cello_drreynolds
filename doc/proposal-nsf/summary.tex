%%%%%%%%% MASTER -- compiles the 4 sections

\documentclass[11pt,letterpaper]{article}

%%%%%%%%%%%%%%%%%%%%%%%%%%%%%%%%%%%%%%%%%%%%%%%%%%%%%%%%%%%%%%%%%%%%%%%%%
\pagestyle{plain}                                                      %%
%%%%%%%%%% EXACT 1in MARGINS %%%%%%%                                   %%
\setlength{\textwidth}{6.5in}     %%                                   %%
\setlength{\oddsidemargin}{0in}   %% (It is recommended that you       %%
\setlength{\evensidemargin}{0in}  %%  not change these parameters,     %%
\setlength{\textheight}{8.5in}    %%  at the risk of having your       %%
\setlength{\topmargin}{0in}       %%  proposal dismissed on the basis  %%
\setlength{\headheight}{0in}      %%  of incorrect formatting!!!)      %%
\setlength{\headsep}{0in}         %%                                   %%
\setlength{\footskip}{.5in}       %%                                   %%
%%%%%%%%%%%%%%%%%%%%%%%%%%%%%%%%%%%%                                   %%
\newcommand{\required}[1]{\section*{\hfil #1\hfil}}                    %%
\renewcommand{\refname}{\hfil References Cited\hfil}                   %%
\bibliographystyle{plain}                                              %%
%%%%%%%%%%%%%%%%%%%%%%%%%%%%%%%%%%%%%%%%%%%%%%%%%%%%%%%%%%%%%%%

\usepackage{epsfig}
\usepackage{url}

% \newcommand{\cello}{\textsf{Cello}}
% \newcommand{\enzo}{\textsf{Enzo}}
% \newcommand{\enzoii}{\textsf{Enzo}-\texttt{II}}
% \newcommand{\lcaperf}{\textsf{lcaperf}}
% \newcommand{\lcatest}{\textsf{lcatest}}
% 
% \newcommand{\pp}{\texttt{++}}
% \newcommand{\cpp}{C\pp}
% \newcommand{\charm}{\textsf{Charm\pp}}
% \newcommand{\chombo}{\textsf{Chombo}}
% \newcommand{\samrai}{\textsf{SAMRAI}}
% \newcommand{\paramesh}{\textsf{PARAMESH}}
% \newcommand{\gadget}{\textsf{GADGET}}
% \newcommand{\alps}{\textsf{ALPS}}
% \newcommand{\clawpack}{\textsf{CLAWPACK}}
% \newcommand{\grace}{\textsf{GrACe}}
% \newcommand{\carpet}{\textsf{Carpet}}
% \newcommand{\flash}{\textsf{FLASH}}
% 
% \newcommand{\code}[1]{\textsf{#1}}
% 
% \newcounter{figctr}
% 
% \newcommand{\FIGURE}[3]{
% \noindent
% \parbox{\textwidth}{
% %\ \\ \hrule \ \\
% \begin{center}
% #3
% \end{center}%
% \ \nolinebreak%
% \refstepcounter{figctr}%
% \begin{center}%
% \begin{minipage}{7.0in}
% \textbf{Figure \thefigctr}. #1
% \end{minipage}
% \end{center}
% \label{#2}
% %\ \\ \hrule \ \\
% }}


% INCLUDE DEVELOPMENT TEXT

\newcommand{\devel}[1]{\textbf{#1}}

% EXCLUDE DEVELOPMENT TEXT

% \newcommand{\devel}[1]{}


%=======================================================================
% Commands
%=======================================================================

%\newcommand{\SUBSUBSECTION}[1]{\subsubsection{#1} }
\newcommand{\SUBSUBSECTION}[1]{\textbf{#1.}~}

\newcommand{\cello}{\textsf{Cello}}
\newcommand{\enzo}{\textsf{Enzo}}
\newcommand{\enzoii}{\mbox{\textsf{Enzo}\nolinebreak-\nolinebreak\texttt{P}}}
\newcommand{\lcaperf}{\textsf{lcaperf}}
\newcommand{\lcatest}{\textsf{lcatest}}

\newcommand{\pp}{\texttt{++}}
\newcommand{\cpp}{C\pp}
\newcommand{\charm}{\textsf{Charm\pp}}
\newcommand{\chombo}{\textsf{Chombo}}
\newcommand{\samrai}{\textsf{SAMRAI}}
\newcommand{\paramesh}{\textsf{PARAMESH}}
\newcommand{\gadget}{\textsf{GADGET}}
\newcommand{\clawpack}{\textsf{CLAWPACK}}
\newcommand{\grace}{\textsf{GrACe}}
\newcommand{\alps}{\textsf{ALPS}}
\newcommand{\carpet}{\textsf{Carpet}}
\newcommand{\flash}{\textsf{FLASH}}

\newcommand{\samr}{SAMR}
\newcommand{\camr}{CAMR}
\newcommand{\uamr}{UAMR}
\newcommand{\SAMR}{patch-based AMR}
\newcommand{\CAMR}{octree-based AMR}
\newcommand{\UAMR}{unstructured mesh AMR}

\newcommand{\code}[1]{\textsf{#1}}
% not note because conflicts with beamer
\newcommand{\note}[1]{\devel{\eighthnote\ \textit{#1} \\}} 
\newcommand{\pargraph}[1]{\devel{\P\ \textbf{#1} \\}}

\newcommand{\todo}{\devel{$\circ$}}
\newcommand{\done}{\devel{$\bullet$}}
\newcommand{\useyes}{\textcolor{green}{$\bullet$}}
\newcommand{\usemaybe}{\textcolor{yellow}{$\bullet$}}
\newcommand{\useno}{\textcolor{red}{$\bullet$}}
\newcommand{\halfdone}{\devel{\textcolor{gray}{$\bullet$}}}

%=======================================================================

\newcommand{\TITLE}[3]{
\title{ {\huge #1}  \\ \vspace{0.1in}
     {\small Document Version: #3} \vspace{-0.1in}
    }
\author{      #2 \\
        San Diego Supercomputer Center \\
        Laboratory for Computational Astrophysics\\
        University of California, San Diego}
\maketitle}

%=======================================================================
\newcounter{figctr}

% \newcommand{\FIGURE}[3]{
% \noindent
% \parbox{\textwidth}{
% %\ \\ \hrule \ \\
% \begin{center}
% #3
% \end{center}%
% \ \nolinebreak%
% \refstepcounter{figctr}%
% \begin{center}%
% \begin{minipage}{\textwidth}
% \textbf{Figure \thefigctr}. #1
% \end{minipage}
% \end{center}
% \label{#2}
% %\ \\ \hrule \ \\
% }}

\newcommand{\FIGURE}[3]{
\begin{figure}[htp]
  \begin{center}
    #3
  \end{center}
  \caption{#1}
  \label{#2}
\end{figure}}

\newcommand{\FIGURETWOCOL}[3]{
\noindent
%\ \\ \hrule \ \\
\begin{center}
#3
\end{center}%
\ \nolinebreak%
\refstepcounter{figctr}%
\begin{center}%
\begin{minipage}{3in}
\textbf{Figure \thefigctr}. #1
\end{minipage}
\end{center}
\label{#2}
%\ \\ \hrule \ \\
}



% NSF proposal generation template style file.
% based on latex stylefiles  written by Stefan Llewellyn Smith and
% Sarah Gille, with contributions from other collaborators.

\DeclareFontFamily{OT1}{psyr}{}
\DeclareFontShape{OT1}{psyr}{m}{n}{<-> psyr}{}
\def\times{{\fontfamily{psyr}\selectfont\char180}}


\renewcommand{\refname}{\centerline{References cited}}

% this handles hanging indents for publications
% \def\rrr#1\\{\par
%\medskip\hbox{\vbox{\parindent=2em\hsize=6.12in
%\hangindent=4em\hangafter=1#1}}}

\def\baselinestretch{1}

 \addtolength{\parskip}{\baselineskip}
\setlength{\parindent}{0pt}

% \setcounter{secnumdepth}{2}

%-----------------------------------------------------------------------
% NSF
%-----------------------------------------------------------------------
%
% FIVE SOFTWARE FOCUS AREAS
%
%  * 1. software for HPC systems
%    2. software for digital data management
%    3. software for broadband and networking
%    4. middleware
%    5. cybersecurity
%
% CROSS CUTTING ISSUES
%
%  * software sustainability
%
%  * software self-manageability
%
%  - software power/energy efficiency
%
%
% HPC SOFTWARE ISSUES
%
%  * deep-memory hierarchies
%  * multi-core architectures
%  * heterogeneous/hybrid systems
%  * architecture agnostic
%
% SDCI REQUIREMENTS
%
%  * identify software focus area and category in title
%  * support for at least one cross-cutting issue
%  * identify multiple application areas in science or engineering
%    - missing capability required
%    - specific examples of how tool will impact science research
%  * clear description of how approach compares to existing approaches
%  * explicit outreach and education plan for additional end user groups
%  * explicit description of the engineering process used
%    - design, development, release, deployments, tool interoperability,
%    - evaluation plan that includes end users [ref nmi.cs.wisc.edu]
%  * list of tangible metrics to measure success, with end users
%    - quantitative + qualitative definition of "working prototype"
%    - steps from prototype to production use
%  * compelling discussion of software's potential use by broader communities
%    - use cases with relevant domain scientists
%  * sustainability plan beyond the award lifetime
%  * identify open source licence
%    
%-----------------------------------------------------------------------


% Project summary
% Introduction
%    Extreme parallelism
%    Extreme AMR
%    Existing AMR frameworks
%       PARAMESH
%       Chombo
%       SAMRAI
%       GADGET
% Software requirements
% Software design
%    High level components
%    Data structures
%       Patch coalescing  for ``shallow'' AMR
%       Targeted refinement with backfill for ``deep'' AMR
%    Task scheduling
%    Load balancing
% Implementation
%    Parallelism
%    Fault tolerance
%    Software implementation
% Development plan
% Milestones and deliverables

\usepackage{natbib}

\newcommand{\delete}[1]{}

%=======================================================================

\begin{document}

\noindent
\textbf{Project Summary}

% Cello + Enzo-II
%
%   formalization of Enzo-P development as a physics application built
%   on a new independent extreme AMR framework
%
%
%
This project proposes to develop \cello, a new parallel adaptive mesh
refinement (AMR) software framework.  The purpose of \cello\ is to
enable researchers to write multi-physics AMR applications that can
harness the enormous computing power of current and future world-class
HPC platforms.  The distinguishing characteristic relative to existing
AMR frameworks is the aggressive pursuit at the onset of extreme
scalability, both in terms of software data structures and hardware
parallelism.
%
Integral to developing \cello\ will be developing \enzoii, a petascale
astrophysics and cosmology application built on top of the \cello\
framework.  \enzoii\ will not only help drive development of the
underlying \cello\ framework, but it will serve as a highly scalable
variant of the \enzo\ terascale astrophysics and cosmology community
code.  Both the \enzoii\ science application, and the underlying
independent \cello\ parallel AMR software framework, will be released
and supported as community software.

\textit{Software sustainability} will be realized under the dual
support of the Laboratory for Computational Astrophysics (LCA) at the
University of California San Diego, and the San Diego Supercomputer
Center, organizations devoted to the long-term maintenance of---and
user support for---community scientific codes and HPC cyberinfrastructure.
%
\textit{Software self-management} will be an integral component of the
\cello\ software design, with software resilience a high priority.
The \cello\ framework will be designed to detect hardware and software
faults, identify performance and numerical issues, and dynamically
reconfigure to always perform with the highest possible efficiency on
currently available hardware components.
%
\textit{Energy efficiency} can be considered implicit in the
underlying adaptive mesh refinement approach, which dynamically
targets computational resources where they are required, and avoids
expending resources where they are not.  The adaptivity of AMR
translates directly to energy savings.

%-----------------------------------------------------------------------
% The proposed AMR framework would enable application
% developers to write multiphysics applications for simulating phenomena
% on an unprecedented range of spacial and temporal scales.

% AMR will be implemented using a modified octree-based approach.  Nodes
% of an octree will be associated with both logically Cartesian grid
% blocks and groups of particles, for Eulerian, Lagrangian, and hybrid
% methods for solving coupled hyperbolic, elliptic, and parabolic PDE's.
% Unlike standard octree-based AMR, grid block sizes may vary, and
% individual blocks may be distributed, such that optimal ``unigrid''
% parallel performance is recovered in subregions of the problem domain
% that are smooth.  Additionally, higher refinement factors, such as
% $r=4$ or even $r=8$, will be supported for particularly ``deep'' AMR
% problems; this will be designed in a way that still maintains at most
% $r=2$ level jumps, and will retain the high efficiency of the octree
% data structure.

% Parallelization will be primarily data parallel, with task
% parallelization also available to augment the data parallelism.  A
% variety of parallelization technologies will be supported from the
% start, beginning with the message-driven processor virtualization
% approach provided by the
% \charm\ framework~\cite{KaBo07}~\cite{wwwcharm}.  Other
% parallelization methods used will include one-sided message passing
% via the MPI-2.0 library~\cite{wwwmpi}, the partitioned global address
% space (PGAS) approach via the UPC language~\cite{wwwupc}~\cite{upc},
% and shared memory parallel programming using the OpenMP
% API~\cite{wwwopenmp}.  Hybrid parallelism will be supported, including
% MPI with OpenMP and UPC with MPI.  Dynamic load balancing of parallel
% tasks will include hierarchical methods, which will improve mapping
% the distributed software data structures to the hierarchical hardware
% components of HPC platforms.

% Fault tolerance and software resilience are also crucial factors at
% extreme scales.  The checkpoint-to-disk paradigm is known to be
% ultimately non-scalable, so this project will support other alternatives,
% including checkpoint-to-memory, or other options as they become
% technologically feasible.  Our framework will be designed at the onset
% to be resistant to memory, compute component, network, disk, and
% software failures.
% 
% This project plans to make the framework and accompanying documentation freely
% available for use by the research community.  The result of this
% project will be a software framework that allows application
% developers to write multiphysics applications for simulating phenomena
% on an unprecedented range of spacial and temporal scales for many
% years to come.

% The proposed AMR framework would enable application
% developers to write 

%-----------------------------------------------------------------------

[@@@ intellectual merit]


%-----------------------------------------------------------------------

% \begin{itemize}
% \item importance of advancing knowledge / understanding in field
% \item qualifications of team
% \item extent of creative, original, transformative concepts
% \item how well-conceived and organized
% \item sufficient access to resources
% \end{itemize}

%-----------------------------------------------------------------------

[@@@ broader impact]


%-----------------------------------------------------------------------

% \begin{itemize}
% \item advance discovery and understanding while promote teaching, training, learning
% \item broaden underrepresented groups
% \item enhance research / education infrastructure
% \item disseminate results to enhance science and technology
% \item benefits to society
% \end{itemize}

\end{document}

%==================================================================

