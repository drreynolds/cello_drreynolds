%======================================================================@
\chapter{Running \cello} \label{c:running}
%======================================================================@

\cello\ is run using either a script or through a GUI portal (that runs
the script).  \cello\ is run using a script instead of directly for
the following reasons:

\begin{itemize}
\item It encapsulates machine-dependent parallel job submission procedures
\item It allows machine configuration-dependent settings, such as number of nodes, processors, and cores to be defined in the input file instead of separately in a job submission script.
\item It controls running multiple executables, such as data migration, visualization, or 
\item It can store metadata for later use, such as code version, compile options, library versions, etc., that can be useful but tedious or error-prone for users to perform manually
\item It can verify the correctness of the input file before starting the job, saving the user time, or at least having to remember to check.
\end{itemize}

\begin{tabular}{|lll|} \hline
\textbf{Command} & \textbf{Arguments} & \textbf{Description} \\ \hline
\code{cello\_run} & \textit{cello-input-file} & Run \cello\ interactively \\
\code{cello\_submit} & \textit{cello-input-file} & Submit \cello\ to a batch queue \\
\code{cello\_monitor} & & Text monitor for a \cello\ run \\
\code{cello\_view} & & GUI monitor for a \cello\ run \\ \hline
\end{tabular}
