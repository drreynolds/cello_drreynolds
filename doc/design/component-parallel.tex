
%=======================================================================
\section{Parallel Component} \label{s:component-parallel}
%=======================================================================

Hardware platform parallelism will be considered to be multilevel,
including nodes, processors, and cores.  Computational tasks will be
flexibly organized into hierarchical levels to aid mapping to multiple
hardware parallelization levels, including grid patches, grid patch
subblocks, and multiple simulations.  Task sizes in different levels
will allow flexibility to help optimize granularity for the different
given parallelization level components.

Flexible parallelism paradigms: map parallelism to tasks

Automatic code generation (or other) of parallel tasks to implement
parallelism and optimize performance

Specify parallelism type and parameters.  For example, non-blocking
MPI, MPI-2, hybrid MPI/UPC, performance-related parameters such as
buffer size, etc.

Specifiy parallelism and parameters

Specifiy method for controling parallelism

   parallelization method (MPI buffered/blocking, MPI2 Get)

\begin{itemize}
\item MPI (send/recv and type, one-sided and type, what level)
\item OpenMP (num threads, what level)
\item UPC (num threads, what level)
\item pthreads (num threads, what level)
\item cooperative parallelism
\item levels for each if multiple
\end{itemize}

%-----------------------------------------------------------------------
\subsection{Use Cases}
%-----------------------------------------------------------------------
%-----------------------------------------------------------------------
\subsection{Parameters}
%-----------------------------------------------------------------------

%=======================================================================
\subsection{MPI Send/Recv}

%=======================================================================
\subsection{MPI2 Get}

%=======================================================================
\subsection{OpenMP}

%=======================================================================
\subsection{Collaberative parallelism}

%=======================================================================
\subsection{Pipelining}


