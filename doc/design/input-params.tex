%=======================================================================
\section{Parameter List} \label{s:params}
%=======================================================================

Parameters are organized into sections.  Sections are listed below,
and parameters in each section are listed in subsequent subsections.
Sections are closely associated with \cello\ components, described
in \S\ref{c:components}.

\begin{tabular}{|ll|} \hline
\textbf{Section} & \textbf{Description} \\ \hline
\code{Ensemble}   & Parameters defining an ensemble of simulations \\
\code{Simulation} & Parameters defining a single simulation \\
\code{Control}    & Parameters defining high-level control \\
\code{Physics}    & Parameters defining physics \\
\code{Units}      & Parameters defining units \\
\code{Problem}    & Parameters defining the problem setup \\
\code{Domain}     & Parameters defining the problem domain \\
\code{Region}     & Parameters defining a subregion of the domain \\
\code{Boundary}    & Parameters defining and initializing boundary conditions \\
\code{Field}       & Parameters defining and initializing a numerical field \\
\code{Matter}      & Parameters defining matter properties \\
\code{Method}      & Parameters controlling a specific numerical method \\
\code{Grid}        & Parameters controlling the AMR grid \\
\code{Particle}    & Parameters controlling particles\\
\code{Parallel}    & Parameters controlling parallelism \\
\code{Output}      & Parameters controlling field data output \\
\code{Monitor}     & Parameters controlling user-monitoring information \\
\code{Performance} & Parameters controlling performance monitoring \\
\code{Error}       & Parameters controlling fault-tolerance and error recovery \\
\hline
\end{tabular}


%-----------------------------------------------------------------------
\subsection{Ensemble parameters}
%-----------------------------------------------------------------------
%-----------------------------------------------------------------------
\subsection{Simulation parameters}
%-----------------------------------------------------------------------

%-----------------------------------------------------------------------
\subsection{Control parameters} \label{ss:params-control}
%-----------------------------------------------------------------------

Given Physics, Algorithms, and Data structures, specify the top-level
sequencing and properties of the simulation.  For example, ordering of
physics modules, whether to do hierarchical time-stepping, up to what
level, whether to sub-cycle some physics, etc. [Is this a useful
category?]  Also include things like floors and limits(?), and IO
dumps

Global simulation control.

Output types and parameters

\begin{itemize}
\item checkpoint (dump all)
\item output (specific fields)
\item movies (type and rate)
\item analysis (type of analysis, rate)
\item level of output (files for timestep, time, etc.)
\end{itemize}

%-----------------------------------------------------------------------
\subsection{Physics parameters} \label{ss:params-physics}
%-----------------------------------------------------------------------

Specify physics modules and physics parameters, including
hydrodynamics, self-gravity, gravitational constant, imposed gravity,
chemistry, cosmological expansion, star formation, etc.  Physics is in
the problem domain.

Specify physics components

\begin{itemize}
\item hydrodynamics
\item  cosmological expansion
\item self-gravity
\end{itemize}


%-----------------------------------------------------------------------
\subsection{Units parameters} \label{ss:params-units}
%-----------------------------------------------------------------------


 Specify units and optional scalings for individual
 fields.  [Merge units with Control?] [Merge scaling with Fields?] 
 [Dynamic scaling, e.g.~to keep average of all fields near one.]

%-----------------------------------------------------------------------
\subsection{Problem parameters} \label{ss:parms-problem}
%-----------------------------------------------------------------------

Problem parameters include initial conditions and boundary conditions.

Different types of boundary conditions are supported, including
periodic, in- and out-flow, specified, and dynamic.  Different
boundary conditions can be specified for the entire domain, on
separate faces, on subregions of faces, or on specific zones.
Different boundary conditions can be specified for different fields.

\subsubsection{Use Cases}

\begin{verbatim}
   problem {
      boundary {
         x:lower = reflecting
         x:upper = { type = reflecting }
         y       = { type = periodic }
         z       = { type = inflow,  value = 1.0 }
         z       = { outflow, 1.0 }
      }
   }
\end{verbatim}

\begin{verbatim}
   XM = boundary { x = domain:lower[0] }
   XP = boundary { x = domain:upper[0] }
   YM = boundary { y = domain:lower[1] }
   YP = boundary { y = domain:upper[1] }
   ZM = boundary { z = domain:lower[2] }
   ZP = boundary { z = domain:upper[2] }
   field {
      name = "density"
      value(XM) = 0
      value(XP) = 0
      value(YM) = value (YP)
      value(ZM) = +t
      value(ZM) = -t
   }
\end{verbatim}

%-----------------------------------------------------------------------
\subsection{Domain parameters} \label{ss:parms-domain}
%-----------------------------------------------------------------------

The \code{domain} function is used to specify properties of the
domain.  Domains are boxes aligned with the axes of the computational
coordinate system, and are uniquely determined by the spacial
dimension, and the lowest and highest points in the domain.

\begin{verbatim}
   domain { 
      dimension = 3
      lower     = <-3e9,-3e9,-3e9>
      upper     = <3e9,3e9,3e9>
   }
\end{verbatim}

\begin{verbatim}
   domain { 3, <-3e9,-3e9,-3e9>, <3e9,3e9,3e9> }  // Implicit ordering
\end{verbatim}

\begin{verbatim}
   domain { 
      dimension = 3
      upper     = 3e9        // expand scalar to vector
      lower     = -upper     // parameters can be accessed as values
   }
\end{verbatim}

Include errors.

%-----------------------------------------------------------------------
\subsubsection{Parameters}
%-----------------------------------------------------------------------

 \todo\ \textit{Decide: allow defaults?  allow optional parameters?  Special
 \code{OPT\_} prefix for optional parameters?  Write out explicit copy
 of input file?}

The \code{domain} function has three parameters

\begin{tabular}{lll} \\
Name & Type & Restrictions \\ \hline
\code{dimension} & Scalar & $1-3$ \\
\code{lower}     & Vector & length = \code{dimension} \\
\code{upper}     & Vector & length = \code{dimension}, \code{upper} $>$ \code{lower}
\end{tabular}

Lower and upper points are given in units given by \code{units},
described in \S\ref{ss:params-units}.

%-----------------------------------------------------------------------
\subsubsection{Restrictions}
%-----------------------------------------------------------------------

\begin{enumerate}
\item The dimension must be 1, 2, or 3.
\item The number of coordinates in both lower and upper points must equal the dimension.
\item Each coordinate of the lower point must be strictly greater than the corresponding coordinate of the upper point.
\end{enumerate}

%-----------------------------------------------------------------------
\subsection{Region parameters} \label{s:region}
%-----------------------------------------------------------------------


Specify partitions of the domain into regions.  Each region contains
different materials with different properties.  Example partitions may
be half-planes, spheres, boxes, or specified using a file containing a
zone bit mask.  Default is region 0, first region is region 1, etc.
Use solid modeling representations?

\subsubsection{Use Cases}

\begin{verbatim}
   region {
      x + y + z < 0.5
   }
\end{verbatim}

\begin{verbatim}
   BOX = region {
      (x > 0) &&
      (x < 1) &&
      (y > 0) && (y < 1) &&
      (z > 0) && (z < 1)
   }
\end{verbatim}

\begin{verbatim}
   region {
      union {
         region {BOX, translate = 0.0*x, scale = 0.1}
         region {BOX, translate = 0.1*x, scale = <0.1,0.1,0.1>}
         region {BOX, translate = 0.2*x, scale = 0.1}
      }
   }
\end{verbatim}

\begin{verbatim}
   region {
      x = domain:lower[0]
   }
\end{verbatim}

\begin{verbatim}
   region {
      x*x + y*y + z*z < 1
   }
\end{verbatim}

\begin{verbatim}
   region {
      bitmask = "filename.hdf5"
   }
\end{verbatim}


%-----------------------------------------------------------------------
\subsection{Boundary parameters}
%-----------------------------------------------------------------------

%-----------------------------------------------------------------------
\subsection{Field parameters}
%-----------------------------------------------------------------------

Scalar and vector fields for each material, such as
 density, energy, velocity, etc.  [Merge with Materials?]  Specify
 values, or input from files.

\begin{verbatim}
   field {
      name     = "density"
      name     = "rho"
      type     = scalar
      location = center
   }
   field {
      name     = "velocity"
      name     = "u"
      type     = vector
      location = center
   }
   field {
      name     = "temperature"
      name     = "T"
      type     = scalar
      location = center
   }
   field {
      name     = "B"
      type     = computed
      location = face
   }
\end{verbatim}

%-----------------------------------------------------------------------
\subsection{Matter parameters} \label{s:params-matter}
%-----------------------------------------------------------------------

 Matter defines properties of matter, such as the matter type (baryonic
 or dark matter), and gas constants.

\subsubsection{Use Cases}

\begin{verbatim}
   Matter {
      type = dark
   }
\end{verbatim}

\begin{verbatim}
   Matter {
      type  = gas_ideal
      gamma = 1.4
      region { (x < 0) || (x > 1) }
   }
\end{verbatim}

\subsubsection{Parameters}

%-----------------------------------------------------------------------
\subsection{Method parameters}
%-----------------------------------------------------------------------

 Specify the algorithms and algorithm parameters
 to use for each physics component.  Each physics component has a
 default; some components may have only one available
 (e.g.~cosmological expansion).  Algorithms is in the solution domain.

\begin{itemize}
\item PPM hydro (dual-energy, etc.)
\item gravity solver (FAC, smoother, levels, etc.)
\end{itemize}

\subsubsection{Use Cases}
\subsubsection{Parameters}

%-----------------------------------------------------------------------
\subsection{Grid parameters} \label{ss:params-grid}
%-----------------------------------------------------------------------


Specify data structures and data structure parameters for distributed
AMR hierarchies, such as number of mesh levels, grid patch properties,
rebuild algorithm, dynamic load balancing, refinement criteria, etc.

\begin{itemize}
\item grid
\begin{item}
\item min\_levels 
\item max\_levels 
\end{item}
\item level
\item grid
\begin{itemize}
\item min\_size
\item max\_size
\item max\_aspect
\item quantum
\end{itemize}
\end{itemize}


\subsubsection{Use Cases}
\subsubsection{Parameters}


%-----------------------------------------------------------------------
\subsection{Particle parameters} \label{ss:params-particles}
%-----------------------------------------------------------------------

Specifies data structures and data structure parameters related
to distributed particles.  

\begin{itemize}
\item min\_group\_size
\item max\_group\_size
\end{itemize}


\subsubsection{Use Cases}
\subsubsection{Parameters}


%-----------------------------------------------------------------------
\subsection{Parallel parameters} \label{ss:params-parallel}
%-----------------------------------------------------------------------

Specify parallelism type and parameters.  For example, non-blocking
MPI, MPI-2, hybrid MPI/UPC, performance-related parameters such as
buffer size, etc.

Specifiy parallelism and parameters

\begin{itemize}
\item MPI (send/recv and type, one-sided and type, what level)
\item OpenMP (num threads, what level)
\item UPC (num threads, what level)
\item pthreads (num threads, what level)
\item cooperative parallelism
\item levels for each if multiple
\end{itemize}


\subsubsection{Use Cases}
\subsubsection{Parameters}


%-----------------------------------------------------------------------
\subsection{Output parameters} \label{ss:params-output}
%-----------------------------------------------------------------------

Output parameters.


\subsubsection{Use Cases}


\begin{verbatim}
output { 
   name = "data"
   format = hdf5
   type   = [data, input]
   fields = ["density", "velocity", "temperature"]
   file = ["data-%6s" cycle_number]
   cycle = 0:10:90
   cycle = 100:100:900
   cycle = 1000
}
\end{verbatim}

\begin{verbatim}
output { 
   name      = "restart"
   format    = hdf5
   type      = [data, input]
   fields    = all
   file      = ["restart-%6s" cycle_number]
   time_cpu  = 0.5 # CPU hours
   overwrite = true
   copies    = 2
}
\end{verbatim}

\begin{verbatim}
output { 
   name      = "movie"
   file      = ["movie-%6s" cycle_number]
   time      = :10:
   extract   = x == 12
}
\end{verbatim}


\subsubsection{Parameters}


%-----------------------------------------------------------------------
\subsection{Monitor parameters} \label{ss:params-monitor}
%-----------------------------------------------------------------------

High-level monitoring of the run at a summary level, such as current
timestep, problem time, wall time, cpu time, etc.


\subsubsection{Use Cases}

\begin{verbatim}
   monitor {
     type   = html
     amount = verbose
   }
\end{verbatim}

\subsubsection{Parameters}


%-----------------------------------------------------------------------
\subsection{Performance parameters} \label{ss:parames-performance}
%-----------------------------------------------------------------------

Performance monitoring and optimization(?) parameters

%-----------------------------------------------------------------------
\subsubsection{Use Cases}
%-----------------------------------------------------------------------
%-----------------------------------------------------------------------
\subsubsection{Parameters}
%-----------------------------------------------------------------------

%-----------------------------------------------------------------------
\subsection{Error parameters} \label{ss:params-error}
%-----------------------------------------------------------------------

Fault tolerance and adaptivity parameters.

\begin{itemize}
\item Power outage
\item Network connectivity dropouts
\item Corrupt RAM
\item Crashed hard drive
\item Moth in relay switch
\item Crashed node
\end{itemize}

\begin{itemize}
\item fault tolerance methodology
\item adaptivity
\end{itemize}

\subsubsection{Use Cases}
\subsubsection{Parameters}
