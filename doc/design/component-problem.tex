%=======================================================================
\section{Problem Component} \label{s:component-problem}
%=======================================================================

Problem parameters specify the setup of the physical problem,
including initial conditions of relevant data fields and boundary
conditions.

  dimensionality
  domain extents
  initial conditions (materials, regions, input)
 boundary conditions (periodic, in-/out-flow, specified, dynamic)

Problem parameters include initial conditions and boundary conditions.

Different types of boundary conditions are supported, including
periodic, in- and out-flow, specified, and dynamic.  Different
boundary conditions can be specified for the entire domain, on
separate faces, on subregions of faces, or on specific zones.
Different boundary conditions can be specified for different fields.
%-----------------------------------------------------------------------
\subsection{Use Cases}
%-----------------------------------------------------------------------

\begin{verbatim}
   problem {
      boundary {
         x:lower = reflecting
         x:upper = { type = reflecting }
         y       = { type = periodic }
         z       = { type = inflow,  value = 1.0 }
         z       = { outflow, 1.0 }
      }
   }
\end{verbatim}

\begin{verbatim}
   XM = boundary { x = domain:lower[0] }
   XP = boundary { x = domain:upper[0] }
   YM = boundary { y = domain:lower[1] }
   YP = boundary { y = domain:upper[1] }
   ZM = boundary { z = domain:lower[2] }
   ZP = boundary { z = domain:upper[2] }
   field {
      name = "density"
      value(XM) = 0
      value(XP) = 0
      value(YM) = value (YP)
      value(ZM) = +t
      value(ZM) = -t
   }
\end{verbatim}

%-----------------------------------------------------------------------
\subsection{Parameters}
%-----------------------------------------------------------------------
