%=======================================================================
\section{Field Component} \label{s:component-field}
%=======================================================================

Scalar and vector fields for each material, such as
 density, energy, velocity, etc.  [Merge with Materials?]  Specify
 values, or input from files.

%-----------------------------------------------------------------------
\subsection{Use Cases}
%-----------------------------------------------------------------------
\begin{verbatim}
   field {
      name     = "density"
      name     = "rho"
      type     = scalar
      location = center
   }
   field {
      name     = "velocity"
      name     = "u"
      type     = vector
      location = center
   }
   field {
      name     = "temperature"
      name     = "T"
      type     = scalar
      location = center
   }
   field {
      name     = "B"
      type     = computed
      location = face
   }
\end{verbatim}

%-----------------------------------------------------------------------
\subsection{Parameters}
%-----------------------------------------------------------------------

%=======================================================================
\subsection{\code{Units} Component} \label{ss:component-units}
%=======================================================================

 Specify units and optional scalings for individual
 fields.  
 \note{Dynamic scaling, e.g.~to keep average of all fields near one.}

%-----------------------------------------------------------------------
\subsubsection{Use Cases}
%-----------------------------------------------------------------------
%-----------------------------------------------------------------------
\subsubsection{Parameters}
%-----------------------------------------------------------------------

%=======================================================================
\subsection{\code{Matter} Component} \label{ss:component-matter}
%=======================================================================

 Specify units and optional scalings for individual
 fields.  
 \note{Dynamic scaling, e.g.~to keep average of all fields near one.}

%-----------------------------------------------------------------------
\subsubsection{Use Cases}
%-----------------------------------------------------------------------

%-----------------------------------------------------------------------
\subsubsection{Parameters}
%-----------------------------------------------------------------------

\begin{verbatim}
   Matter {
      type = dark
   }
\end{verbatim}

\begin{verbatim}
   Matter {
      type  = gas_ideal
      gamma = 1.4
   }
\end{verbatim}


\subsection{ Field  Component}
Scalar fields on an Array or Mesh hierarchy

A Field represents a discretized scalar field. A Field is typically
discretized on an AMR hierarchy, but can be accessed at the Array or a
Block levels as well. In addition to the array of elements, a Field
includes an identifier for the field, the location of the field values
with respect to computational cells (cell centered, face-centered,
etc.), the index into the Block or Array of the field, optional
minimum or maximum allowed values, units, and user-defined tags.

Attributes:

\begin{tabular}{ll}
    \code{name}        & String defining the field's name, e.g. "density", "velocity-x", etc. \\
    \code{id} 	& Integer identifying the Field \\
    \code{array} &	Array containing Field values for the containing Patch \\
    \code{index} &	Index of the specific array in the 4D Array \\
    \code{position} &	cell position, defined as (0,0,0) <= (px,py,pz) <= (1,1,1). (.5,.5,.5)=cell centered \\
    \code{min} &	minimum allowed value for the Field \\
    \code{max} &	maximum allowed value for the Field \\
    \code{min\_action} &	what should be done if Field goes below min (e.g. 1: set to min, 2: warning, 4: error, g: retry with smaller timestep, etc.) \\
    \code{max\_action} &	what should be done if Field goes above max
\end{tabular}

A set of Fields defined on a Block, along with groups of Particles
associated with the block, are the main input to (and output from)
Methods. Methods have access to the Field's values, size, cell
position, units, tags, limits, etc.

Actual Field objects are stored in the Mesh Patch object. Patch objects
also contain associated Particle objects, and a Box object defining
the Patch's extents and position.  List of fields


Field supporting classes

Field operations
Field hierarchy
Structure
Descriptions
Field class
Attributes
Functions
Usage

Issues:

\begin{itemize}
\item What operations are the responsibility of Field versus Array,
  Task, Patch, etc.
\item see below: Field stores only global properties of field, not
  actual data
\item How to store global attributes (name, index, min, max, position,
  etc.) versus non-global (values)
\item Single global Field object for each field
\item Fields don't store actual Arrays, Patches do--Field only stores
  Array index.
\end{itemize}
