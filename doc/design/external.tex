
%======================================================================@
\chapter{External Libraries} \label{c:external}
%======================================================================@

This chapter discusses related external software libraries and
frameworks.  We include brief descriptions and scope of each library,
useful features, and list reasons why we decided to use or not use the
library with \cello.

% @REF@ External libraries referenced in compiling.tex

\begin{tabular}{|ll|} \hline
\textbf{Package} & \textbf{description} \\ \hline
\usemaybe\ \code{AMPI}    & \textit{Adaptive MPI with dynamic load balancing and multitreading} \\
\useno\ \code{arprec}  & \textit{Arbitrary precision package} \\ 
\usemaybe\ \code{Blitz++}   & \textit{Numerical library using expression templates} \\
\useno\ \code{BoxLib} & \textit{Library for block structured finite difference methods} \\
\usemaybe\ \code{CHARM++}  & \textit{Machine independent parallel programming system} \\
\useno\ \code{DAGH}    & \textit{Distributed Adaptive Grid hierarchy} \\
\useno\ \code{Global Arrays} & \textit{Supports large arrays distributed across nodes} \\
\usemaybe\ \code{hypre}   & \textit{Scalable linear solvers}  \\& \url{http://www.llnl.gov/CASC/hypre/} \\
\useno\ \code{lcaperf} & \textit{Parallel performance toolkit}  \\& \url{http://mngrid.ucsd.edu/packages/lcaperf/} \\
\useyes\ \code{lcatest} & \textit{Parallel testing infrastructure}  \\& \url{http://mngrid.ucsd.edu/packages/lcatest/} \\
\useyes\ \code{libmatheval} & \textit{Parse and evaluate symbolic expressions}  \\& \url{http://www.gnu.org/software/libmatheval/} \\
\usemaybe\ \code{P3DFFT}  & \textit{Parallel 3D FFT} \\& \url{http://www.sdsc.edu/us/resources/p3dfft_download.html} \\
\useyes\ \code{PAPI}    & \textit{Performance API}  \\& \url{http://icl.cs.utk.edu/papi/} \\
\useno\ \code{Paramesh} & \textit{Parallel adaptive mesh refinement} \\
\useno\ \code{PETSc}  & \textit{Object-oriented toolkit for constructing parallel scientific applications} \\
\useno\ \code{POOMA}   & \textit{Parallel object-oriented methods and applications} \\
\useyes\ \code{S2PLOT} & \textit{3D Plotting Library} \\
\useno\ \code{SAMRAI}  & \textit{Structured adaptive mesh refinement application infrastructure} \\
\useyes\ \code{SPRNG}   & \textit{Scalable parallel random number generator} \\& \url{http://sprng.cs.fsu.edu/} \\
\usemaybe\ \code{TAU}      & \textit{Portable profiling and tracing toolkit} \\
\usemaybe\ \code{VisIt}   & \textit{Parallel visualization tool} \\& \url{https://wci.llnl.gov/codes/visit/home.html} \\
\usemaybe\ \code{VPython} & \textit{3D Programming for Ordinary Mortals} \\
\hline
\end{tabular}

\section{\usemaybe\ \code{AMPI}}    
\section{\useno\ \code{arprec}}  
\section{\usemaybe\ \code{Blitz++}}   

\code{Blitz++} is a ``C++ class library for scientific computing which
provides performance on par with Fortran 77/90''.  It provides fast
operations for dense vectors and multidimensional arrays.  Support
is included for array interleaving, array blocking, and reference counting.
Since portability may be an issue and since it's a work under development, 
use of \code{Blitz++} will be optional.

\section{\useno\ \code{BoxLib}} 
\section{\usemaybe\ \code{CHARM++}}  
\section{\useno\ \code{DAGH}}    
\section{\useno\ \code{Global Arrays}} 
\section{\usemaybe\ \code{hypre}}   
\section{\useno\ \code{lcaperf}} 
\section{\useyes\ \code{lcatest}} 
\section{\usemaybe\ \code{P3DFFT}}  
\section{\useyes\ \code{PAPI}}    
\section{\useno\ \code{Paramesh}} 
\section{\useno\ \code{PETSc}}  
\section{\useno\ \code{POOMA}}   
\section{\useyes\ \code{S2PLOT}} 
\section{\useno\ \code{SAMRAI}}  
\section{\useyes\ \code{SPRNG}}   
\section{\usemaybe\ \code{TAU}}      
\section{\usemaybe\ \code{VisIt}}   
\section{\usemaybe\ \code{VPython}} 

% \hline

