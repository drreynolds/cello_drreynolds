
%======================================================================@
\chapter{External Libraries} \label{c:external}
%======================================================================@

This chapter discusses related external software libraries and frameworks.
We include brief descriptions and scope of each library,
useful features, and list reasons why we decided to use or
not use the library with \cello.

\begin{tabular}{|ll|} \hline
\textbf{Package} & \textbf{description} \\ \hline
\code{lcaperf} & \textit{Parallel performance toolkit}  \\
& \url{http://mngrid.ucsd.edu/packages/lcaperf/} \\
\code{lcatest} & \textit{Parallel testing infrastructure}  \\
& \url{http://mngrid.ucsd.edu/packages/lcatest/} \\
\code{PAPI}    & \textit{Performance API}  \\
& \url{http://icl.cs.utk.edu/papi/} \\
\code{hypre}   & \textit{Scalable linear solvers}  \\
& \url{http://www.llnl.gov/CASC/hypre/} \\
\code{P3DFFT}  & \textit{Parallel 3D FFT} \\
& \url{http://www.sdsc.edu/us/resources/p3dfft_download.html} \\
\code{SPRNG}   & \textit{Scalable parallel random number generator} \\
& \url{http://sprng.cs.fsu.edu/} \\
\code{VisIt}   & \textit{Parallel visualization tool} \\
& \url{https://wci.llnl.gov/codes/visit/home.html} \\
\code{arprec}  & \textit{Arbitrary precision package} \\ 
\code{POOMA}   & \textit{Parallel object-oriented methods and applications} \\
\code{SAMRAI}  & \textit{Structured adaptive mesh refinement application infrastructure} \\
\code{PETSc}  & \textit{Object-oriented toolkit for constructing parallel scientific applications} \\
\code{Paramesh} & \textit{Parallel adaptive mesh refinement} \\
\code{AMPI}    & \textit{Adaptive MPI with dynamic load balancing and multitreading} \\
\code{CHARM++}  & \textit{Machine independent parallel programming system} \\
\code{DAGH}    & \textit{Distributed Adaptive Grid hierarchy} \\
\code{TAU}      & \textit{Portable profiling and tracing toolkit} \\
\code{Blitz++}   & \textit{Numerical library using expression templates} \\
\code{BoxLib} & \textit{Library for block structured finite difference methods} \\
\code{Global Arrays} & \textit{Supports large arrays distributed across nodes} \\
\code{VPython} & \textit{3D Programming for Ordinary Mortals} \\
\hline
\end{tabular}

