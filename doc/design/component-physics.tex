
%=======================================================================
\section{Physics Component} \label{s:component-physics}
%=======================================================================

   Defines what physics to simulate in the computational universe.
   Used to define which physics processes are enabled, such as
   self-gravity, hydrodynamics, cosmological expansion, etc.  Also
   defines any parameters associated with the physics of the problem
   being solved, such as cosmological parameters and the gravitational
   constant.


 matter
 hydrodynamics
  cosmological expansion
 self-gravity

Hydrodynamics


Specify physics modules and physics parameters, including
hydrodynamics, self-gravity, gravitational constant, imposed gravity,
chemistry, cosmological expansion, star formation, etc.  Physics is in
the problem domain.

Specify physics components

\begin{itemize}
\item hydrodynamics
\item  cosmological expansion
\item self-gravity
\end{itemize}

%-----------------------------------------------------------------------
\subsection{Use Cases}
%-----------------------------------------------------------------------
%-----------------------------------------------------------------------
\subsection{Parameters}
%-----------------------------------------------------------------------

%=======================================================================
\subsection{Matter Component} \label{ss:component-matter}
%=======================================================================

 Matter defines properties of matter, such as the matter type (baryonic
 or dark matter), and gas constants.

%-----------------------------------------------------------------------
\subsection{Use Cases}
%-----------------------------------------------------------------------

\begin{verbatim}
   Matter {
      type = dark
   }
\end{verbatim}

\begin{verbatim}
   Matter {
      type  = gas_ideal
      gamma = 1.4
      region { (x < 0) || (x > 1) }
   }
\end{verbatim}
