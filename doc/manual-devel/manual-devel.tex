%0       1         2         3         4         5         6         7         8
%2345678901234567890123456789012345678901234567890123456789012345678901234567890

%=======================================================================
\documentclass{article}
%=======================================================================


% INCLUDE DEVELOPMENT TEXT

\newcommand{\devel}[1]{\textbf{#1}}

% EXCLUDE DEVELOPMENT TEXT

% \newcommand{\devel}[1]{}


%=======================================================================
% Document layout
%=======================================================================

\setlength{\topmargin}{0.0in}
\setlength{\oddsidemargin}{0.0in}
\setlength{\evensidemargin}{0.0in}
\setlength{\textwidth}{6.0in}
\setlength{\textheight}{9.0in}

%=======================================================================
% Packages
%=======================================================================

\usepackage{wasysym}
\usepackage{epsfig}
\usepackage{url}

%=======================================================================
% Commands
%=======================================================================

\newcommand{\cello}{\textsf{Cello}}
\newcommand{\enzo}{\textsf{Enzo}}
\newcommand{\lcaperf}{\textsf{lcaperf}}
\newcommand{\lcatest}{\textsf{lcatest}}

\newcommand{\code}[1]{\textsf{#1}}

\newcommand{\note}[1]{\devel{\eighthnote\ \textit{#1} \\}}
\newcommand{\pargraph}[1]{\devel{\P\ \textbf{#1} \\}}

\newcommand{\todo}{\devel{$\circ$}}
\newcommand{\done}{\devel{$\bullet$}}
\newcommand{\halfdone}{\devel{\textcolor{gray}{$\bullet$}}}

\newcommand{\PROJECT}{\cello}

\newcommand{\TITLE}[3]{
\title{ {\huge \PROJECT\ #1}  \\ \vspace{0.1in}
     {\small Document Version: \textbf{#3}} \vspace{-0.1in}
    }
\author{      #2 \\
        Laboratory for Computational Astrophysics\\
        University of California, San Diego}
\maketitle}

%=======================================================================


%=======================================================================

\begin{document}

%=======================================================================
\TITLE{\cello\ Developer Manual}{James Bordner}{$Rev$}
%=======================================================================

\tableofcontents

%=======================================================================
\section{Introduction} \label{s:intro}
%=======================================================================

The purpose of this document is to specify guidelines and procedures
for software development in the \cello\ project.  Guidelines and
procedures in this document should not be considered to be fixed:
suggestions for modifications and improvements are always welcome.
But they should be followed.

Software development is not easy, and software development as a group
is even harder.  But, fortunately, there are a few general guidelines
that, if followed, can greatly improve the quality of code and reduce
the time spent writing, debugging, and maintaining code.

%==================================================================
\section{General guidelines for software devopment}
%==================================================================

First, some software development philosophy.  There are two competing
forces in software: functionality and complexity.  Implementing new
functionality in a code increases its complexity.  That's
inevitable---you can't get something for nothing.  However, this
inevitable complexity can and should be controlled.  Nothing kills a
software project deader than out-of-control complexity.

There are two ways to control complexity: write new code to be as
simple as possible, and rewrite existing code to be as simple as
possible (this is called \textit{refactoring}).  And for a large
software project to be successful in the long-term, both of these are
required.

One overriding guideline to writing good code that is very helpful is
\textit{avoid repeated code}.  This is important and bears repeating:
\textit{avoid repeated code}.  If you write or find two pieces of
code that are similar, spend some time to think about how the similar
bits can be factored out.  

A direct corrollary to this is \textit{don't write a new function by
copying an existing function and changing just a few lines}.  Copying
code means copying bugs, it means any new functionality that involves
modifying the code will have to be done multiple times, and it makes
it easy to forget to modify all copies when bug fixes or modifications
are necessary.  Copying code may make it easier to write new code in
the short-term, but in the long-term the complexity of the program as
a whole will grow unbounded.

Working together on a software application involves a lot more than
just writing code.  Successful software development also requires bug
tracking, continuous refactoring, and rigorous testing.

Developing quality software requires disciplined and rigorous
thinking.  Don't code when sleepy or otherwise not in top form.  One
moment's slip-up can lead to hours or days of wasted effort in
tracking down a bug.  

Spending more time up-front in thinking about (and documenting!) 
requirements and design will reduce the time spent in coding and
debugging by an even greater amount, so that the overall time spent
will generally be less.  Spending more time at the beginning planning
before writing a single line of code can be a difficult habit to get
into, but the payoff in both time and reduced frustration can be
great.

The correct way to develop software is \textit{not} to sit down at a
terminal and start coding.  Wrong, wrong, wrong.  First you figure out
precisely \textit{what} you want the new code to do, then you figure
out \textit{how} the new code will do it.  Then, and only then, do you
actually start coding.

\begin{enumerate}
\item \textbf{Requirements}: What should the new code do?
\item \textbf{Design}: How should the new code be written to satisfy these requirements?
\item \textbf{Implementation}: Write the code according to the design.
\item \textbf{Test}: Verify that the implementation satisfies the requirements.
\end{enumerate}

These steps should not necessarily be done just once.  When working on
the design, you may realize that some of the requirements should be
changed somewhat; or you may realize during implementation that
there's a better approach, and that improving the design may be
helpful.  Software development works best as an iterative process.

However, these steps should always be done in order.  Whenever you
visit a step, always propogate the changes to all subsequent steps.
For example, if you decide you need to add a new requirement, you
should also update the design accordingly, then update the
implementation, and then retest.

%==================================================================
\section{Coding standards}
%==================================================================


   \pargraph{ ifdef around new code }
   \pargraph{ If you're modifying existing code, refactor it }

      \note{This includes comments and documentation}
      \note{Always test refactored code as thoroughly as new code}


\pargraph{ computing indexes }
   \pargraph{ access arrays using array notation, not pointer notation}
\pargraph{ don't use \code{switch()} if an \code{if-else} construct will do}

   \pargraph{ avoid nested conditionals } 
   \pargraph{ factor out redundant code that occurs in both ``if'' and ``else'' clauses }
   
   \pargraph{ Always use braces for multi-line if and for statements }
   \pargraph{ Don't test for null before delete; do test for null before dereferencing (e.g. DeleteFluxes.C) }
   \pargraph{Write paired statements at the same time: e.g. \code{new()} and \code{delete()}; \code{open()} and \code{close()}, etc. }
   \pargraph{Use \code{memory:new()} and \code{memory:delete()}, not \code{new()} or \code{delete()}.}
   \pargraph{Prefer simple code to complex code}
   \pargraph{Comment complicated code }
   \pargraph{Avoid ``magic constants''}
   \pargraph{Use src/TEMPLATE/* files}
    \pargraph{Commit changes to code repository frequently}
    \pargraph{Avoid committing lots of independent code changes all at once}
    \pargraph{Be descriptive and complete in repository commit log messages}
    \pargraph{Versioning is based on iterations through the requirements/design/implement/test/debug cycle}
    \pargraph{Name boolean variables \code{isBlah}}
    \pargraph{Avoid communicating between functions using side-effects, such as global variables}
    \pargraph{Simplify complex logic ( \code{NewOrOld == NEW\_AND\_OLD || NewOrOld == OLD\_ONLY}). }
    \pargraph{remove obsolete code }
    \pargraph{Use macros only when necessary; prefer inline functions}
    \pargraph{Functions should not be too long}


    \pargraph{ Check other people's code. }
    \pargraph{ Encourage others to check your code. }
    \pargraph{ Write functions for small tasks }
    \pargraph{ Write a collection of functions for larger tasks, and group them in a class }
    \pargraph{ Figure out exactly what you're going to code before you code it }
    \pargraph{ Feel free to refactor other people's code.  Always test thoroughly. }
    \pargraph{ Encourage others to refactor your code.}
    \pargraph{ Eschew global variables. }

    \pargraph{ Use descriptive but short function and method names }
    \pargraph{ Name classes beginning with an upper-case letter; e.g.~\code{Grid}, \code{Patch}, \code{File}}

     \pargraph{Avoid non-obvious abbreviations, e.g. OOC (out-of-core), }

    \pargraph{Use whitespace freely for readability}

    \pargraph{Indent loops and conditionals consistently}

    \pargraph{Limit line lengths to 80 characters }

    \pargraph{Use only one code statement per line }
    \pargraph{Split up complicated statements into multiple simpler statements if appropriate}

    
    \pargraph{All files should have a full header comment }

    \pargraph{avoid too many levels of inheritance in class heirarchies }
    \pargraph{Prefer class composition to class inheritence}

    \pargraph{Use multiple inheritance sparingly}
    \pargraph{Avoid operator overloading for large data structures}


    \pargraph{For Fortran, use Fortran 77}
    \pargraph{Don't change names of files, functions, or executables}
    \pargraph{Avoid duplicate code}
    \pargraph{If multiple functions/filenames are required that do similar things, keep the name prefix similar and change the suffix.}

\begin{verbatim}
http://www.softwareqatest.com/qatfaq1.html
http://wilma.vub.ac.be/~se4/papers/codingconventions.html
\end{verbatim}

%==================================================================
\section{\cello\ Software Development Organization}
%==================================================================


\pargraph{Other documentation}

  \note{project management plan: high level organization of \cello\ project}

\pargraph{Subversion}

\pargraph{Bug list}

   \note{BUGS file}
   \note{currently in subversion with code, but may move}
   \note{maintained by jbordner}

\pargraph{Make system}

\pargraph{Testing}

   
%==================================================================
\section{Procedures for common coding tasks}
%==================================================================

\note{Include how to document, who to tell, who to consult with}

Software development tasks
\pargraph{How do I port \cello\ to a new platform?}

Coding tasks
\pargraph{How do I add a new problem type to \cello?}
\pargraph{How do I add a new physics step in \code{EvolveLevel()}?}
\pargraph{How do I make global changes?}

\pargraph{How do I add a new \code{\#ifdef}}
\pargraph{Adding a new global ifdef (config changes)}
\pargraph{Adding a new configuration setting}
\pargraph{Adding a new particle type}
\pargraph{adding new baryon fields}
\pargraph{Adding a new source code file}
\pargraph{adding a new output file}
\pargraph{adding a new input file}
\pargraph{adding a new parameter}
\pargraph{reporting a bug}
\pargraph{adding new data members to the grid class}



\subsection{Commiting new code to the repository}

The following checklist should be used when updating the \cello\ CVS repository:

\begin{enumerate}
   \item ``\verb+cvs update+'' to obtain the latest changes from the repository.
      CVS will not let you check in changes otherwise.  If there are
      conflicts (``\verb+cvs update |& grep '^C'+''), you will need to fix
      them.  Consult with the author of the code conflicting with
      yours if needed.

   \item Test your updates thoroughly.  Having other people double-check
      your changes is a very good idea.

   \item Remember to \verb+cvs add+ any new files (or \verb+cvs delete+
    old files) before updating the repository.  Otherwise your
   copy may work fine, but other peoples' copies won't.  The command
   \verb+cvs update | grep '?'+ will tell you what files are in your
   local copy but aren't in the repository.

   \item Decide on how to increment the version tag
   \texttt{v}$x$\_$y$\_$z$, which represents version $x$.$y$.$z$.  (The
   \texttt{v}$x$\_$y$\_$z$ format is used for tags because CVS does not
   allow ``\texttt{.}'' in tags, and because CVS tags must begin with a letter).

\begin{enumerate}
      \item increment $x$ if it's a seriously major update (large-scale,
          widespread modifications affecting the whole code, e.g. 
          implementing 64-bit integers, or adding MHD).
      \item increment $y$ if it's a smaller update (mostly localized but 
          non-trivial mods, e.g. adding a new movie format or problem type).
      \item increment $z$ for minor bug fixes or minor local modifications.
\end{enumerate}

   \item Edit the \texttt{VERSION} file to reflect the new version tag.

   \item Edit the \texttt{amr\_mpi/src/version.def} file to reflect the new
      version.

   \item Update the \texttt{CHANGES} file to document all modifications.
   It's a good idea to use \texttt{cvs diff} (or \verb+cvs diff --brief+) to remind you of what changes you made.  If changes are
   major, feel free to add a file to the \texttt{doc/} subdirectory
   containing a more detailed description of the changes.

   \item Double-check that you've done all of the above steps.  Including
      this one.

   \item \texttt{cvs commit}.  You will be prompted for a description of the
      changes--feel free to cut-and-paste what you wrote in the
      \texttt{CHANGES} file for that.

   \item Tag the CVS repository with the new version using \texttt{cvs tag} \textit{version}, where ``\textit{version}'' is identical to the contents of the \texttt{VERSION} file.
\end{enumerate}

\pargraph{updating code from CVS}
\pargraph{resolving conflicts with CVS}
\pargraph{refactoring}
\pargraph{Adding an external library}
\pargraph{Global changes}
\pargraph{Changing existing code (name of a function, name of a file, etc.)}
\pargraph{File names}
\pargraph{Function names}
\pargraph{Class/struct names}
\pargraph{variable names}
\pargraph{\#ifdef names}

\pargraph{Declare all functions, constants, and types that are intended to be
used outside the file in a header file.}

\pargraph{Declare static functions, file-local constants, and types at the top
of the file.}


\pargraph{Comment all functions}

  Unless the name of the function makes it obvious what the function
  does (and this should be your goal when choosing a name), you should
  add a comment describing the use and purpose of the function.

\pargraph{Choose identifier names with care.}
  Proper identifiers naming is an important way to help readers
  understand your code.  In particular, except for indices (where
  generic i, j, etc. may be appropriate) identifiers should rarely
  have one-letter names.  Very long names can also be a nuisance as
  they can make programs hard to format.  Name selection is
  non-trivial. By K\&R convention, variable names start with a
  lowercase letter, type names with an uppercase letter.

\pargraph{Avoid embedded constants.}
%
Except for trivial constants (say 0, 1), it is much better to use
\#define and set an identifier (all capital letters) equal to the
constant than to use the constant in code directly (a practice known
as “wired-in numbers”).  This allows you to aid the reader, as the
name you choose for the constant should help him/her figure out what
the constant represents. Also (in many cases) when you need to change
the constant you can do so by changing its value in one place, rather
than in many places in the code.  You can use const instead of
\#define to declare constants (though this is more the custom in C++
than in C).

\pargraph{Minimize use of global variables.}  
%
When you do use one, define it in a .c file, not in a header file,
since header files are meant to be included in multiple .c files and
each of them will try to define the variable.  Not good.  You can put
an extern declaration for it in a header if necessary.  Try to keep to
static file-global variables, the C equivalent of Java class
variables, plus local variables.

\pargraph{Avoid long functions (over about 60 lines).}  
%
Use helper functions.

\pargraph{Avoid ``cut-and-paste'' coding.}  
%

The ``cut-and-paste'' method of writing code leads to duplicated code,
which is a serious coding error.  

\textit{``Number one in the stink parade is duplicated code.''}  --Martin Fowler, ``Refactoring: Improving the Design of Existing Code

    \note{More code to test, maintain and document.}
    \note{Maintenance efforts require making the same changes at multiple locations.}
    \note{More bugs, since if the same code is repeated at several locations, changes and fixes at one location may not get propagated to all the other locations.}
    \note{Larger executable size.}


\pargraph{Avoid ``cut-and-paste'' coding.}  

Use functions to 'factor' out the repeated code.

\pargraph{Use consistent indentation.}  
%
Poorly indented code is difficult to understand.  Emacs knows how to
do this.  Avoid using tabs for indenting: use spaces.

\pargraph{No goto's.}

\pargraph{Comment all non-obvious code.}
%
  Or better, rewrite the code so that it becomes obvious.  If you run
  across someone else's complicated code and figure out what it does,
  add a comment to help the next person.

\pargraph{Use for and while appropriately.}
%
  Generally use for when you have an index that changes each
  iteration.  Use while when you don't.

\pargraph{For complicated conditionals, use switch instead of if-then-else's.}

\pargraph{For simple conditionals, use if-then-else instead of switch}

\pargraph{Use the C++ STL}.  Use the C++ Standard Template Library
containers (e.g. std::list, std::stack, std::vector) and types
(e.g. std::string) instead of rolling your own.


%==================================================================
\section{Bad Code Smells}
%==================================================================

(From  ``\textit{Refactoring: Improving the Design of Existing Code},'' Martin Fowler and Kent Beck.)

\pargraph{Duplicated Code}
\pargraph{Long Method}
\pargraph{Large Class}
\pargraph{Long Parameter List}
\pargraph{Divergent Change}
\pargraph{Shotgun Surgery}
\pargraph{Feature Envy}
\pargraph{Data Clumps}
\pargraph{Primitive Obsession}
\pargraph{Switch Statements}
\pargraph{Parallel Inheritance Hierarchies}
\pargraph{Lazy Class}
\pargraph{Speculative Generality}
\pargraph{Temporary Field}
\pargraph{Message Chains}
\pargraph{Middle Man}
\pargraph{Inappropriate Intimacy}
\pargraph{Alternative Classes with Different Interfaces}
\pargraph{Incomplete Library Class}
\pargraph{Data Class}
\pargraph{Refused Bequest}
\pargraph{Comments}


%==================================================================
\section{Header comment templates}
%==================================================================

\subsection{File header comment}

\pargraph{Each file should start with a comment which includes your name, the
date you wrote the file, the name of the file.}

Make it doxygen-readable.

\subsection{Function header comment}

C++
\begin{verbatim}
   FILENAME
   SUMMARY
   DESCRIPTION

   FUNCTIONS
   PARAMETERS

   PUBLIC FUNCTIONS

   AUTHOR
   DATE
   CHANGES (version/date/who/detail)
\end{verbatim}


%==================================================================
\section{File naming conventions}
%==================================================================

\subsection{Prefixes and Suffixes}

\begin{verbatim}
*.hpp
*.cpp
*.hdf4        HDF4      
*.hdf5        HDF5      
*.exe         binary executable
*.text        text file for human consumption
in.*.cello    Cello input parameter file
out.*.cello   Cello output files
\end{verbatim}


\end{document}

%==================================================================

