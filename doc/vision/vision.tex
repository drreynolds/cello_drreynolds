%0       1         2         3         4         5         6         7         8
%2345678901234567890123456789012345678901234567890123456789012345678901234567890
%=======================================================================
\documentclass[11pt]{article}
%=======================================================================


% INCLUDE DEVELOPMENT TEXT

\newcommand{\devel}[1]{\textbf{#1}}

% EXCLUDE DEVELOPMENT TEXT

% \newcommand{\devel}[1]{}


%=======================================================================
% Document layout
%=======================================================================

\setlength{\topmargin}{0.0in}
\setlength{\oddsidemargin}{0.0in}
\setlength{\evensidemargin}{0.0in}
\setlength{\textwidth}{6.0in}
\setlength{\textheight}{9.0in}

%=======================================================================
% Packages
%=======================================================================

\usepackage{wasysym}
\usepackage{epsfig}
\usepackage{url}

%=======================================================================
% Commands
%=======================================================================

\newcommand{\cello}{\textsf{Cello}}
\newcommand{\enzo}{\textsf{Enzo}}
\newcommand{\lcaperf}{\textsf{lcaperf}}
\newcommand{\lcatest}{\textsf{lcatest}}

\newcommand{\code}[1]{\textsf{#1}}

\newcommand{\note}[1]{\devel{\eighthnote\ \textit{#1} \\}}
\newcommand{\pargraph}[1]{\devel{\P\ \textbf{#1} \\}}

\newcommand{\todo}{\devel{$\circ$}}
\newcommand{\done}{\devel{$\bullet$}}
\newcommand{\halfdone}{\devel{\textcolor{gray}{$\bullet$}}}

\newcommand{\PROJECT}{\cello}

\newcommand{\TITLE}[3]{
\title{ {\huge \PROJECT\ #1}  \\ \vspace{0.1in}
     {\small Document Version: \textbf{#3}} \vspace{-0.1in}
    }
\author{      #2 \\
        Laboratory for Computational Astrophysics\\
        University of California, San Diego}
\maketitle}

%=======================================================================


%=======================================================================

\begin{document}

%=======================================================================
\TITLE{\cello\ Vision and Scope Document \\
      \Large{\enzo: The Next Generation}}{James Bordner}{$Rev$}
%=======================================================================

%=======================================================================
\section{Problem Statement}
%=======================================================================

%-----------------------------------------------------------------------
\subsection{Project background}
%-----------------------------------------------------------------------


    The \cello\ Project\footnote{While the origin of the name
    ``\enzo'' has been lost in mist of time, the name ``\cello'' was
    derived from the computational term ``cell'', which is a synonym
    of the computational term ``zone'', which in turn is an anagram of
    ``enzo''.  Analogously, the \cello\ code will be derived from
    \enzo's code by rearranging it, modifying it, and, uh, sticking an
    ``o'' on the end.}  is designed to be the successor to \enzo.
    \enzo\ was conceived in the early 1990's by Michael L.~Norman and
    Greg Bryan to be a multi-resolution astrophysics and cosmology
    application, implemented using structured (patch-based) adaptive
    mesh refinement (SAMR).  It incorporated a modified high-order
    Piecewise Parabolic Method (PPM) solver for hydrodynamics, and a
    Particle-Mesh (PM) method for dark matter dynamics.  So far in its
    14 year lifetime, \enzo\ simulations have led to important
    scientific contributions to the fields of astrophysics, cosmology,
    and turbulence.

    Greg Bryan began implementing \enzo\ in 1994.  He wrote it in C++
    with Fortran 77 computational kernels for speed, using primarily a
    procedural (structured) programming paradigm.  Initially he
    targeted shared memory platforms such as the SGI Origin, which was
    later modified to support distributed memory platforms via the
    Message Passing Interface (MPI).

    Beginning around 2000, other developers began to modify \enzo\ in
    earnest: David Collins added MHD physics, Dan Reynolds added
    RHD physics, Greg Bryan himself modified AMR algorithms to improve
    performance, and Robert Harkness continually made innumerable
    modifications to greatly improve performance and scaling, and
    almost single-handedly brought \enzo\ into the Terascale Era.

    The software development strategy used has been very informal.
    \enzo\ was developed without any supporting documents, such as
    requirement specifications, design description, or test document.
    It was also written without any explicit coding standard, though
    the original author closely followed implicit coding guidelines,
    such as including uniform file header comment blocks, and using
    descriptive variable, function, and file names.  The original
    author also implemented many application tests to help ensure the
    accuracy of the computed solutions.  

% Enzo is getting harder to develop
    The relaxed software development strategy used with \enzo,
    combined with the shift from single-person to multi-person
    development, have resulted in a rapid increase in \enzo's
    complexity.  This increased software complexity translates
    directly into increased difficulty in modifying and maintaining
    the code.  Furthermore, \enzo's ``modification-resistant''
    structured programming paradigm is a further hinderance to
    its continued development.
    
% pargraph: Enzo requires further development

% pargraph: Above two points motivate cello

    These two issues motivate the \cello\ Project.  While \enzo's
    developmont is still progressing vigorously, its software
    complexity and structure are making development increasingly slow
    and difficult.  Yet modifications are still needed, both to
    increase the physics capabilities required by users, and to
    improve the performance and scaling to bring \enzo\ from the
    Terascale Era into the Petascale Era.  By redeveloping \enzo's
    functionality for the same user base with 1) a more structured
    software development strategy, 2) using the modification-friendly
    object oriented programming paradigm, and 3) with more flexible
    and scalable datastructures, we can specifically target the
    hierarchical and massive parallelism and deep memory hierarchies
    of Petascale platforms \textit{a priori}.

% Future relationship between Enzo and Cello

    We believe that \enzo\ is the best solution in the short term,
    whereas \cello\ is the best solution in the long term.  To use a
    computational analogy, \enzo\ is like a short but relatively slow
    vector operation, whereas \cello\ is like a long but relatively
    fast vector operation.  \cello\ will require some time before it
    will start producing scientific results, but once it is
    operational, \cello's improved modifyability and scalability will
    eventually result in \cello\ users producing scientific results
    more frequently than \enzo\ users.  To optimize scientific results
    in both the short and long term, we believe it makes sense to
    continue development with \enzo\ while concurrently begining
    development on \cello.
  
%-----------------------------------------------------------------------
\subsection{Stakeholders}
%-----------------------------------------------------------------------


    To clarify roles among stakeholders---and for fun if not
    profit---we assign ``effective'' titles

\textbf{Executives}
\begin{itemize}
%
    \item  Prof.~Michael L.~Norman, effective \textit{Chief Executive Officer}
%
    \item Prof.~Greg Bryan, effective \textit{Chief Science Officer}.
    Since \cello\ is intended to be the successor to \enzo, it is
    important that \cello\ maintains the same high scientific
    standards set by the original application developed by Greg.
%
    \item Dr.~Robert Harkness, effective \textit{Chief Technology
    Officer}.  Robert is the most knowledgeable of the latest
    enhancements to \enzo, and he has the deepest understanding of how
    best to take advantage of PetaScale platforms
%
    \item The Development Team:
    \begin{itemize}
    \item Dr.~James Bordner (Team Leader, OOP design, documentation)
    \item Dr.~Robert Harkness (low-level design, performance, scaling)
    \item Rick Wagner (unit testing, user-interface, analysis specialist)
    \item Dr.~Alexei Kritsuk (hydrodynamics and turbulence specialist)
    \item David Collins (MHD specialist)
    \item Dr.~Dan Reynolds (RHD specialist)
    \item Prof.~Brian O'Shea (star formation specialist)
    \item Dr.~Paschal Paschos (chemistry and cooling specialist)
    \item Stephen Skory (quality assurance and application testing)
    \end{itemize}
    \end{itemize}

%-----------------------------------------------------------------------
\subsection{Users}
%-----------------------------------------------------------------------


    There are several partitions of users into groups.  One
    partitioning of users is into core developers, collaborrators, and
    community users.

    \textbf{Core users} are the scientists and graduate students in
    the Laboratory for Computational Astrophysics.  Most core users
    will also participate in development of \cello, and virtually all
    developers (except the \cello\ team leader Bordner) will be core
    users.  Core users include Norman, Kritsuk, Padoan, Harkness,
    Collins, Paschos, Wagner, and Skory.

    \textbf{Collaborative users} include all scientists not currently
    in the Laboratory for Computational Astrophysics, but who will use
    \cello\ and coauthors papers with current LCA members.
    Collaborative users may include Tom Abel, John Hayes, Bryan
    O'Shea, David Tytler, Dan Reynolds.

    \textbf{Community users} are all users not in the previous two
    groups.  We expect that \cello\ functionality will be driven
    primarily by the needs of the core and collaborative users, since
    they are fairly representative of general community users.
    However, \cello\ will be designed specifically to be a community
    code, so must meet the needs of community users as a whole.

    Another partitioning is users that will run \cello\ as provided,
    and users that will modify the code for their own particular
    needs.  For example, the latter group may want to add new physics
    support to \cello, or may want to use \cello\ to help develop
    their own numerical methods.

    Yet another partitioning of users is into scientists and students.
    Some users will be using \cello\ to perform computational
    experiments that will increase the global scientific knowledgebase,
    whereas other users will be using \cello\ in a more educational
    or learning capacity.

    Each partitioning of users leads to specific requirements of the
    \cello\ design and implementation.

%-----------------------------------------------------------------------
\subsection{Risks}
%-----------------------------------------------------------------------

    \begin{itemize} 
%
    \item Uncertain future of development team members (especially
    graduate students nearing graduation)
%
    \item Uneven capabilities of development team members (especially
    software development methodology and object oriented programming)
    % solution: 1. training;  2. isolating
%
    \item Uncertain future parallel platform characteristics
%
    \item Future platforms may have difficult to program accelerator cards
%
    \item Future platforms may have limited memory per shared memory node
%
    \item May become platform dependent: IBM Blue Waters
%
    \item Political pressure to use software languages, frameworks, or
    libraries, e.g.~CHARM++ or UPC, even if they may not be in the
    project's best interest
%
    \item Personal resistance to use software languages, frameworks,
    or libraries, even if they may be in the project's best interest
%
    \item Discontinued or uncertain future development of dependent libraries
%
    \begin{itemize}
        \item \code{lcaperf}
        \item \code{PAPI}
        \item \code{hypre}
        \item \code{P3DFFT}
        \item \code{SPRNG}
        \item \code{VisIt}
        \item \code{arprec}
    \end{itemize}
%
    \item Uncertain future usability of parallel languages
%
    \begin{itemize}
        \item \code{MPI}
        \item \code{OpenMP}
        \item \code{UPC}
    \end{itemize}
%
\end{itemize}

   


%-----------------------------------------------------------------------
\subsection{Assumptions}
%-----------------------------------------------------------------------

    % This is the list of assumptions that the stakeholders, users, or
    % project team have made. Often, these assumptions are generated
    % during a Wideband Delphi estimation session (see Chapter 3). If
    % Wideband Delphi is being used, the rest of the vision and scope
    % document should be ready before the Delphi meeting and used as the
    % basis for estimation. The assumptions generated during the
    % estimation kickoff meeting should then be reviewed, and any
    % nontechnical assumptions should be copied into this
    % section. (Technical assumptions---meaning assumptions that affect
    % the design and development but not the scope of the
    % project---should not be included in this document. The estimate
    % results will still contain a record of these assumptions, but they
    % are not useful for this particular audience.)

%=======================================================================
\section{Vision of the Solution}
%=======================================================================

%-----------------------------------------------------------------------
\subsection{Vision statement}
%-----------------------------------------------------------------------

    % The goal of the vision statement is to describe what the project
    % is expected to accomplish. It should explain what the purpose of
    % the project is. This should be a compelling reason, a solid
    % justification for spending time, money, and resources on the
    % project. The best time to write the vision statement is after
    % talking to the stakeholders and users and writing down their
    % needs; by this time, a concrete understanding of the project
    % should be starting to jell.

%   Cello is the next generation of Enzo

    The purpose of the \cello\ project is to produce the next
    generation of the open source software application \enzo\ for
    high-performance computational astrophysics and cosmology.  It
    will be used both as a testbed for experimenting with new software
    organization, parallelization, distributed datastructure,
    algorithm, and implementation techniques, as well as for enabling
    cutting-edge astrophysics and cosmological science simulations on
    the largest parallel high performance computers available.
    Objectives are to reliably provide high-quality numerical
    solutions, computed by distributing the workload efficiently
    across $10^5$ to $10^6$ computational cores, while maintaining a
    high level of utilization of available computational resources.

%   Primary goal

    The primary goal of \cello\ is to increase scientists' ability to
    perform numerical experiments of high scientific worth.  


%   Characteristics supporting the primary goal

    To support the primary goal if improved science, we recognize that
    \cello\ must improve on a range of characteristics.  Compared to
    \enzo, \cello\ should be more highly scalable, modifyable,
    useable, controlable, and robust.

%   Scalable

    \textbf{Scalable.}

    \textbf{Efficient on PetaScale platforms.}

%   Modifyable

    \textbf{Modifyable.}

    \enzo\ code is tightly coupled, repetitious, and difficult to read
    and understand.  Yet \enzo\ is an application that must be
    modified continually.  One reason is because rapidly evolving
    supercomputer platforms put ever-increasing demands on application
    scalability and efficiency.  A second reason is because scientific
    progress requires computational physics researchers and graduate
    students to solve new physics problems, which requires adding new
    computational physics capabilities.  Examples of capabilities
    currently being implemented in \enzo\ are radiative transfer,
    magnetohydrodynamics, and turbulence modeling.

    \enzo\ must be continually modified, yet is not specificially
    designed to be modified.  It was written primarily using the
    structured programming paradigm, which is known to be effective
    for designing and implementing large-scale applications, but is
    not ideal for subsequent redesign and modification.  As \enzo\ has
    been repeatedly modified by successive researchers and students,
    its complexity has increased rapidly, in particular much faster
    than the inherent complexity of the underlying functionality being
    implemented.

    To maintain control over how effectively \cello\ can be modified
    and enhanced to keep up with rapidly changing supercomputer
    platform characteristics and new physics capabilities, we also
    wish to keep \cello\ not overly-dependent on external libraries.
    Some libraries, such as HDF5 and MPI, are of course indispensible.
    But using some other libraries, frameworks, or languages, may lead to 
    \cello\ being a ``slave'' to another software group.


%   Usable

    \textbf{Ueable.}

    A secondary, but nontheless important, goal is to help
    physics students learn about astrophysics and cosmology, and
    about numerical methods for solving problems in these
    domains.  Keeping the use focused on the physics issues requires
    \cello\ to not be required to focus excessively on technical
    issues.

    Useablility will also be enhanced by \cello\
    Global performance-related measurements will be continuously
    collected for simulations.  This will include parallel
    communication amount, rates, and time; memory and computational
    load balance efficiency; memory usage and reference rates;
    floating point operation counts and rates; disk storage rates and
    amounts, and time.

    \textbf{Instructive.}  

%   Controlable

    \textbf{Powerful.}

    Input files will support a much more powerful grammar to allow for
    much greater control of \cello.  \enzo\ uses a flat list of
    parameters of simple type, which has limited scalability in terms
    of expressive range.

    \cello\ will use a hierarchical list of parameters, and will
    permit more complex types, such as mathematical expressions and
    composite spacial/temporal subregions.  While extra code will be
    required to parse and interpret the input files, the code in
    \enzo\ used for defining and initializing specific problem types
    will not be required, since the input file grammar will be
    sufficiently powerful to implement them without problem-specific
    code.  Overall, this may even result in less code, while
    supporting a much greater scope of problem setup without requiring
    modifying the code.
    
%   Robust


%

%  Performance monitoring

    %-----------------------------------------------------------------------
    \subsubsection{Performance monitoring}  
%




    % Enzo input files
    % flat list
    % limited expressibility
    % inherently limits power of application
% 
    \enzo\ currently uses a large ``flat'' list of input parameters.
    Because they are not organized explicitly into sections, it can be
    difficult for users to learn and write parameter files.
%
    Also, parameters have limited expressibility---each parameter is
    typically an integer with a small number of valid values.
%
    The limited input file grammar inherently limits the power of
    \enzo\ itself.  For example, input control is not sufficiently
    powerful to define a new test problem that is not already
    implemented explicitly in code.  Running a new test problem
    requires enhancing the \enzo\ codebase.


%-----------------------------------------------------------------------
\subsection{List of features}
%-----------------------------------------------------------------------

    % This section contains a list of features. A feature is as a
    % cohesive area of the software that fulfills a specific need by
    % providing a set of services or capabilities. Any software
    % package---in fact, any engineered product---can be broken down
    % into features. The project manager can choose the number of
    % features in the vision and scope document by changing the level of
    % detail or granularity of each feature, and by combining multiple
    % features into a single one. Sometimes those features are small
    % ("screw-top cap," "holds one liter of liquid"); sometimes they are
    % big ("four-wheel drive," "seats seven passengers"). It is useful
    % to describe a product in about 10 features in the vision and scope
    % document , because this usually yields a level of complexity that
    % most people reading it are comfortable with. Adding too many
    % features will overwhelm most readers.

    % Each feature should be listed in a separate paragraph or bullet
    % point. It should be given a name, followed by a description of the
    % functionality that it provides. This description does not need to
    % be detailed; it can simply be a few sentences that give a general
    % explanation of the feature. However, if there is more information
    % that a stakeholder or project team member feels should be
    % included, it is important to include that information. For
    % example, it is sometimes useful to include a use case (see Chapter
    % 6), as long as it is written in such a way that all of the
    % stakeholders can read and understand it.

    %-----------------------------------------------------------------------
    \subsubsection{Physics capabilities}

    The initial scope of physics capabilities of \cello\ will be the
    same as \enzo: representations of baryonic and dark matter,
    hydrodynamics, self-gravity, multi-species chemistry, radiative
    cooling, cosmological expansion, star formation,
    magnetohydrodynamics, and radiation transfer.

    Additional physics capabilities are expected to be implemented
    during \cello's lifetime.  [more]

    %-----------------------------------------------------------------------
    \subsubsection{Numerical methods}

    Numerical methods will be a combination of existing numerical
    methods taken from \enzo, and new and improved methods that
    have been developed more recently.

    \begin{description}
% 
    \item[Hydrodynamics: ] Modified Piecewise Parabolic Method (PPM)
    solvers, including both the existing method in \enzo, a
    non-operator split variation, and a reduced stencil version.
%
    \item[Dark matter dynamics: ] a Particle-Mesh (PM) method
%
    \item[Self-gravity: ]  
    \end{description}

    %-----------------------------------------------------------------------
    \subsubsection{Parallelism} 

% Hierarchical parallel tasks

    To take advantage of the hierarchical parallel nature of modern
    supercomputers, \cello\ will support multiple levels of
    parallelization in its methods and datastructures.  This will
    include the coarse-grain parallelism to concurrently run
    simulations in an ensemble at the node or supernode level,
    medium-grain parallelism to evolve grid patches or particle groups
    of a simulation concurrently across a distributed memory subsystem
    at the node or socket level, and fine-grain parallelism to evolve
    grid cells within grid patches or particles within a group using
    shared-memory parallelism at the socket or core level.

% Hierarchical parallel technologies

    This parallelization will be modularized (as much as
    technologically feasible) to improve flexibility in choosing the
    best method(s) of parallelization for a given problem on a given
    parallel platform.  MPI (two-sided), MPI2 (one-sided), OpenMP, and
    possibly UPC support will be included, as well as flexibility in
    choosing hybrid schemes such as MPI + OpenMP, or MPI + UPC.  Other
    schemes based on shared memory array libraries or POSIX pthreads
    may also be considered.  Parallelization methods will be primarily
    data parallelism, supporting both distributed memory and shared
    memory in isolation or combination.  Support for collaberative
    (functional) parallelism and pipelining will also be considered.

%  Load balancing

    Multiple levels of parallel tasks will be load balanced using
    hierarchical dynamic load balancing algorithms.  Load balancing
    schemes will use dynamically measured performance data gathered by
    the running simulation to make load balancing decisions, and will
    allow flexibility in optimizing the parallel distribution of
    computation, memory storage, or a combination of both.  Combining
    flexible hierarchical parallelization schemes with hierarchical
    load balancing algorithms, together with scalable methods and
    efficient AMR data-structures, are expected to lead to high
    parallel efficiency and scalability.

%   Task scheduling (reduced synchronization)

    %-----------------------------------------------------------------------
    \subsubsection{Datastructures}

    
    As with \enzo, computations will be performed at multiple spacial
    and temporal resolutions using adaptive mesh refinement (AMR).
    This will permit the physics modules to capture the full range of
    scales of interest, but without excessive computation and memory
    storage.  However, the AMR approach and data-structures will be
    totally redesigned and reimplemented.  Compared to \enzo, the new
    design will include the following features:

    \begin{itemize}
    \item Improved load balancing
    \item Improved data locality
    \item Reduced memory usage and improved memory scaling
    \item Control of task sizes at different parallelization levels
    \item Improved cache usage through array blocking and padding techniques
    \item Support for hierarchical parallelism, e.g.~MPI + OpenMP
    \item Object-oriented design to reduce overall code complexity and 
          increase flexibility and modifyability
    \end{itemize}

    % Octree to improve load balancing and datastructure complexity
    % patches to maintain efficiency
    % variable patch size to reduce octree overhead
    %   (this is new)

    % patches may be distributed (MPI), threaded (OMP), and blocked (cache)

    The properties of individual grid patches, and details of how
    scalar and vector fields are stored within grid patches, will be
    flexible to permit optimizing the use of node and processor/core
    parallelism, as well as deep memory/cache hierarchies.  This
    includes hierarchical subblocking of grid patch arrays to optimize
    MPI task size, OMP task size, and cache block size and
    configuration.  Additionally, arrays may be padded to reduce cache
    thrashing effects for low-associativity caches, and may be
    interleaved to improve data locality (or to ease use of existing
    computational kernels that use interleaved array data).  These
    capabilities, together with efficient methods and implementation
    of computations, are expected to lead to high single-thread
    computational efficiency and data movement through memory/cache
    hierarchies.  Implementation of the \code{Array} class is expected
    to use the object-oriented ``decorator pattern''.


    % Particles


%-----------------------------------------------------------------------
\subsection{Scope of phased release (optional)}
%-----------------------------------------------------------------------

    % Sometimes software projects are released in phases: a version of
    % the software with some subset of the features is released first,
    % and a newer, more complete version is released later. This section
    % describes the plan for a phased release, if that approach is to be
    % taken.

    % This is useful when there is an important deadline for the
    % software, but developing the entire software project by that
    % deadline would be unrealistic. The most common way to compromise
    % on this release date is to divide the features into two or more
    % releases. In that case, this section should identify specifically
    % when those versions will be released, and which features will be
    % included in each version. It's reasonable to divide one feature up
    % between two releases, as long as it is made clear exactly how that
    % will happen.

    % If a project manager needs to release a project in phases, it is
    % critical that the project team be consulted. Some features are
    % much more difficult to divide than others, and the engineers might
    % see dependencies between features that are not clear to the
    % stakeholders and project manager. After the phased release plan is
    % written down and agreed upon, the project team should always be
    % asked to re-estimate the effort and a new project plan should be
    % generated (see below). This will ensure that the phased release is
    % feasible and compatible with the organization's priorities.

%-----------------------------------------------------------------------
\subsection{Features that will not be developed}
%-----------------------------------------------------------------------


    % Features are often left out of a project on purpose. When a
    % feature is explicitly left out of the software, it should be added
    % to this section to tell the reader that a decision was made to
    % exclude it. For example, one way to handle an unrealistic deadline
    % is by removing one or more features from the software, in which
    % case the removed features should be moved into this section. The
    % reason these features should be moved rather than deleted from the
    % document is that otherwise, readers might assume that they were
    % overlooked and bring them up in a review. This is especially
    % important during the review of the document because it allows
    % everyone to agree on the exclusion of the feature (or object to
    % it).

\end{document}

%==================================================================
