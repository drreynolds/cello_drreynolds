%0       1         2         3         4         5         6         7         8
%2345678901234567890123456789012345678901234567890123456789012345678901234567890
%=======================================================================
\documentclass[12pt]{article}
%=======================================================================


% INCLUDE DEVELOPMENT TEXT

\newcommand{\devel}[1]{\textbf{#1}}

% EXCLUDE DEVELOPMENT TEXT

% \newcommand{\devel}[1]{}


%=======================================================================
% Document layout
%=======================================================================

\setlength{\topmargin}{0.0in}
\setlength{\oddsidemargin}{0.0in}
\setlength{\evensidemargin}{0.0in}
\setlength{\textwidth}{6.0in}
\setlength{\textheight}{9.0in}

%=======================================================================
% Packages
%=======================================================================

\usepackage{wasysym}
\usepackage{epsfig}
\usepackage{url}

%=======================================================================
% Commands
%=======================================================================

\newcommand{\cello}{\textsf{Cello}}
\newcommand{\enzo}{\textsf{Enzo}}
\newcommand{\lcaperf}{\textsf{lcaperf}}
\newcommand{\lcatest}{\textsf{lcatest}}

\newcommand{\code}[1]{\textsf{#1}}

\newcommand{\note}[1]{\devel{\eighthnote\ \textit{#1} \\}}
\newcommand{\pargraph}[1]{\devel{\P\ \textbf{#1} \\}}

\newcommand{\todo}{\devel{$\circ$}}
\newcommand{\done}{\devel{$\bullet$}}
\newcommand{\halfdone}{\devel{\textcolor{gray}{$\bullet$}}}

\newcommand{\PROJECT}{\cello}

\newcommand{\TITLE}[3]{
\title{ {\huge \PROJECT\ #1}  \\ \vspace{0.1in}
     {\small Document Version: \textbf{#3}} \vspace{-0.1in}
    }
\author{      #2 \\
        Laboratory for Computational Astrophysics\\
        University of California, San Diego}
\maketitle}

%=======================================================================


%=======================================================================

\begin{document}

%=======================================================================
\TITLE{Vision and Scope Document}{James Bordner}{$Rev$}
%=======================================================================

\section{Problem Statement}
\subsection{Project background}

    \textit{This section contains a summary of the problem that the project
    will solve. It should provide a brief history of the problem and
    an explanation of how the organization justified the decision to
    build software to address it. This section should cover the
    reasons why the problem exists, the organization's history with
    this problem, any previous projects that were undertaken to try to
    address it, and the way that the decision to begin this project
    was reached.}

    The \cello\ Project is designed to be \enzo: The Next Generation.
    \enzo\ was initially conceived in the early 1990's by Michael
    L.~Norman and Greg Bryan to be a structured (patch-based) adaptive
    mesh refinement (SAMR) cosmological application.  It incorporated
    a modified high-order Piecewise Parabolic Method (PPM) solver for
    hydrodynamics, and a Particle-Mesh (PM) method for dark matter
    dynamics.

    Greg Bryan began coding \enzo\ in the Fall of 1994.  He
    implemented it in C++ and Fortran 77, using primarily a structured
    programming paradigm.  Initially he targeted shared memory
    machines such as the SGI Origin, and later added support for
    distributed memory machines via the Message Passing Interface
    (MPI).  Beginning around 2000, other developers began to modify
    \enzo\ in earnest, especially Robert Harkness, who added support
    for 64-bit integers, packed AMR output, [\ldots].

    \enzo\ was developed without any supporting documents, such as
    requirement specifications, design description, or test document.
    It was also written without any explicit coding standard, though
    the original author closely followed implicit coding guidelines,
    such as including uniform file header comment blocks, and using
    descriptive variable and function names.  The original author also
    implemented many application tests to help ensure the accuracy
    of the computed solutions.

    A fundamental flaw of \enzo\ is that it is continually modified,
    yet not designed to be modified.  It was written primarily using
    the structured programming paradigm, which is known to be
    effective for designing and implementing large-scale applications,
    but is not ideal for subsequent redesign and modification.  As
    \enzo\ has been repeatedly modified by successive researchers and
    students, its complexity has increased steadily and without bound.

    Yet \enzo\ is an application that must be modified continually,
    for at least two reasons.  First, because rapidly evolving
    supercomputer platforms put ever-increasing demands on application
    scalability and efficiency; and second, because a primary usage
    model is physics researchers and graduate students who want to
    solve new physics problems, which necessarily requires adding new
    computational physics capabilities.
    

\subsection{Stakeholders}

    \textit{This is a bulleted list of the stakeholders. Each stakeholder may
    be referred to by name, title, or role ("support group manager,"
    "CTO," "senior manager"). The needs of each stakeholder are
    described in a few sentences.}

\subsection{Users}

    \textit{This is a bulleted list of the users. As with the stakeholders,
    each user can either be referred to by name or role ("support
    rep," "call quality auditor," "home web site user")however, if
    there are many users, it is usually inefficient to try to name
    each one. The needs of each user are described.}

\subsection{Risks}

    \textit{This section lists any potential risks to the project. It should
    be generated by a project team's brainstorming session. It could
    include external factors that may impact the project, or issues or
    problems that could potentially cause project delays or raise
    issues. (The process for assessing and mitigating risk below can
    be used to generate the risks for this section.)}

\subsection{Assumptions}

    \textit{This is the list of assumptions that the stakeholders, users, or
    project team have made. Often, these assumptions are generated
    during a Wideband Delphi estimation session (see Chapter 3). If
    Wideband Delphi is being used, the rest of the vision and scope
    document should be ready before the Delphi meeting and used as the
    basis for estimation. The assumptions generated during the
    estimation kickoff meeting should then be reviewed, and any
    nontechnical assumptions should be copied into this
    section. (Technical assumptions---meaning assumptions that affect
    the design and development but not the scope of the
    project---should not be included in this document. The estimate
    results will still contain a record of these assumptions, but they
    are not useful for this particular audience.)}

\section{Vision of the Solution}

\subsection{Vision statement}

    \textit{The goal of the vision statement is to describe what the project
    is expected to accomplish. It should explain what the purpose of
    the project is. This should be a compelling reason, a solid
    justification for spending time, money, and resources on the
    project. The best time to write the vision statement is after
    talking to the stakeholders and users and writing down their
    needs; by this time, a concrete understanding of the project
    should be starting to jell.}

\subsection{List of features}

    \textit{This section contains a list of features. A feature is as a
    cohesive area of the software that fulfills a specific need by
    providing a set of services or capabilities. Any software
    package---in fact, any engineered product---can be broken down
    into features. The project manager can choose the number of
    features in the vision and scope document by changing the level of
    detail or granularity of each feature, and by combining multiple
    features into a single one. Sometimes those features are small
    ("screw-top cap," "holds one liter of liquid"); sometimes they are
    big ("four-wheel drive," "seats seven passengers"). It is useful
    to describe a product in about 10 features in the vision and scope
    document , because this usually yields a level of complexity that
    most people reading it are comfortable with. Adding too many
    features will overwhelm most readers.}

    \textit{Each feature should be listed in a separate paragraph or bullet
    point. It should be given a name, followed by a description of the
    functionality that it provides. This description does not need to
    be detailed; it can simply be a few sentences that give a general
    explanation of the feature. However, if there is more information
    that a stakeholder or project team member feels should be
    included, it is important to include that information. For
    example, it is sometimes useful to include a use case (see Chapter
    6), as long as it is written in such a way that all of the
    stakeholders can read and understand it.}

\subsection{Scope of phased release (optional)}

    \textit{Sometimes software projects are released in phases: a version of
    the software with some subset of the features is released first,
    and a newer, more complete version is released later. This section
    describes the plan for a phased release, if that approach is to be
    taken.}

    \textit{This is useful when there is an important deadline for the
    software, but developing the entire software project by that
    deadline would be unrealistic. The most common way to compromise
    on this release date is to divide the features into two or more
    releases. In that case, this section should identify specifically
    when those versions will be released, and which features will be
    included in each version. It's reasonable to divide one feature up
    between two releases, as long as it is made clear exactly how that
    will happen.}

    \textit{If a project manager needs to release a project in phases, it is
    critical that the project team be consulted. Some features are
    much more difficult to divide than others, and the engineers might
    see dependencies between features that are not clear to the
    stakeholders and project manager. After the phased release plan is
    written down and agreed upon, the project team should always be
    asked to re-estimate the effort and a new project plan should be
    generated (see below). This will ensure that the phased release is
    feasible and compatible with the organization's priorities.}

\subsection{Features that will not be developed}

    \textit{Features are often left out of a project on purpose. When a
    feature is explicitly left out of the software, it should be added
    to this section to tell the reader that a decision was made to
    exclude it. For example, one way to handle an unrealistic deadline
    is by removing one or more features from the software, in which
    case the removed features should be moved into this section. The
    reason these features should be moved rather than deleted from the
    document is that otherwise, readers might assume that they were
    overlooked and bring them up in a review. This is especially
    important during the review of the document because it allows
    everyone to agree on the exclusion of the feature (or object to
    it).}

\end{document}

%==================================================================
