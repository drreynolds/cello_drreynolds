%0       1         2         3         4         5         6         7         8
%2345678901234567890123456789012345678901234567890123456789012345678901234567890

%=======================================================================
\documentclass{article}
%=======================================================================

\usepackage{color}
\usepackage{verbatim}

% INCLUDE DEVELOPMENT TEXT

\newcommand{\devel}[1]{\textbf{#1}}

% EXCLUDE DEVELOPMENT TEXT

% \newcommand{\devel}[1]{}


%=======================================================================
% Document layout
%=======================================================================

\setlength{\topmargin}{0.0in}
\setlength{\oddsidemargin}{0.0in}
\setlength{\evensidemargin}{0.0in}
\setlength{\textwidth}{6.0in}
\setlength{\textheight}{9.0in}

%=======================================================================
% Packages
%=======================================================================

\usepackage{wasysym}
\usepackage{epsfig}
\usepackage{url}

%=======================================================================
% Commands
%=======================================================================

\newcommand{\cello}{\textsf{Cello}}
\newcommand{\enzo}{\textsf{Enzo}}
\newcommand{\lcaperf}{\textsf{lcaperf}}
\newcommand{\lcatest}{\textsf{lcatest}}

\newcommand{\code}[1]{\textsf{#1}}

\newcommand{\note}[1]{\devel{\eighthnote\ \textit{#1} \\}}
\newcommand{\pargraph}[1]{\devel{\P\ \textbf{#1} \\}}

\newcommand{\todo}{\devel{$\circ$}}
\newcommand{\done}{\devel{$\bullet$}}
\newcommand{\halfdone}{\devel{\textcolor{gray}{$\bullet$}}}

\newcommand{\PROJECT}{\cello}

\newcommand{\TITLE}[3]{
\title{ {\huge \PROJECT\ #1}  \\ \vspace{0.1in}
     {\small Document Version: \textbf{#3}} \vspace{-0.1in}
    }
\author{      #2 \\
        Laboratory for Computational Astrophysics\\
        University of California, San Diego}
\maketitle}

%=======================================================================


%=======================================================================

\begin{document}
\tableofcontents
%=======================================================================
\TITLE{\cello\ User Manual}{James Bordner}{$Rev$}
%=======================================================================

%=======================================================================
\section{Introduction} \label{s:intro}
%=======================================================================

%=======================================================================
\section{Parameters}
%=======================================================================

\definecolor{darkgreen}{rgb}{0.0,0.5,0.0}

\newcommand{\Parameter}[8]
% 1 Group
% 2 subgroup
% 3 parameter
% 4 type
% 5 type restrictions
% 6 description
% 7 example
% 8 default
{\subsubsection{\code{#2} \code{#3}}
\begin{tabular}{|l|l|l|l|}  \hline & & & \\ [-1.5ex]
   \textcolor{red}{\code{#1}} & 
   \textcolor{red}{\code{#2}} &
   \textcolor{black}{\textbf{\code{#3}}} & \\ [1ex] \hline 
   \multicolumn{4}{|l|}{} \\ [-1.5ex]
   \multicolumn{4}{|l|}{\textcolor{darkgreen}{#4}  \textcolor{blue}{#5} } \\
   \multicolumn{4}{|l|}{} \\ [-1.5ex] \hline 
   \multicolumn{4}{|l|}{} \\ [-0.5ex] 
  \multicolumn{4}{|l|}{\parbox[l]{5.75in}{\raggedright\textsf{\textit{\textcolor{magenta}{#6}}}}} \\ [-1.5ex]
   \multicolumn{4}{|l|}{} \\ [1ex]  \hline
  \multicolumn{4}{l}{\parbox[l]{5.75in}{#7}} \\ [-2.5ex]
   \multicolumn{4}{l}{} \\
\end{tabular}}


% \newcommand{\Parameter}[8]
% % 1 Group
% % 2 subgroup
% % 3 parameter
% % 4 type
% % 5 type restrictions
% % 6 description
% % 7 example
% % 8 default
% {\subsubsection{\code{#1}: \code{#3}} 
% \begin{tabbing}
% xxxxxxxxxxxxxx\=\kill
% Parameter: \> \textbf{\code{#3}} \\
% Full parameter: \> \code{#1} : #2 : \textbf{\code{#3}} \\
% Type: \> #4 \\
% Restrictions: \> #5 \\
% Description: \> \parbox[t]{5.75in}{\raggedright #6} \\
% Examples: \> \parbox[t]{5.75in}{\raggedright #7} \\
% Default: \> #8
% \end{tabbing}
% }



%-----------------------------------------------------------------------
\subsection{\code{Boundary}}
%-----------------------------------------------------------------------

\Parameter
{Boundary}
{[ \code{lower} $|$ \code{upper} ]}
{type}
{string $|$ list(string)}
{ [ \code{"reflecting"} $|$ \code{"inflow"}$|$ \code{"outflow"}$|$
  \code{"periodic"}$|$ \code{"dirichlet"}$|$ \code{"neumann"} ] }
{ Type of boundary condition.  Corresponds to \enzo\ parameters
  \code{LeftFaceBoundaryCondition} and
  \code{RightFaceBoundaryCondition}, whose allowed values are 0 = reflecting, 1 = outflow, 2 = inflow, 3 = periodic.}
{
\begin{tabbing}
xxx\=xxx\=xxx\kill
 \code{Boundary \{} \\
\>\code{\textbf{type} = "reflecting";} \\
\>   \code{lower \{ \textbf{type} = "inflow"\}} \\
\>   \code{upper \{ } \\
\>\>      \code{\textbf{type} = [ "reflecting" , "outflow" , "reflecting" ]} \\
\>   \code{\}} \\
\code{\}}
\end{tabbing}
}
{}

%--------------------------------------------------
\Parameter
{Boundary}
{[ \code{lower} $|$ \code{upper} ]}
{value}
{scalar expression $|$ list(scalar expression) }
{}
{ Value for the boundary inflow, outflow, neumann, or dirichlet
  boundary condition.  \incomplete{how to handle both ``lower'' and
    ``upper'', and field names} .}
{}

%--------------------------------------------------
\Parameter
% Group
{Boundary} 
% subgroup
{[ \code{lower} $|$ \code{upper} ]} 
% parameter
{file} 
% type
{string $|$ list(string)} 
% type restrictions
{} 
% description
{File from which to read boundary conditions. \incomplete{how to
    handle both ``lower'' and ``upper'', and field names} } 
% example
{} 
% default
{}

%-----------------------------------------------------------------------
\subsection{\code{Domain}}
%-----------------------------------------------------------------------

\Parameter
% Group
{Domain} 
% subgroup
{} 
% parameter
{extent} 
% type
{list(scalar)} 
{[xmin, xmax \{, ymin, ymax \{, zmin,zmax\}\}]}
% description
{Extent of the domain} 
% example
{} 
% default
{}

%-----------------------------------------------------------------------
\subsection{\code{Field}}
%-----------------------------------------------------------------------

\Parameter
% Group
{Field} 
% subgroup
{} 
% parameter
{courant} 
% type
{scalar} 
% type restrictions
{} 
% description
{Courant condition for fields} 
% example
{} 
% default
{}

%--------------------------------------------------
\Parameter
% Group
{Field} 
% subgroup
{} 
% parameter
{fields} 
% type
{list(string)} 
% type restrictions
{} 
% description
{List of field names} 
% example
{} 
% default
{}

%--------------------------------------------------
\Parameter
% Group
{Field} 
% subgroup
{} 
% parameter
{groups} 
% type
{list(string)} 
% type restrictions
{} 
% description
{List of group names} 
% example
{} 
% default
{}

%--------------------------------------------------
\Parameter
% Group
{Field}
% subgroup
{}
% parameter
{alignment}
% type
{integer}
% type restrictions
{}
% description
{Alignment of start of each field's array in bytes}
% example
{}
% default
{}

%--------------------------------------------------
\Parameter
% Group
{Field}
% subgroup
{}
% parameter
{padding}
% type
{integer}
% type restrictions
{}
% description
{Padding between field arrays in bytes}
% example
{}
% default
{}

\Parameter
% Group
{Field}
% subgroup
{[ \textit{field-name} ]} 
% parameter
{ghosts}
% type
{list(int)}
% type restrictions
{}
% description
{List of ghost zone depths per axis for field if specified, default for all fields if not}
% example
{}
% default
{}

%--------------------------------------------------
\Parameter
% Group
{Field} 
% subgroup
{[ \textit{field-name} ]} 
% parameter
{precision} 
% type
{string} 
% type restrictions
{[  \code{"single"} 
$|$ \code{"double"} 
$|$ \code{"extended72"}
$|$ \code{"extended80"}
$|$ \code{"quadruple"} ] } 
% description
{Precision for the field if specified, or default precision for all fields if not.} 
% example
{} 
% default
{}

%--------------------------------------------------
\Parameter
% Group
{Field}
% subgroup
{[ \textit{field-name} ]} 
% parameter
{minimum}
% type
{list(scalar,string)}
% type restrictions
{\code{"none"} $|$ \code{"assign"} $|$ \code{"warning"} $|$ \code{"error"}}
% description
{minimum value(s) and actions }
% example
{}
% default
{}

%--------------------------------------------------
\Parameter
% Group
{Field}
% subgroup
{[ \textit{field-name} ]} 
% parameter
{maximum}
% type
{list(scalar,string)}
% type restrictions
{\code{"none"} $|$ \code{"assign"} $|$ \code{"warning"} $|$ \code{"error"}}
% description
{maximum value(s) and actions }
% example
{}
% default
{}

%--------------------------------------------------
\Parameter
% Group
{Field}
% subgroup
{[ \textit{field-name} ]} 
% parameter
{centering}
% type
{list(logical)}
% type restrictions
{}
% description
{Whether the variable is centered on the corresponding axis.}
% example
{}
% default
{}

%--------------------------------------------------
\Parameter
% Group
{Field}
% subgroup
{\textit{field-name}}
% parameter
{alias}
% type
{list(string)}
% type restrictions
{}
% description
{List of other names this field may be referred to}
% example
{}
% default
{}

%-----------------------------------------------------------------------
\subsection{\code{Initial}}
%-----------------------------------------------------------------------

\Parameter
% Group
{Initial}
% subgroup
{\textit{field-name}}
% parameter
{value}
% type
{list(scalar,logical-expression)}
% type restrictions
{}
% description
{Initial value for the field}
% example
{}
% default
{}

%--------------------------------------------------
\Parameter
% Group
{Initial}
% subgroup
{\textit{field-name}}
% parameter
{file}
% type
{string}
% type restrictions
{}
% description
{File containing the initial conditions for the field}
% example
{}
% default
{}

%--------------------------------------------------
\Parameter
% Group
{Initial}
% subgroup
{}
% parameter
{time}
% type
{scalar}
% type restrictions
{}
% description
{Initial time in code units}
% example
{}
% default
{}

%--------------------------------------------------
\Parameter
% Group
{Initial}
% subgroup
{}
% parameter
{redshift}
% type
{scalar}
% type restrictions
{}
% description
{initial redshift for cosmological runs}
% example
{}
% default
{}

%--------------------------------------------------
\Parameter
% Group
{Initial}
% subgroup
{}
% parameter
{cycle}
% type
{integer $|$ list(integer)}
% type restrictions
{}
% description
{initial cycle number, or list of cycle numbers for each AMR level}
% example
{}
% default
{}

%-----------------------------------------------------------------------
\subsection{\code{Stopping}}
%-----------------------------------------------------------------------

\Parameter
% Group
{Stopping}
% subgroup
{}
% parameter
{time}
% type
{scalar}
% type restrictions
{}
% description
{stopping time}
% example
{}
% default
{}

%--------------------------------------------------
\Parameter
% Group
{Stopping}
% subgroup
{}
% parameter
{redshift}
% type
{scalar}
% type restrictions
{}
% description
{stopping redshift}
% example
{}
% default
{}
% default
{}

%--------------------------------------------------
\Parameter
% Group
{Stopping}
% subgroup
{}
% parameter
{cycle}
% type
{integer $|$ list(integer)}
% type restrictions
{}
% description
{stopping cycle, or list of stopping cycles for AMR levels}
% example
{}


%-----------------------------------------------------------------------
\subsection{\code{Memory}}
%-----------------------------------------------------------------------

\Parameter
% Group
{Memory}
% subgroup
{}
% parameter
{limit}
% type
{integer}
% type restrictions
{}
% description
{Set limit on the number of local bytes available for dynamic memory}
% example
{}
% default
{}

%--------------------------------------------------
\Parameter
% Group
{Memory}
% subgroup
{}
% parameter
{threshold}
% type
{scalar}
% type restrictions
{}
% description
{Set efficiency threshold before displaying a warning message}
% example
{}
% default
{}

%--------------------------------------------------
\Parameter
% Group
{Memory}
% subgroup
{}
% parameter
{fill}
% type
{list(string)}
% type restrictions
{}
% description
{List of strings to fill memory after being allocated, and before being deallocated}
% example
{}
% default
{}

%--------------------------------------------------
\Parameter
% Group
{Memory}
% subgroup
{}
% parameter
{tracking}
% type
{logical}
% type restrictions
{}
% description
{Set memory tracking on or off. Turning off resorts to native new[]/delete[].'}
% example
{}
% default
{}

%-----------------------------------------------------------------------
\subsection{\code{Mesh}}
%-----------------------------------------------------------------------

\Parameter
% Group
{Mesh}
% subgroup
{}
% parameter
{root\_size}
% type
{list(integer)}
% type restrictions
{}
% description
{Size of the root grid, e.g. [400,400]}
% example
{}
% default
{}

%--------------------------------------------------
\Parameter
% Group
{Mesh}
% subgroup
{}
% parameter
{max\_levels}
% type
{integer}
% type restrictions
{}
% description
{Maximum AMR mesh levels (with root = 1), assuming refinement by two}
% example
{}
% default
{}

%--------------------------------------------------
\Parameter
% Group
{Mesh}
% subgroup
{}
% parameter
{refinement}
% type
{integer}
% type restrictions
{[2 $|$ 4 $|$ 8]}
% description
{Refinement factor.}
% example
{}
% default
{}

%--------------------------------------------------
\Parameter
% Group
{Mesh}
% subgroup
{}
% parameter
{balanced}
% type
{logical}
% type restrictions
{}
% description
{Whether the tree is balanced or "full" (ala Flash) or not}
% example
{}
% default
{}

%--------------------------------------------------
\Parameter
% Group
{Mesh}
% subgroup
{}
% parameter
{backfill}
% type
{logical}
% type restrictions
{}
% description
{Whether to backfill for \code{refine} $>$ 2}
% example
{}
% default
{}

%--------------------------------------------------
\Parameter
% Group
{Mesh}
% subgroup
{}
% parameter
{coalesce}
% type
{logical}
% type restrictions
{}
% description
{whether to coalesce small patches into one big one}
% example
{}
% default
{}

%--------------------------------------------------
\Parameter
% Group
{Mesh}
% subgroup
{}
% parameter
{patch\_size}
% type
{list(integer)}
% type restrictions
{}
% description
{minimum and maximum allowed patch sizes}
% example
{}
% default
{}

%--------------------------------------------------
\Parameter
% Group
{Mesh}
% subgroup
{}
% parameter
{block\_size}
% type
{list}
% type restrictions
{}
% description
{Minimum and maximum allowed block sizes}
% example
{}
% default
{}

%-----------------------------------------------------------------------
\subsection{\code{Method}}
%-----------------------------------------------------------------------

\Parameter
% Group
{Method}
% subgroup
{}
% parameter
{sequence}
% type
{list(string)}
% type restrictions
{}
% description
{Sequence of numerical methods to apply}
% example
{}
% default
{}

\Parameter
% Group
{Method}
% subgroup
{ppm}
% parameter
{use\_minimum\_pressure\_support}
% type
{logical}
% type restrictions
{}
% description
{Enzo \code{UseMinimumPressureSupport}}
% example
{}
% default
{}
 	 	 	 	
\Parameter
% Group
{Method}
% subgroup
{ppm}
% parameter
{minimum\_pressure\_support\_parameter}
% type
{integer}
% type restrictions
{}
% description
{Enzo MinimumPressureSupportParameter}
% example
{}
% default
{}

 	 	 	 	
\Parameter
% Group
{Method}
% subgroup
{ppm}
% parameter
{temperature\_floor}
% type
{scalar}
% type restrictions
{}
% description
{Temperature floor, was "tiny\_number"}
% example
{}
% default
{}

 	 	 	 	
\Parameter
% Group
{Method}
% subgroup
{ppm}
% parameter
{pressure\_floor}
% type
{scalar}
% type restrictions
{}
% description
{Pressure floor, was "tiny\_number"}
% example
{}
% default
{}

 	 	 	 	
\Parameter
% Group
{Method}
% subgroup
{ppm}
% parameter
{density\_floor}
% type
{scalar}
% type restrictions
{}
% description
{Density floor, was "tiny\_number"}
% example
{}
% default
{}

 	 	 	 	
\Parameter
% Group
{Method}
% subgroup
{ppm}
% parameter
{number\_density\_floor}
% type
{scalar}
% type restrictions
{}
% description
{Number density floor, was "tiny\_number"}
% example
{}
% default
{}

%--------------------------------------------------
\Parameter
% Group
{Method}
% subgroup
{ppm}
% parameter
{diffusion}
% type
{logical}
% type restrictions
{}
% description
{PPM diffusion parameter}
% example
{}
% default
{}

%--------------------------------------------------
\Parameter
% Group
{Method}
% subgroup
{ppm}
% parameter
{flattening}
% type
{logical}
% type restrictions
{}
% description
{PPM flattening parameter}
% example
{}
% default
{}

%--------------------------------------------------
\Parameter
% Group
{Method}
% subgroup
{ppm}
% parameter
{steepening}
% type
{logical}
% type restrictions
{}
% description
{PPM steepening parameter}
% example
{}
% default
{}

%--------------------------------------------------
\Parameter
% Group
{Method}
% subgroup
{ppm}
% parameter
{dual\_energy}
% type
{logical}
% type restrictions
{}
% description
{Whether to use dual-energy formalism}
% example
{}
% default
{}

%--------------------------------------------------
\Parameter
% Group
{Method}
% subgroup
{ppm}
% parameter
{dual\_energy\_eta\_1}
% type
{scalar}
% type restrictions
{}
% description
{Dual-energy formalism parameter}
% example
{}
% default
{}

%--------------------------------------------------
\Parameter
% Group
{Method}
% subgroup
{ppm}
% parameter
{dual\_energy\_eta\_2}
% type
{scalar}
% type restrictions
{}
% description
{Dual-energy formalism parameter}
% example
{}
% default
{}

%--------------------------------------------------
\Parameter
% Group
{Method}
% subgroup
{ppm}
% parameter
{pressure\_free}
% type
{logical}
% type restrictions
{}
% description
{``pressure free flag''}
% example
{}
% default
{}

%-----------------------------------------------------------------------
\subsection{\code{Output}}
%-----------------------------------------------------------------------

\Parameter
% Group
{Output}
% subgroup
{\textit{file-group}}
% parameter
{name}
% type
{list(string,variable)}
% type restrictions
{}
% description
{Name for the output files, where the first element is a format string (ala printf()), and remaining elements are scalar expressions. E.g. \code{["wave\_pool-t=\%3.1f.data", t]}}
% example
{}
% default
{}

%--------------------------------------------------
\Parameter
% Group
{Output}
% subgroup
{\textit{file-group}}
% parameter
{times}
% type
{list(scalar)}
% type restrictions
{}
% description
{Start, stop, and interval times to output}
% example
{}
% default
{}

%--------------------------------------------------
\Parameter
% Group
{Output}
% subgroup
{\textit{file-group}}
% parameter
{redshifts}
% type
{list(scalar)}
% type restrictions
{}
% description
{Start, stop, and interval redshifts to output}
% example
{}
% default
{}

%--------------------------------------------------
\Parameter
% Group
{Output}
% subgroup
{\textit{file-group}}
% parameter
{cycles}
% type
{list(integer}
% type restrictions
{}
% description
{start, stop, and interval cycles to output. \incomplete{what about fine vs coarse?}}
% example
{}
% default
{}

%-----------------------------------------------------------------------
\subsection{\code{Parallel}}
%-----------------------------------------------------------------------

%--------------------------------------------------
\Parameter
{Parallel}{}{size}
% type
{list(integer)}{}
% description
{List of parallelization levels, e.g.~[8,32] for 8 cores/cpu, 32 cpu's/node, and any number of nodes.}
% example
{}
% default
{}

%--------------------------------------------------
\Parameter
% Group
{Parallel}{}{type}
% type
{list(string)}
% type restrictions
{\code{"shared"} $|$ \code{"distributed"}}
% description
{List of parallelization types, either \code{"shared"}-memory or \code{"distributed"}-memory. Other types may be added later, such as \code{"gpu"} for GPU's, or \code{"virtual"} for CHARM++}
% example
{}
% default
{}

%--------------------------------------------------
\Parameter
% Group
{Parallel}{}{method}
% type
{list(string)}
{\code{"mpi"} $|$ \code{"omp"} $|$ \code{"upc"} $|$ \code{"serial"}}
% description
{List of parallelization methods, e.g. ["omp","upc","mpi"] for OMP within cpu, UPC within node, and MPI between nodes.}
% example
{}
% default
{}

%--------------------------------------------------
\Parameter
% Group
{Parallel}
% subgroup
{balance}
% parameter
{frequency}
% type
{list(integer)}
% type restrictions
{}
% description
{List of load balancing timesteps for each level, e.g. [1,8,64]}
% example
{}
% default
{}

%--------------------------------------------------
\Parameter
% Group
{Parallel}
% subgroup
{balance}
% parameter
{method}
% type
{string}
% type restrictions
{}
% description
{Method to use for load balancing. Default is "none".}
% example
{}
% default
{}

%--------------------------------------------------
\Parameter
% Group
{Parallel}
% subgroup
{schedule}
% parameter
{method}
% type
{string}
% type restrictions
{}
% description
{Method for scheduling parallel tasks}
% example
{}
% default
{}

%--------------------------------------------------
\Parameter
% Group
{Parallel}
% subgroup
{mpi}
% parameter
{sided}
% type
{integer}
% type restrictions
{}
% description
{Type of MPI: 1 (single-sided get/put) or 2 (double-sided send/receive). \incomplete{what about multiple mpi levels?}}
% example
{}
% default
{}

%--------------------------------------------------
\Parameter
% Group
{Parallel}
% subgroup
{mpi}
% parameter
{send\_type}
% type
{list(string)}
% type restrictions
{ \code{"immediate"} $|$ \code{"blocking"} $|$ \code{"standard"} $|$ \code{"buffered"} $|$ \code{"synchronous"} $|$ \code{"ready"}}
% description
{List of MPI send variations}
% example
{}
% default
{}

%--------------------------------------------------
\Parameter
% Group
{Parallel}
% subgroup
{mpi}
% parameter
{recv\_type}
% type
{list(string)} 	
% type restrictions
{\code{"immediate"} $|$ \code{"blocking"}}
% description
{List [sic] of MPI recv variations.} 
% example
{}
% default
{}

%--------------------------------------------------
\Parameter
% Group
{Parallel}
% subgroup
{mpi}
% parameter
{get\_type}
% type
{list(string)}
% type restrictions
{}
% description
{MPI get variations}
% example
{}
% default
{}

%--------------------------------------------------
\Parameter
% Group
{Parallel}
% subgroup
{mpi}
% parameter
{put\_type}
% type
{list(string)}
% type restrictions
{}
% description
{MPI put variations}
% example
{}
% default
{}

%--------------------------------------------------
\Parameter
% Group
{Parallel}
% subgroup
{mpi}
% parameter
{permute}
% type
{list(integer)}
% type restrictions
{}
% description
{Specification of reordering of MPI tasks. \incomplete{how to specify reordering?  automatic/dynamic reordering?}}
% example
{}
% default
{}

%-----------------------------------------------------------------------
\subsection{\code{Physics}}
%-----------------------------------------------------------------------

 	 	 	 	
\Parameter
% Group
{Physics}
% subgroup
{}
% parameter
{dimensions}
% type
{integer}
% type restrictions
{}
% description
{Number of physical dimensions in the problem, 1, 2, or 3}
% example
{}
% default
{}

\Parameter
% Group
{Physics}
% subgroup
{}
% parameter
{gamma}
% type
{scalar}
% type restrictions
{}
% description
{ideal gas law constant}
% example
{}
% default
{}

%--------------------------------------------------
\Parameter
% Group
{Physics}
% subgroup
{}
% parameter
{gravity}
% type
{logical}
% type restrictions
{}
% description
{Whether gravity is enabled}
% example
{}
% default
{}

%--------------------------------------------------
\Parameter
% Group
{Physics}
% subgroup
{}
% parameter
{gravity\_constant}
% type
{scalar}
% type restrictions
{}
% description
{``used only in SetMinimumSupport''}
% example
{}
% default
{}

%--------------------------------------------------
\Parameter
% Group
{Physics}
% subgroup
{}
% parameter
{multi\_species}
% type
{integer}
% type restrictions
{}
% description
{}
% example
{}
% default
{}

%--------------------------------------------------
\Parameter
% Group
{Physics}
% subgroup
{}
% parameter
{cosmology}
% type
{logical}
% type restrictions
{}
% description
{Turn on or off comoving coordinates}
% example
{}
% default
{}

%--------------------------------------------------
\Parameter
% Group
{Physics}
% subgroup
{cosmology}
% parameter
{comoving\_box\_size}
% type
{scalar}
% type restrictions
{}
% description
{Enzo CosmologyComovingBoxSize}
% example
{}
% default
{}
 	 	 	 	
%--------------------------------------------------
\Parameter
% Group
{Physics}
% subgroup
{cosmology}
% parameter
{initial\_redshift}
% type
{scalar}
% type restrictions
{}
% description
{Enzo CosmologyInitialRedshift}
% example
{}
% default
{}
 		 	 	
%--------------------------------------------------
\Parameter
% Group
{Physics}
% subgroup
{cosmology}
% parameter
{hubble\_constant\_now}
% type
{scalar}
% type restrictions
{}
% description
{Hubble constant}
% example
{}
% default
{}

%--------------------------------------------------
\Parameter
% Group
{Physics}
% subgroup
{cosmology}
% parameter
{omega\_lamda\_now}
% type
{scalar}
% type restrictions
{}
% description
{}
% example
{}
% default
{}

%--------------------------------------------------
\Parameter
% Group
{Physics}
% subgroup
{cosmology}
% parameter
{omega\_matter\_now}
% type
{scalar}
% type restrictions
{}
% description
{}
% example
{}
% default
{}

%--------------------------------------------------
\Parameter
% Group
{Physics}
% subgroup
{cosmology}
% parameter
{max\_expansion\_rate}
% type
{scalar}
% type restrictions
{}
% description
{}
% example
{}
% default
{}

%=======================================================================
\section{Test problems}
%=======================================================================
\subsection{Implosion}
\verbatiminput{../../input/implosion.in}

\subsection{Shock Pool}
\verbatiminput{../../input/shock-pool.in}


%=======================================================================
\section{Publishing \cello-derived Results}
%=======================================================================

Publications using results from \cello\ should include the following
acknowledgement:

\begin{quotation}
Calculations were performed using version \textit{version} of the
Cello astrophysics and cosmology application.  \cello\ was developed
by the Laboratory for Computational Astrophysics at the University of
California, San Diego.
\end{quotation}

% @REF@ Referenced in design/compiling.tex

The version can be obtained using \code{make show-version}.  


If the \code{S2PLOT} package was used, please provide the following
acknowledgement:

\begin{quotation}
  Three-dimensional visualisation was conducted with the S2PLOT
   progamming library.
\end{quotation}

and reference the following paper:

\begin{quotation}
  D.~G.~Barnes, C.~J.~Fluke, P.~D.~Bourke and O.~Parry, 2006, PASA, 23, 82.
\end{quotation}

\end{document}

%==================================================================

