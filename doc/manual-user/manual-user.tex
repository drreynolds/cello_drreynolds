%0       1         2         3         4         5         6         7         8
%2345678901234567890123456789012345678901234567890123456789012345678901234567890

%=======================================================================
\documentclass{article}
%=======================================================================

\usepackage{color}

% INCLUDE DEVELOPMENT TEXT

\newcommand{\devel}[1]{\textbf{#1}}

% EXCLUDE DEVELOPMENT TEXT

% \newcommand{\devel}[1]{}


%=======================================================================
% Document layout
%=======================================================================

\setlength{\topmargin}{0.0in}
\setlength{\oddsidemargin}{0.0in}
\setlength{\evensidemargin}{0.0in}
\setlength{\textwidth}{6.0in}
\setlength{\textheight}{9.0in}

%=======================================================================
% Packages
%=======================================================================

\usepackage{wasysym}
\usepackage{epsfig}
\usepackage{url}

%=======================================================================
% Commands
%=======================================================================

\newcommand{\cello}{\textsf{Cello}}
\newcommand{\enzo}{\textsf{Enzo}}
\newcommand{\lcaperf}{\textsf{lcaperf}}
\newcommand{\lcatest}{\textsf{lcatest}}

\newcommand{\code}[1]{\textsf{#1}}

\newcommand{\note}[1]{\devel{\eighthnote\ \textit{#1} \\}}
\newcommand{\pargraph}[1]{\devel{\P\ \textbf{#1} \\}}

\newcommand{\todo}{\devel{$\circ$}}
\newcommand{\done}{\devel{$\bullet$}}
\newcommand{\halfdone}{\devel{\textcolor{gray}{$\bullet$}}}

\newcommand{\PROJECT}{\cello}

\newcommand{\TITLE}[3]{
\title{ {\huge \PROJECT\ #1}  \\ \vspace{0.1in}
     {\small Document Version: \textbf{#3}} \vspace{-0.1in}
    }
\author{      #2 \\
        Laboratory for Computational Astrophysics\\
        University of California, San Diego}
\maketitle}

%=======================================================================


%=======================================================================

\begin{document}

%=======================================================================
\TITLE{\cello\ User Manual}{James Bordner}{$Rev$}
%=======================================================================

%=======================================================================
\section{Introduction} \label{s:intro}
%=======================================================================

%=======================================================================
\section{Parameters}
%=======================================================================


\newcommand{\Parameter}[6]
{\subsubsection{\code{#1}}
\begin{tabular}{|l|l|l|}  \hline & & \\ [-1.5ex]
   \textcolor{red}{\code{#1}} & 
   \textcolor{red}{\code{#2}} &
   \textcolor{red}{\textbf{\code{#3}}} \\ [1ex] \hline 
   \multicolumn{3}{|l|}{} \\ [-1.5ex]
   \multicolumn{3}{|l|}{\textcolor{magenta}{#4}  \textcolor{magenta}{ #5 } } \\     \multicolumn{3}{|l|}{} \\ [-1.5ex] \hline 
   \multicolumn{3}{|l|}{} \\ [-1.5ex]
  \multicolumn{3}{|l|}{\parbox[l]{5.75in}{\raggedright\textsf{\textit{\textcolor{blue}{#6}}}}} \\[-1.5ex]
   \multicolumn{3}{|l|}{} \\ \hline  
\end{tabular}}

\Parameter
{Boundary}
{[ lower $|$ upper $|$ $\emptyset$ ]}
{type}
{string $|$ list(string) }
{ [ \code{"reflecting"} $|$ \code{"inflow"}$|$ \code{"outflow"}$|$
  \code{"periodic"}$|$ \code{"dirichlet"}$|$ \code{"neumann"} ] }
{ Type of boundary condition.  Corresponds to \enzo\ parameters
  \code{LeftFaceBoundaryCondition} and
  \code{RightFaceBoundaryCondition}, whose allowed values are 0 = reflecting, 1 = outflow, 2 = inflow, 3 = periodic.}

\begin{tabbing}
xxx\=xxx\=xxx\=xxx\kill
\> \code{Boundary \{} \\
\>\>\code{\textbf{type} = "periodic";} \\
\>\>   \code{lower \{ } \\
\>\>\>      \code{\textbf{type} = [ "reflecting" , "reflecting" , "reflecting" ]} \\
\>\>   \code{\}} \\
\>\>   \code{upper \{ \textbf{type} = "reflecting"\}} \\
\>\code{\}}
\end{tabbing}


% Boundary 	[face] 	value 	scalar expression 	value for the boundary condition, which may be time-varying
% Boundary 	[face] 	file 	string 	file from which to read the boundary conditions
% 				
% Domain 		extent 	list 	extent of the domain, [xmin, xmax], [xmin, xmax, ymin, ymax], or [xmin, xmax, ymin, ymax, zmin, zmax]
% 				
% Field 		courant 	scalar 	Courant condition for fields
% Field 		fields 	list 	List of field names [implicit?] [optional (e.g. for ordering)?]
% Field 		groups 	list 	List of group names [implicit?] [optional?]
% Field 		precision 	string 	default precision for all fields
% Field 		alignment 	integer 	Alignment of start of each field's array in bytes
% Field 		padding 	integer 	Padding between field arrays in bytes
% Field 	field_name 	minimum 	list 	minimum value(s) and actions ["none", "assign", "warning", "error"]
% Field 	field_name 	maximum 	list 	maximum value(s) and actions ["none", "assign", "warning", "error"]
% Field 	field_name 	centering 	list 	list of logicals specifying whether the variable is centered on the corresponding axis
% Field 	field_name 	precision 	string 	precision of the field
% Field 	field_name 	groups 	list 	string list of group names [e.g. "color"]
% Field 	field_name 	alias 	list 	string list of other names this field goes by
% Initial 	field_name	value 	list 	list of scalar-expression / logical-expression pairs
% Initial 	field_name	file 	string 	file containing the initial conditions for the field
% Initial 		time 	scalar 	initial time in code units
% Initial 		redshift 	scalar 	initial redshift for cosmological runs
% Initial 		cycle 	list 	initial cycle number, or list of cycle numbers for each AMR level
% 				
% Memory 		limit 	integer 	Set limit on the number of local bytes available for dynamic memory
% Memory 		warning 	scalar 	Set efficiency threshold before displaying a warning message
% Memory 		fill 	list 	List of strings to fill memory after being allocated, and before being deallocated
% Memory 		tracking 	logical 	Set memory tracking on or off. Turning off resorts to native new[]/delete[].'
% 				
% Mesh 		root_size 	list 	size of the root grid, e.g. [400,400]
% Mesh 		max_levels 	integer 	maximum levels (with root = 1), assuming refinement by two
% Mesh 		refinement 	integer 	refinement factor = 2, 4, etc.
% Mesh 		balanced 	logical 	whether the tree is balanced or "full" (ala Flash) or not
% Mesh 		backfill 	logical 	whether to backfill for refine > 2
% Mesh 		coalesce 	logical 	whether to coalesce small patches into one big one
% Mesh 		patch_size 	list 	minimum and maximum allowed patch sizes
% Mesh 		block_size 	list 	minimum and maximum allowed block sizes
% 				
% Method 		sequence 	list 	sequence of numerical methods to apply
% Method 	ppm 	diffusion 	logical 	PPM diffusion parameter
% Method 	ppm 	flattening 	logical 	PPM flattening parameter
% Method 	ppm 	steepening 	logical 	PPM steepening parameter
% Method 	ppm 	dual_energy 	logical 	whether to use dual-energy formalism
% Method 	ppm 	dual_energy_eta_1 	scalar 	dual-energy formalism parameter
% Method 	ppm 	dual_energy_eta_2 	scalar 	dual-energy formalism parameter
% Method 	ppm 	pressure_free 	logical 	"pressure free flag"
% 				
% Output 	file_name 	name 	list 	name for the output files, where the first element is a format string (ala printf()), and remaining elements are scalar expressions. E.g. ["wave_pool-t=%3.1f.data", t]
% Output 	file_name	times 	list 	start, stop, and interval times to output
% Output 	file_name	redshifts 	list 	start, stop, and interval redshifts to output
% Output 	file_name	cycles 	list 	start, stop, and interval cycles to output [what about fine vs coarse?]
% 				
% Parallel 	balance 	frequency 	list 	list of load balancing timesteps for each level, e.g. [1,8,64]
% Parallel 	balance 	method 	string 	method to use for load balancing. Default is "none".
% Parallel 	schedule 	method 	string 	method for scheduling parallel tasks
% Parallel 		levels 	list 	E.g. [8,32] for 8 cores/cpu, 32 cpu's/node, and any number of nodes
% Parallel 		type 	list 	E.g. ["omp","upc","mpi"] for OMP within cpu, UPC within node, and MPI between nodes
% Parallel 	mpi 	sided 	integer 	Type of MPI: 1 (single-sided get/put) or 2 (double-sided send/receive)
% Parallel 	mpi 	send_type 	list 	List of MPI send variations: "immediate", "blocking", "standard","buffered","synchronous","ready"
% Parallel 	mpi 	recv_type 	list 	List [sic] of MPI recv variations: immediate", "blocking"
% Parallel 	mpi 	get_type 	string 	MPI get variations (...)
% Parallel 	mpi 	put_type 	string 	MPI put variations (...)
% Parallel 	mpi 	permute 	list 	Specification of reordering of MPI tasks
% 				
% Physics 		gamma 	scalar 	ideal gas law constant
% Physics 		cosmology 	logical 	turn on or off comoving coordinates
% Physics 	cosmology	hubble_constant_now 	scalar 	
% Physics 	cosmology	omega_lamda_now 	scalar 	
% Physics 	cosmology	omega_matter_now 	scalar 	
% Physics 	cosmology	max_expansion_rate 	scalar 	
% Physics 		gravity 	logical 	whether gravity is included
% Physics 	gravity	constant 	scalar 	"used only in SetMinimumSupport"
% Physics 		multi_species 	integer 	
% 				
% Stopping 		time 	scalar 	stopping time
% Stopping 		redshift 	scalar 	stopping redshift
% Stopping 		cycle 	list 	stopping cycle, or list of stopping cycles for AMR levels

%=======================================================================
\section{Publishing \cello-derived Results}
%=======================================================================

Publications using results from \cello\ should include the following
acknowledgement:

\begin{quotation}
Calculations were performed using version \textit{version} of the
Cello astrophysics and cosmology application.  \cello\ was developed
by the Laboratory for Computational Astrophysics at the University of
California, San Diego.
\end{quotation}

% @REF@ Referenced in design/compiling.tex

The version can be obtained using \code{make show-version}.  


If the \code{S2PLOT} package was used, please provide the following
acknowledgement:

\begin{quotation}
  Three-dimensional visualisation was conducted with the S2PLOT
   progamming library.
\end{quotation}

and reference the following paper:

\begin{quotation}
  D.~G.~Barnes, C.~J.~Fluke, P.~D.~Bourke and O.~Parry, 2006, PASA, 23, 82.
\end{quotation}

\end{document}

%==================================================================

