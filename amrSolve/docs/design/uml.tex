% \newcommand{\includepsgrid}{\psgrid}
\newcommand{\includepsgrid}{}

%=======================================================================
% Point
%=======================================================================

   \newcommand{\drawClassPoint}{%
      \umlClass[umlShadow=false]{Point}{%
         \hline
         Point () \\
         get () \\
         set () \\
	 operator () () \\
         operator <= () \\
         sort ()%
   }}

   \newcommand{\umlPoint}{
   \begin{center}
   \begin{pspicture}(3,3.5)\includepsgrid
     \rput[lb](0,0){\rnode{ClassPoint}{\drawClassPoint}}
   \end{pspicture}
   \end{center}}

%=======================================================================
% Grid
%=======================================================================

   \newcommand{\drawClassGrid}{%
      \umlClass[umlShadow=false]{Grid}{%
         \hline
         Grid () \\
         \textit{attach} () \\
         \textit{release} () \\
         size() \\
         range() \\
%         parent() \\
%         numNeighbors () \\
%         neighbor () \\
%         numChildren () \\
%         child () \\
         isNeighbor() \\
         isLocal () \\
         print () \\
         geomview ()%
   }}

   \newcommand{\umlGrid}{
   \begin{center}
   \begin{pspicture}(3,5)\includepsgrid
     \rput[lb](0,0){\rnode{ClassGrid}{\drawClassGrid}}
   \end{pspicture}
   \end{center}}

%=======================================================================
% Level
%=======================================================================

   \newcommand{\drawClassLevel}{%
      \umlClass[umlShadow=false]{Level}{%
         \hline
         Level () \\
         \textit{attach} () \\
         \textit{release} () \\
         numGrids() \\
         grid() \\
         assertNeighbors() \\
         geomview ()%
   }}

   \newcommand{\umlLevel}{
   \begin{center}
   \begin{pspicture}(4,4)\includepsgrid
     \rput[lb](0,0){\rnode{ClassLevel}{\drawClassLevel}}
   \end{pspicture}
   \end{center}}

%=======================================================================
% Hierarchy
%=======================================================================

   \newcommand{\drawClassHierarchy}{%
      \umlClass[umlShadow=false]{Hierarchy}{%
         \hline
         Hierarchy () \\
         \textit{attach} () \\
         \textit{release} () \\
         numLevels() \\
         level() \\
         assertParent() \\
         geomview ()%
   }}

   \newcommand{\umlHierarchy}{
   \begin{center}
   \begin{pspicture}(3,4)\includepsgrid
     \rput[lb](0,0){\rnode{ClassHierarchy}{\drawClassHierarchy}}
   \end{pspicture}
   \end{center}}

%=======================================================================
% Iterator
%=======================================================================

   \newcommand{\drawClassIterator}{%
      \umlClass[umlShadow=false]{\textit{Iterator}}{%
         \hline
         Iterator () \\
         operator ++() \\
         operator * () \\
         operator begin () \\
         operator end () \\
         operator reset ()%
   }}

   \newcommand{\drawClassItGrids}{%
      \umlClass[umlShadow=false]{ItGrids}{%
   }}

   \newcommand{\drawClassItLevels}{%
      \umlClass[umlShadow=false]{ItLevels}{%
   }}

   \newcommand{\drawClassItChildren}{%
      \umlClass[umlShadow=false]{ItChildren}{%
   }}

   \newcommand{\drawClassItNeighbors}{%
      \umlClass[umlShadow=false]{ItNeighbors}{%
   }}

   \newcommand{\umlIterator}{%
   \begin{center}
   \begin{pspicture}(8,3.5)\includepsgrid
     \rput[lb](0,0){\rnode{ClassIterator}{\drawClassIterator}}
     \rput[lb](5.5,3){\rnode{ClassItLevels}{\drawClassItLevels}}
     \rput[lb](5.5,2){\rnode{ClassItGrids}{\drawClassItGrids}}
     \rput[lb](5.5,1){\rnode{ClassItChildren}{\drawClassItChildren}}
     \rput[lb](5.5,0){\rnode{ClassItNeighbors}{\drawClassItNeighbors}}
     \pnode(4,0.5){pnode1}
   \end{pspicture}
   \ncEVE[armA=1]{ClassIterator}{ClassItLevels}
   \ncEVE[armA=1]{ClassIterator}{ClassItGrids}
   \ncEVE[armA=1]{ClassIterator}{ClassItChildren}
   \ncEVE[armA=1]{ClassIterator}{ClassItNeighbors}
   \ncputicon{umlHerit}
   \end{center}}


%=======================================================================
% Vector
%=======================================================================

   \newcommand{\drawClassVector}{%
      \umlClass[umlShadow=false]{Vector}{%
         \hline
         Vector () \\
         size() \\
         dot() \\
         zaxpy() \\
         clear() \\
         copy() \\
         refresh()%
   }}

   \newcommand{\umlVector}{
   \begin{center}
   \begin{pspicture}(2,4)\includepsgrid
     \rput[lb](0,0){\rnode{ClassVector}{\drawClassVector}}
   \end{pspicture}
   \end{center}}

%=======================================================================
% Matrix
%=======================================================================

   \newcommand{\drawClassMatrix}{%
      \umlClass[umlShadow=false]{Matrix}{%
         \hline
         Matrix () \\
         size() \\
         matvec() \\
         residual()%
   }}

   \newcommand{\umlMatrix}{
   \begin{center}
   \begin{pspicture}(2.5,2.5)\includepsgrid
     \rput[lb](0,0){\rnode{ClassMatrix}{\drawClassMatrix}}
   \end{pspicture}
   \end{center}}

%=======================================================================
% Solver
%=======================================================================

   \newcommand{\drawClassSolver}{%
      \umlClass[umlShadow=false]{Solver}{%
         \hline
         Solver () \\
         apply() \\
         status()%
   }}

   \newcommand{\umlSolver}{
   \begin{center}
   \begin{pspicture}(2,2)\includepsgrid
     \rput[lb](0,0){\rnode{ClassSolver}{\drawClassSolver}}
   \end{pspicture}
   \end{center}}
